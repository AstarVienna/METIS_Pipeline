\subsection{Overscan}
\label{overscan:main}

The overscan (or prescan) region on a CCD can consist of physical
pixels on the detector that are not illuminated, or it can consist of a set
of virtual pixels created by reading out the serial register either before or
after transferring the charge from the CCD for each column/row. 
A detector may have multiple regions (amplifiers) and each one usually has 
an associated overscan region.

This module is intended to be applied to a single overscan region.

In HDRL the overscan correction value for each row/column of pixels is computed
from a rectangular running sub-region of the full overscan region of the
detector.

A small sub-region will increase the uncertainty of the correction but can
account for rapid spatial changes in the bias level, while a large sub-region
will decrease the uncertainty but smooth out spatial changes.

The actual correction value can be computed by estimating the
\textit{location/first moment} of the pixel value distribution within
the sub-region.  Several estimation methods, like mean, median, mode
or sigma/minmax-clipped\footnote{For convenience we denote the minmax
clipping algorithm and the kappa sigma clipping algorithm as
sigma/minmax-clipped whenever the statement applies to both algorithm}
mean, are available.

A typical overscan correction may be essentially independent of row/column,
or it may vary smoothly or with abrupt gradient changes. Hence the 
implementation of a location estimation method as opposed to the fitting of an
analytical function.

When choosing the overscan region, note that the first few rows/columns
that are read out after the image area will have an artificially high bias 
level due to the charge transfer efficiency not being 100\%. It is 
recommended therefore not to include the 2-3 rows/columns right next 
to the image region in the overscan region.

An interesting phenomenon is that a very bright star/object may increase the
bias level for the few rows/columns that it covers. Hence it is not advisable
to set the size of the sub-region to a value that is much larger than the image
point-spread-function FWHM.

This module allows the computation of the overscan correction for an image 
from a predefined overscan region. It also provides a function that applies 
the overscan correction to the image, using the result of the overscan 
computation.

\subsubsection{Overscan Computation}
\label{overscan:computation}

The Overscan is computed using the following function call:

\begin{lstlisting}
hdrl_overscan_compute_result * oscan_result = hdrl_overscan_compute(
        const cpl_image         *   source,
        const hdrl_parameter    *   params) ;
\end{lstlisting}

\paragraph{Inputs}

The input parameters that need to be passed to the function are created with:

\begin{lstlisting}
hdrl_parameter * params = 
    hdrl_overscan_parameter_create(hdrl_direction  correction_direction,
                                   double          ccd_ron,
                                   int             box_hsize,
                                   hdrl_parameter  collapse,
                                   hdrl_parameter  rect_region) ;
\end{lstlisting}

The collapsing parameters are created with one of the following function calls:

\begin{lstlisting}
hdrl_parameter * collapse = 
    hdrl_collapse_mean_parameter_create(void) ;
    hdrl_collapse_median_parameter_create(void) ;
    hdrl_collapse_weighted_mean_parameter_create(void) ;
    hdrl_collapse_sigclip_parameter_create(double      kappa_low,
                                           double      kappa_high,
                                           int         niter) ;
                                           
    hdrl_collapse_minmax_parameter_create(double       nlow,
                                          double       nhigh) ;
    
    hdrl_collapse_mode_parameter_create(double         histo_min,
                                        double         histo_max,
                                        double         bin_size,
                                        hdrl_mode_type mode_method,
                                        cpl_size       error_niter) ;
\end{lstlisting}

The Overscan region specification is done with:
\begin{lstlisting}
hdrl_parameter * rect_region =
    hdrl_rect_region_parameter_create(cpl_size llx, 
                                      cpl_size lly,
                                      cpl_size urx, 
                                      cpl_size ury) ;
\end{lstlisting}

\paragraph{Outputs}

The results of the Overscan computation (mostly 1D CPL or HDRL images of
type DOUBLE) are contained in a structure that can be accessed with the 
following functions:

\begin{lstlisting}
hdrl_image * correction =  
    hdrl_overscan_compute_result_get_correction(oscan_result) ;
cpl_image * contribution =
    hdrl_overscan_compute_result_get_contribution(oscan_result) ;
cpl_image * chi2 = 
    hdrl_overscan_compute_result_get_chi2(oscan_result) ;
cpl_image * red_chi2 = 
    hdrl_overscan_compute_result_get_red_chi2(oscan_result) ;
cpl_image * sigclip_reject_low = 
    hdrl_overscan_compute_result_get_sigclip_reject_low(oscan_result) ;
cpl_image * sigclip_reject_high = 
    hdrl_overscan_compute_result_get_sigclip_reject_high(oscan_result) ;
cpl_image * minmax_reject_low = 
    hdrl_overscan_compute_result_get_minmax_reject_low(oscan_result) ;
cpl_image * minmax_reject_high = 
    hdrl_overscan_compute_result_get_minmax_reject_high(oscan_result) ;
\end{lstlisting}

\paragraph{Algorithm}

Figure~\ref{fig:overscan_algo} gives a general description of the overscan 
computation method.
\putgraph{18}{overscan_computation_algorithm.png}{overscan_algo}{Overscan Computation Algorithm description}

The module extracts the pixel data for the overscan region from the
input image, including the bad pixel map. 

The source image may contain more than the overscan region that is actually 
needed by the computation.
\verb+rect_region+ defines the overscan region in the source image. The bad pixels 
that might be present in the image are taken into account (i.e. excluded 
from the computations). 

Each pixel of the resulting 1D images are computed from a 
running sub-window of the overscan region (see Figure~\ref{fig:overscan}). 
\verb+box_hsize+ defines the half size of the sub-window used for the 
computation (in the direction orthogonal to \verb+correction_direction+). 
If \verb+box_hsize+ value is \verb+HDRL_OVERSCAN_FULL_BOX+, the calculation is
done on the whole overscan region instead of a running sub-window. 
In this case, all pixels of the resulting 1D images will be identical.

The running sub-window defined by \verb+box_hsize+ is used to compute
the output pixel of the different 1D images.
When reaching the edges, the sub-window is shrinking such that the
current pixels stays in the center of the sub-window.

\verb+correction_direction+ is \verb+HDRL_X_AXIS+
(resp. \verb+HDRL_Y_AXIS+) if the overscan region has to be collapsed
along the X (resp. Y) axis in order to create the 1D resulting images
(correction, error, contribution, the $\chi^{2}$ (chi2) and the
reduced $\chi^{2}$ (\verb+red_chi2+), additionally
\verb+sigclip_reject_low+, \verb+_high+, and \verb+niter+ if the
collapsing method is sigma/minmax-clipping rejection as well as
\verb+histo_min+, \verb+histo_max+, \verb+bin_size+,
\verb+mode_method+, and \verb+error_niter+ if the collapsing method is
the mode).

The possible collapsing methods can be mean, median, weighted mean,
mode, sigma clipping, or minmax rejection. The appropriate function
needs to be used at the creation of the collapse parameters.

In case the collapsing method used is sigma-clipping, an iterative 
(niter iterations) rejection is applied in the sub-window before the 
computation of the results. \verb+kappa_low+ and \verb+kappa_high+ are used to
define what pixels need to be rejected.

The uncertainty on each pixel value is assumed to be \verb+ccd_ron+. 
\verb+ccd_ron+ is the CCD read-out noise. The parameter is mandatory, 
must be strictly non-negative. It is used for the error, the chi2 and 
the \verb+red_chi2+ computation.

The output \verb+hdrl_overscan_compute_result+ object contains the Overscan 
computation results that can be accessed with dedicated accessor
functions (see the 'Output' section):

correction is a 1D HDRL image of type double.

Its image part contains the Overscan correction values computed from the 
good pixels of the running sub-window (mean, weighted mean, median or mean 
after rejection, depending on the collapsing method used).

Its error part contains $\frac{ccd\_ron}{\sqrt{contribution}}$ for
mean, weighted mean, and sigma/minmax-clipping collapsing methods.  In
case of the median collapsing, it contains
$\sqrt{\frac{\pi}{2}}\times\frac{ccd\_ron}{\sqrt{contribution}}$ if
contribution is strictly greater than 2 pixels and
$\frac{ccd\_ron}{\sqrt{contribution}}$ when the contribution is 1 or 2
pixels.  The mode method is special in this case as it
calculates the error from the data. The calculation depends
on the parameter \verb+error_niter+. If the parameter is set to 0, it
is doing an analytically error estimation. If the parameter is larger
than 0, the error is calculated by a bootstrap Montecarlo simulation
from the input data with the value of the parameter specifying the
number of simulations. In this case the input data are perturbed with
the bootstrap technique and the mode is calculated \verb+error_niter+
times. From this modes the standard deviation is calculated and
returned as error.

\verb+contribution+ is a 1D CPL image of type integer.  It contains
the number of good pixels of the input running sub-window in case the
collapsing method is mean, weighted mean, mode or median, and the remaining
pixels after the rejection in case the sigma/minmax-clipping method is
used.

chi2 is a 1D CPL image of type double.
It contains $\sum_i{(\frac{source_i - correction}{ccd\_ron})^2}$
where {\it i} is running over the good pixels of the running sub-window.

\verb+red_chi2+ is a 1D CPL image of type double.
It contains the reduced $\chi^{2}$, i.e the $\chi^{2}$ divided by the number of
contributing pixels.

\verb+sigclip_reject_low+ and \verb+_high+ are 1D CPL images of type
double.  They are only returned in case the sigma/minmax-clipping
collapsing method is used.  They indicate the final thresholds of the
siglicp/minmax rejection.

\subsubsection{Overscan Correction}

The Overscan is corrected using the following function call:

\begin{lstlisting}

hdrl_overscan_correct_result * oscan_correct = hdrl_overscan_correct(
        const hdrl_image                    *   source,
        const hdrl_parameter                *   region,
        const hdrl_overscan_compute_result  *   oscan_result) ;
\end{lstlisting}

\paragraph{Inputs}

The \verb+oscan_result+ parameter is obtained by the Overscan computation 
(see ~\ref{overscan:computation}).

The \verb+region+ defines the region in source that needs to be corrected. 
It is created with:
\begin{lstlisting}
hdrl_parameter * region =
    hdrl_rect_region_parameter_create(cpl_size llx, 
                                      cpl_size lly,
                                      cpl_size urx, 
                                      cpl_size ury) ;
\end{lstlisting}

\paragraph{Outputs}

The results of the Overscan correction are contained in a structure that 
can be accessed with the following functions:

\begin{lstlisting}
hdrl_image * corrected =
    hdrl_overscan_correct_result_get_corrected(oscan_correct) ;
cpl_image * badmask =
    hdrl_overscan_correct_result_get_badmask(oscan_correct) ;
\end{lstlisting}

\verb+hdrl_overscan_correct_result_get_corrected+ returns a copy of the input
image with the data row or column wise corrected by the overscan values.

\verb+hdrl_overscan_correct_result_get_badmask+ returns an integer image with
all pixels considered bad by the algorithm set to 1 and all
good pixels set to 0.
Pixels of the input image will be considered bad if there is no overscan value
available to correct it (which is defined by the badpixel mask of the overscan
values image from \verb+hdrl_overscan_compute+)

\paragraph{Algorithm}

\verb+source+ is the input HDRL image that needs to be corrected. Usually the 
image part is the one passed to \verb+hdrl_overscan_compute()+ to compute the 
overscan correction parameters.

\verb+region+ specifies which region of the \verb+source+ image must be 
corrected. If NULL, the whole image is corrected.
The size of \verb+region+ must fit the contents of \verb+oscan_result+.

If the \verb+correction_direction+ value is \verb+HDRL_X_AXIS+ (resp.
\verb+DRL_Y_AXIS+), the correction will be applied row by row (resp. column by
column) to the input image, ie correction value number i will be subtracted to
row (resp. column) number i.

The error part of the source HDRL image is used for error propagation.
Uncertainties in the overscan computation result, if present, are added in
quadrature to the uncertainties on the input image pixel values. 

\verb+oscan_result+ contains all the parameters for the overscan correction. 
It has been produced by \\
\verb+hdrl_overscan_compute()+.

The output \verb+hdrl_overscan_correct_result+ object contains the following 
members:

corrected is a HDRL image of type double of the same size as source.

Its image part had all its pixels within the specified region subtracted the 
proper correction. The pixels outside the specified region remain unchanged.

Its error part had all the pixels within the specified region set to 
$\sqrt{overscan\_computation.error^{2} + source\_error^{2}}$, which is the 
standard linear error propagation.
The pixels outside the specified region remain unchanged.

baddmask is a CPL image identifying the bad pixels.

\subsubsection{Examples}

\paragraph{Overscan Computation}

In the following example, we are using a UVES image whose left 24 pixel 
columns are used for the overscan computation. The image size is 1074x1500.

\putgraph{15}{overscan_example.png}{overscan}{Overscan Computation Example}

The Overscan region is defined with:
\begin{lstlisting}
hdrl_parameter * os_region = 
    hdrl_rect_region_parameter_create(1, 1, 24, 1500) ;
\end{lstlisting}

We would like for the collapsing method to use a sigma-clipping algorithm 
with 5 iterations and with 3.0 for the low and high rejection kappa values:
\begin{lstlisting}
hdrl_parameter * os_collapse = 
    hdrl_collapse_sigclip_parameter_create(3.0, 3.0, 5) ;
\end{lstlisting}

The specified Overscan region needs to be collapsed along the X axis, the
CCD RON is 10.0, and the running box half size is 5 pixels.

In these conditions, the Overscan computation parameter is created with:
\begin{lstlisting}
hdrl_parameter * os_params = 
    hdrl_overscan_parameter_create(HDRL_X_AXIS, 10.0, 5, 
                                    os_collapse, os_region);
\end{lstlisting}

The actual computation occurs here:
\begin{lstlisting}
hdrl_overscan_compute_result * os_computation = 
    hdrl_overscan_compute(image, os_params);
\end{lstlisting}

Let's not destroy \verb+os_computation+ as we need it to apply the correction
in the next section. What we do not need any more are the parameters. They are
deleted with:
\begin{lstlisting}
hdrl_parameter_delete(os_region) ;
hdrl_parameter_delete(os_collapse) ;
hdrl_parameter_delete(os_params) ;
\end{lstlisting}

For convenience, a \verb+_destroy+ function is also provided, that
deletes the parameter and recursively deletes the contained parameters: 
\begin{lstlisting}
hdrl_parameter_destroy(os_params) ;
\end{lstlisting}

\paragraph{Overscan Correction} 

The Overscan Computation result needs now to be used to correct the input
image. We only want to correct the non-overscan part of the image.

In order to do that, we specify the image region we want the correction to 
be applied on:
\begin{lstlisting}
hdrl_parameter * apply_region = 
    hdrl_rect_region_parameter_create(25, 1, 1074, 1500) ;
\end{lstlisting}

The wished bad pixel mask code is 2, the correction function call is:
\begin{lstlisting}
hdrl_overscan_correct_result * os_corrected = 
    hdrl_overscan_correct(image, apply_region, 2, os_computation) ;
\end{lstlisting}

A number of accessor function give access to the elements of the
\verb+os_corrected+ result object.

Finally, the used parameters, or the created objects need to be deleted:
\begin{lstlisting}
hdrl_overscan_compute_result_delete(os_computation) ;
hdrl_parameter_delete(apply_region) ;
hdrl_overscan_correct_result_delete(os_corrected) ;
\end{lstlisting}

\subsubsection{Overscan Parameters in a Recipe} 
\label{sec:algorithms:overscan:parameters}

In the context of a recipe, all input parameters needed by the Overscan 
computation may be provided as input parameters. There is a convenient
interface for doing this. This avoids to update all recipes whenever a
new parameter is offered by the Overscan computation.

A function call allows to generate a parameter list, another one allows
to parse it in order to generate the input Overscan computation
parameters:

\begin{lstlisting}

/* Create whished default Overscan Computation parameters */
 hdrl_parameter * rect_region_def =
     hdrl_rect_region_parameter_create(1, 1, 20, 1000);
 hdrl_parameter * sigclip_def =
     hdrl_collapse_sigclip_parameter_create(3., 3., 5);
 hdrl_parameter * minmax_def =
     hdrl_collapse_minmax_parameter_create(1., 1.);
 hdrl_parameter * mode_def =
     hdrl_collapse_mode_parameter_create(10., 1., 0., HDRL_MODE_MEDIAN, 0);
 cpl_parameterlist * os_comp = hdrl_overscan_parameter_create_parlist(
             RECIPE_NAME, "", "alongX", 0, 0., rect_region_def, "MEDIAN",
             sigclip_def, minmax_def, mode_def);
 hdrl_parameter_delete(rect_region_def);
 hdrl_parameter_delete(sigclip_def);
 hdrl_parameter_delete(minmax_def);
 hdrl_parameter_delete(mode_def);
 for (cpl_parameter * p = cpl_parameterlist_get_first(os_comp) ; 
         p != NULL; p = cpl_parameterlist_get_next(os_comp)) 
     cpl_parameterlist_append(self, cpl_parameter_duplicate(p));
 cpl_parameterlist_delete(os_comp);

\end{lstlisting}

The created parameter list \verb+os_comp_parlist+ simply needs to be fully or
partly appended to the recipe parameter list to make those parameters 
available to the user:
{\footnotesize 
\begin{verbatim}
esorex --man-page hdrldemo_bias
...
--oscan.correction-direction        : Correction Direction. <alongX | alongY> [alongX]
--oscan.box-hsize                   : Half size of running box in pixel
--oscan.ccd-ron                     : Readout noise in ADU. [10.0]
--oscan.calc-llx                    : Lower left x pos. (FITS) defining the region. [1]
--oscan.calc-lly                    : Lower left y pos. (FITS) defining the region. [1]
--oscan.calc-urx                    : Upper right x pos. (FITS) defining the region. [20]
--oscan.calc-ury                    : Upper right y pos. (FITS) defining the region. [0]
--oscan.collapse.method             : Method used for collapsing the data. [MEDIAN]
--oscan.collapse.sigclip.kappa-low  : Low kappa factor. [3.0]
--oscan.collapse.sigclip.kappa-high : High kappa factor. [3.0]
--oscan.collapse.sigclip.niter      : Maximum number of clipping iterations. [5]
...
\end{verbatim}
}
If parlist is the recipe parameters list, we also offer a function to 
generate the Overscan parameters directly from the recipe parameters:

\begin{lstlisting}
/* Parse the Overscan Parameters */
hdrl_parameter * os_params = 
    hdrl_overscan_parameter_parse_parlist(parlist, "hdrldemo_bias");
\end{lstlisting}

