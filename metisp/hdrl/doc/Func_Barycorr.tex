-----------------------------------------------------------------------------
\subsection{Barycentric correction}
\label{sec:algorithms:barycorr:main}

The algorithm derives the barycentric correction of an observation,
i.e. the wavelength shift to apply to a spectrum to compensate for the
motion of the observer with respect to the barycenter of the solar
system, by using the \href{https://github.com/liberfa/erfa}{ERFA}
(Essential Routines for Fundamental Astronomy) library. ERFA is a C
library containing key algorithms for astronomy, and is based on the
\href{http://www.iausofa.org}{SOFA library} published by the
International Astronomical Union (IAU). \\

\subsubsection{Algorithm - short description}
\label{sec:algorithms:barycorr:algo}

The implemented algorithm uses the ERFA function
\href{https://github.com/liberfa/erfa/blob/master/src/apco13.c}{eraApco13()}
to calculate the barycentric correction of an observation. For details
on the implemented low level algorithm please visit the above
mentioned web pages and links therein.

A comparison between the algorithm implemented in the \HDRL and the
one implemented in the ESPRESSO pipeline shows a very good agreement.
For this about 7000 IDPs from 2021 were re-analyzed with the \HDRL
implementation and the differences read as follows:
\begin{itemize}
    \item Mean difference : 0.036 m/s
    \item Median difference: 0.055 m/s
    \item Standard deviation: 0.281 m/s
    \item MAD (median absolute deviation): 0.217 m/s
    \item Maximum deviation: 1.573 m/s
\end{itemize}

\subsubsection{Functions}
\label{sec:algorithms:barycorr:functions}

The \HDRL implements the following function to derive the barycentric correction:
\begin{lstlisting}
cpl_error_code hdrl_barycorr_compute(
        double            ra,
        double            dec,
        const cpl_table * eop_table,
        double            mjdobs,
        double            time_to_mid_exposure,
        double            longitude,
        double            latitude,
        double            elevation,
        double            pressure,
        double            temperature,
        double            humidity,
        double            wavelength ,
        double          * barycorr);
\end{lstlisting}

\subsubsection{Inputs}
\label{sec:algorithms:barycorr:inputs}

The different \textbf{input} function parameters are described as follows:
\begin{itemize}\itemsep-1pt \parskip2pt \parsep2pt
\item \verb+ra+:                     Target right ascension (J2000) in [deg]
\item \verb+dec+:                    Target declination (J2000) in [deg]
\item \verb+eop_table+:              cpl\_table containing the earth orientation parameter
\item \verb+mjdobs+:                 Start of observation in [days]
\item \verb+time_to_mid_exposure+:   Time to mid exposure, e.g. EXPTIME/2. in [s]
\item \verb+longitude+:              Telescope geodetic longitude (+ = East ) in [deg]
\item \verb+latitude+:               Telescope geodetic latitude  (+ = North) in [deg]
\item \verb+elevation+:              Telescope elevation above sea level in [m]
\item \verb+pressure+:               Pressure at the observer in [hPa == mbar]
\item \verb+temperature+:            Ambient temperature at the observer in [deg C]
\item \verb+humidity+:               Relative humidity at the observer [range 0 - 1]
\item \verb+wavelength+:             Observing wavelength in [micrometer]
\end{itemize}


Please note that the pressure, temperature, humidity, and the
wavelength parameters are only tested with a value of 0. No tests with
other values were performed.

The \verb+eop_table+ contains the
\href{https://www.iers.org/IERS/EN/Science/EarthRotation/EOP.html}{Earth
  Orientation Parameter} as a function of time (\verb+MJD-OBS+) and
can be downloaded from
\href{https://www.iers.org/IERS/EN/Home/home\_node.html}{IERS}. For
this purpose the \HDRL offers the 2 functions
\\ \verb+hdrl_download_url_to_buffer()+ and
\verb+hdrl_eop_data_totable()+. The first function allows you to
download the full EOP table from IERS and the second function extracts
and converts the relevant information into a fits table. The latter
contains the appropriate columns and can be passed to the\\
\verb+hdrl_barycorr_compute()+ function.

\begin{lstlisting}
char * hdrl_download_url_to_buffer(
       const char * url,
       size_t     * data_length);  
\end{lstlisting}
This function allows you to download a url into a c data buffer with
\verb+url+ being the url to download from. The length of the buffer
is stored in the function parameter \verb+data_length+. The url to
download the EOP data should
be \verb+https://datacenter.iers.org/products/eop/rapid/standard/finals2000A.data+

Please note, that this function is not threadsafe, to the extent that
it may only be called from the main thread, with no other threads
running. So as long as esorex or alike is not using threads it still
may be called from within a recipe before the the recipe itself, or
hdrl launches any additional threads.

\begin{lstlisting}
cpl_table * hdrl_eop_data_totable (
            const char * eop_data,
            cpl_size     data_length);
\end{lstlisting}
From the above listed
url\footnote{\url{https://datacenter.iers.org/products/eop/rapid/standard/finals2000A.data}},
this function extracts the relevant information from the c data buffer
and converts it into a fitsfile.  In this function, the
\verb+eop_data+ is the downloaded data buffer and \verb+data_length+
the length of the buffer.

As the time resolution of the EOP table is only one day, the
parameters are linearly interpolated to have the most accurate values
at the time of observation.

\subsubsection{Outputs}
\label{sec:algorithms:barycorr:outputs}

The \textbf{computed barycentric correction} is stored in the function
parameter \verb+barycorr+ with the units \verb+[m/s]+.

\subsubsection{Example}
The following code is a skeleton example on how to use HDRL to compute
the barycentric correction of an observation. For a detailed
implementation please see the esotk\_barycorr recipe for the
barycentric correction computation and the esotk\_eop recipe for
the EOP table creation.

\begin{lstlisting}
/* Load EOP file - it was created by another dedicated recipe */
/* For detail on how to create the EOP file see the esotk_eop recipe */

cpl_table * eop_table =
            cpl_table_load(cpl_frame_get_filename(eop_frame), 1, 0);

/* Read the primary header */
 cpl_propertylist * header =
            cpl_propertylist_load(cpl_frame_get_filename(cur_frame), 0);
 
 double ra        = cpl_propertylist_get_double(header, "RA");
 double dec       = cpl_propertylist_get_double(header, "DEC");
 double mjdobs    = cpl_propertylist_get_double(header, "MJD-OBS");
 double exptime   = cpl_propertylist_get_double(header, "EXPTIME");
 double longitude = cpl_propertylist_get_double(header, "ESO TEL GEOLON");
 double latitude  = cpl_propertylist_get_double(header, "ESO TEL GEOLAT");
 double elevation = cpl_propertylist_get_double(header, "ESO TEL GEOELEV");
 double pressure    = 0.0
 double temperature = 0.0
 double humidity    = 0.0
 double wavelength  = 0.0

 double barycorr = 0;
 hdrl_barycorr_compute(ra, dec, eop_table, mjdobs, exptime/2.,
                       longitude,  latitude, elevation,
                       pressure, temperature, humidity, wavelength,
                       &barycorr);
                                 
 cpl_msg_info (cpl_func, "Barycentric correction: %g m/s", barycorr);
\end{lstlisting}
