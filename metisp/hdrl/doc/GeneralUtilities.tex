\subsection{General Objects and Utilities}

\subsubsection{Image Region}

hdrl\_rect\_t is a structure containing 4 numbers (llx, lly, urx, ury)
defining an image rectangular region. Coordinates are following the FITS
convention, i.e start with 1.

\subsubsection{Direction}

hdrl\_direction can be used to specify the 2 images directions
HDRL\_X\_AXIS and HDRL\_Y\_AXIS corresponding to the FITS directions
NAXIS1 and NAXIS2.


\subsubsection{Sigma-Clipping}
\label{sigma-clipping:algo}

\begin{comment}

\begin{verbatim}
cpl_error_code hdrl_kappa_sigma_clip(
        const cpl_vector  * vec,
        const cpl_vector  * vec_err,
        const double        kappa_low,
        const double        kappa_high,
        const int           iter,
        double            * mean_ks,
        double            * mean_ks_err,
        cpl_size          * naccepted,
        double            * reject_low,
        double            * reject_high)
\end{verbatim}

\paragraph{Inputs}
\paragraph{Outputs}
\paragraph{Algorithm}
\end{comment}

The sigma-clipping is applied on a vector data ({\it vec}). An error
vector ({\it vec\_err}) is also passed for the error computation.

An iterative process of rejection of the outlier elements of {\it vec} is 
applied. {\it iter} specifies the maximum number of iterations.
At each iteration, the median and sigma values of the vector are computed and 
used to derive low and high thresholds ($median-kappa\_low \times sigma$ 
and $median+kappa\_low \times sigma$). The values of {\it vec} outside those
bounds are rejected and the remaining values are passed to the next
iteration.

The sigma-clipping function produces the following quantities: {\it mean\_ks}, 
{\it mean\_ks\_err}, {\it naccepted}, {\it reject\_low} and {\it reject\_high}.

The mean value of the remaining elements is stored into {\it mean\_ks}.
{\it mean\_ks\_err} contains $\frac{\sum_i{val_i^{2}}}{N}$ where $val_i$
are the remaining elements of {\it vec\_err} and $N$ the number of
those elements. The $N$ value is stored in {\it naccepted}.

{\it reject\_low} and {\it reject\_high} are the final thresholds
differentiating the rejected pixels from the others.

The iterative process is illustrated in Figure~\ref{fig:sigclip}.

\putgraph{18}{sigclip_algorithm.png}{sigclip}{Sigma-Clipping Algorithm description}
  
Note that the $\sigma$ used for the thresholding in the different iterations 
is not the square root of the variance as one would expect, but an 
estimation computed using the interquartile range (IQR) of the distribution. 
More precisely, $\sigma = \frac{IQR}{1.349}$ 
  
This estimation of the standard deviation is only robust if the
values obey a gaussian distribution. 

As illustrated in figure~\ref{fig:iqr}, the IQR is the distance between the 
upper and the lower quartiles of a distribution.

\putgraph{10}{iqr.png}{iqr}{Interquartile Range (IQR) description}

This estimation of the $\sigma$ tends to be better if the number of samples 
is not too small, and if the distribution is gaussian. The plot in
Figure~\ref{fig:iqrvsvariance} shows that for a low number of samples, 
the error on the sigma estimation using the IQR method can be up to 30\%.

\putgraph{10}{iqr_vs_variance.png}{iqrvsvariance}{IQR estimation error}


However, an error of 30\% in the sigma estimation would only have 
limited consequences as this would only impact the threshold for the good 
pixels selections. The error on the final result would naturally be 
drastically lower.

