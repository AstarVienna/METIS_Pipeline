\section{Software Infrastructure}

\subsection{SVN usage and library integration}
\label{library_integration}

Th \HDRL library should be included into the pipeline library during
the svn checkout by using the svn::external concept. The external
should point to a given library release in the tags and not to the
trunk, i.e. to a sub-directory of
\verb,http://svnhq2.hq.eso.org/p2/tags/Pipelines/common/hdrl, (see
also sect.~\ref{SoftwareInfrastructure:Releases}). In order to build
the library one has to add the \textit{hdrl.m4} from the common
\textit{m4macros} repository (\verb,Pipelines/common/m4macros, in SVN) to the
pipeline sources.
Then one has to modify the \textit{configure.ac} and the top-level
\textit{Makefile.am} as follows:
\begin{itemize}
\item \textit{configure.ac}: 
\begin{verbatim}
# mark hdrl folder as containing a configurable package
AC_CONFIG_SUBDIRS([hdrl])

# check and define all required variables to build and
# link hdrl external located in the hdrl folder
HDRL_CHECK([hdrl])
\end{verbatim}
\item \textit{Makefile.am}:
\begin{verbatim}
SUBDIRS = hdrl ...
\end{verbatim}
\end{itemize}
Moreover, the variable
\verb,$(HDRL_LIBS), must also be added to the link variable,
\verb,$(HDRL_LDFLAGS), to the linker flag variable, and
\verb,$(HDRL_INCLUDES), to the \verb,AM_CPPFLAGS, variable in the
\textit{Makefile.am} of any folder making use of HDRL. As HDRL is
currently a static library it also needs has to be added as a
dependency of objects using it so these are relinked when HDRL
changes. For example:

\begin{verbatim}
hdrldemo_bias_la_LDFLAGS = $(HDRL_LDFLAGS) ...
hdrldemo_bias_la_LIBADD = $(HDRL_LIBS) ...
hdrldemo_bias_la_DEPENDENCY = $(LIBHDRL) ...
\end{verbatim}

In the source files, the only include needed is:

\begin{lstlisting}
#include <hdrl.h>
\end{lstlisting}



\begingroup
\let\cleardoublepage\relax
\let\clearpage\relax

\subsection{Releases}
\label{SoftwareInfrastructure:Releases}

There is no fixed release cycle for the \HDRL library (as e.g. for
CPL), but new releases are feature-driven, i.e. if there are new
functionality/algorithms available and carefully tested a new
release will be announced and the pipeline developer can change the
svn::external to this new release. This has the advantage that the
pipeline developer has more freedom to decide when to update the
pipeline. On the other hand it allows the developer to incorporate the
new \HDRL release to his pipeline on a very short timescale and
prepare a new Paranal/public release. This is only possible as the
library is not installed in the Data Flow System environment in
Paranal (like CPL) but delivered inside each pipeline.

\subsubsection{Release version 1.5.0}
The \HDRL release version 1.5.0 can be included from\\

http://svnhq2.hq.eso.org/p2/tags/Pipelines/common/hdrl/hdrl-1.5.0

In this release we updated and/or added the following algorithms:
\begin{itemize}
\item The computation of the barycenric correction, i.e. the
  wavelength shift to apply to a spectrum to compensate for the motion
  of the observer with respect to the barycenter of the solar system.
  The implemented algorithm derives the barycentric correction of an
  observation by using the
  \href{https://github.com/liberfa/erfa}{ERFA} (Essential Routines for
  Fundamental Astronomy) library. ERFA is a C library containing key
  algorithms for astronomy, and is based on the
  \href{http://www.iausofa.org}{SOFA library} published by the
  International Astronomical Union (IAU).  See
  section~\ref{sec:algorithms:barycorr:main} for detailed information.
\item The build system has been modified to include GSL as direct
   dependency to the hdrl unit-test as direct GSL calls are performed
   when unit-testing the limiting magnitude module.
 \item For the barycentric correction algorithm two additional
   dependencies were added to the \HDRL.
   \begin{itemize}
     \item A dependency to the
       \href{https://github.com/liberfa/erfa}{ERFA} (Essential
       Routines for Fundamental Astronomy) library
     \item A dependency to the
       \href{https://curl.se/libcurl/}{libcurl} (The multiprotocol
       file transfer library) library
   \end{itemize}  
\end{itemize}

\subsubsection{Release version 1.4.0}
The \HDRL release version 1.4.0 can be included from\\

http://svnhq2.hq.eso.org/p2/tags/Pipelines/common/hdrl/hdrl-1.4.0

In this release we updated and/or added the following algorithms:
\begin{itemize}
\item The computation of the limiting magnitude of an image as defined
  in the \textit{ESO Phase 3 Standard}. The limiting magnitude
  characterizes the depth of an observation and is defined as the
  magnitude of a unresolved source whose flux is 5 times the noise
  background, i.e. the magnitude of a point like source detected with
  $\frac{S}{N}=5$. See section~\ref{sec:algorithms:maglim:main} and
  appendix~\ref{chap:algorithms:maglim} for detailed information.
\item To the statistical estimators we added the \textbf{mode} of a
  distribution, i.e. the following algorithms are now supported for
  collapsing imagelists or deriving statistics on images:
  \begin{itemize}
  \item Mean
  \item Weighted mean
  \item min-max rejected mean
  \item $\kappa\sigma$~clipped mean
  \item Median
  \item Mode
  \end{itemize}
  Please note that all methods but the mode are doing \textbf{error
    propagation}. The mode method is special in this case as it
  \textbf{calculates the error from the data}. The error estimation
  can either be done analytically or based on a bootstrap Montecarlo
  simulation. In this case the input data are perturbed with the
  bootstrap technique and the mode is calculated N times (controlled
  with a parameter). From this N modes the standard deviation is
  calculated and returned as error. See
  section~\ref{sec:algorithms:robust_mean:mode} for detailed
  information on the mode algorithm.
\item Due to the addition of the mode, the
  functions\\ \verb+ hdrl_overscan_parameter_create_parlist()+
  and\\ \verb+hdrl_collapse_parameter_create_parlist()+ have
  changed. The two functions now require an additional default mode
  hdrl parameter. See the doxygen information and
  e.g. section~\ref{sec:algorithms:overscan:parameters} and
  section~\ref{sec:algorithms:bias:parameters} for more details.
\end{itemize}


\subsubsection{Release version 1.3.0}
The \HDRL release version 1.3.0 can be included from\\

http://svnhq2.hq.eso.org/p2/tags/Pipelines/common/hdrl/hdrl-1.3.0

In this release we updated and/or added the following algorithms:
\begin{itemize}
\item Resampling of 2-dimensional images and 3-dimensional cubes. A common
  problem in astronomy is the resampling of images (or cubes) onto a common
  grid. Ideally, this is done only once in the data reduction workflow as each
  sub pixel resampling redistributes the flux and leads to correlations. The
  algorithm provided by the \HDRL is doing the 2D and 3D interpolation in
  2-dimensional and 3-dimensional spaces, respectively. Currently there are six
  different interpolation methods implemented:
\begin{center}
\begin{itemize}
\item {\bf Nearest:} Nearest neighbour resampling
\item {\bf Linear:} Weighted resampling using an inverse distance weighting
  function
\item {\bf Quadratic:} Weighted resampling using a quadratic inverse distance
  weighting function
\item {\bf Renka:} Weighted resampling using a Renka weighting function
\item {\bf Drizzle:} Weighted resampling using a drizzle-like weighting scheme
\item {\bf Lanczos:} Weighted resampling using a lanczos-like restricted sinc as
  weighting function
\end{itemize}
\end{center}
  See section~\ref{resampling:main} and appendix~\ref{chap:assessing:resampling}
  for detailed information.
\item The object catalogue generation code (see section~\ref{catalogue:main})
  has been updated. In previous releases, pixels with a value of exactly 0 where
  automatically added to the confidence map as zero\footnote{``Bad'' pixels are
  given a confidence value of zero.} and excluded in all further
  computations. This has been removed.
\end{itemize}

\subsubsection{Release version 1.2.0}
The \HDRL release version 1.2.0 can be included from\\

http://svnhq2.hq.eso.org/p2/tags/Pipelines/common/hdrl/hdrl-1.2.0

In this release we updated and/or added the following algorithms:
\begin{itemize}
\item Detection of fixed pattern noise. A classical example is pick
  noise, i.e. low-amplitude, quasi-periodical patterns super-imposed
  on the normal read-noise. It is due to electronic interference and
  might show up or disappear on short timescales (days or hours). The
  algorithms tries to identify it by the usage of the power spectrum.
  See section~\ref{patternnoise:main} for detailed information.
\item An error in the documentation of the strehl ratio variable
  \verb+m1_radius+ and \verb+m2_radius+ was corrected
  (section~\ref{sec:algorithms:strehl:example}). The code was correct.
\item In the spectral efficiency computation
  (section~\ref{efficiency:main}) a sign error in the atmospheric
  correction (equation~\ref{eq:eff}) was corrected in the
  documentation as well as in the code.
\end{itemize}


\subsubsection{Release version 1.1.0}
The \HDRL release version 1.1.0 can be included from\\

http://svnhq2.hq.eso.org/p2/tags/Pipelines/common/hdrl/hdrl-1.1.0

In this release we added five new algorithms:
\begin{itemize}
\item Computation of the Strehl ratio. The Strehl ratio is defined as
  the ratio of the peak image intensity from a point source compared
  to the maximum attainable intensity using an ideal optical system
  limited only by diffraction over the telescope aperture. The Strehl
  ratio is very frequently used to perform the quality control of the
  scientific data obtained with the AO assisted instrumentation.
\item Computation of the \textit{spectral efficiency} as a function of
  wavelength: The efficiency is used to monitor the system performance
  and health. It is calculated from observing flux standard stars (in
  photometric conditions). Then, the observed 1D spectrum is compared
  with the reference spectrum, as it would be observed outside the
  Earth's atmosphere. The reference spectrum is provided by the user,
  usually via a catalog of standard stars. See
  section~\ref{efficiency:main} for detailed information.
\item Computation of the \textit{spectral response} as a function of
  wavelength: The algorithm is divided in two parts: \textit{Telluric
    correction} and \textit{Response calculation}. In the provided
  implementation the \textit{Telluric correction} is optional and can
  be disabled by the user. See section~\ref{response:main} for
  detailed information.
\item Computation of the \textit{Differential Atmospheric Refraction}
  as a function on wavelength. The differential atmospheric refraction
  is calculated according to the algorithm from Filippenko (1982,
  PASP, 94, 715) using the Owens formula which converts relative
  humidity in water vapor pressure. See section~\ref{refraction:main}
  for detailed information.
\item Computation of the \textit{effective air mass} of an
  observation.  See section~\ref{airmass:main} for detailed
  information.
\end{itemize}

 

\subsubsection{Release version 1.0.0}
The \HDRL release version 1.0.0 can be included from\\

http://svnhq2.hq.eso.org/p2/tags/Pipelines/common/hdrl/hdrl-1.0.0

In order to provide astrometric and photometric calibration
information, the \HDRL implements in this release a functionality to
generate a catalogue of detected objects (i.e. stars, galaxies).

A high-level summary of the implemented data reduction sequence is:
\begin{itemize}
\item estimate the local sky background over the image and track any
    variations at adequate resolution to eventually remove them,
\item detect objects/blends of objects and keep a list of pixels
    belonging to each blend for further analysis (see \cite{Irwin85}
    for details)
  \item parametrise the detected objects, i.e. perform astrometry,
    photometry and a shape analysis.
\end{itemize}

For details see section~\ref{catalogue:main}.

\subsubsection{Release version 0.3.0b1}
The \HDRL release version 0.3.0b1 can be included from

http://svnhq2.hq.eso.org/p2/tags/Pipelines/common/hdrl/hdrl-0.3.0b1

In this release we added an algorithm to do fringe correction. In a
first step the algorithm creates a master-fringe image using a
Gaussian mixture model. A properly scaled version of the
master-fringe image is then used to remove the fringes from the single
images (see section~\ref{fringe:main}).

\subsubsection{Release version 0.2.0}
The \HDRL release version 0.2.0 can be included from

http://svnhq2.hq.eso.org/p2/tags/Pipelines/common/hdrl/hdrl-0.2.0

In this release we added two algorithms to derive a master flatfield
(see section~\ref{flat:main}) and one algorithm to compute the Strehl
ratio (see section~\ref{strehl:main})


\subsubsection{Release version 0.1.5}
The \HDRL release version 0.1.5 can be included from

http://svnhq2.hq.eso.org/p2/tags/Pipelines/common/hdrl/hdrl-0.1.5

The sigma clipping algorithm has been changed. It now uses a scaled
Median Absolute Deviation (MAD) to derive a robust RMS for the
clipping and not anymore the interquartile range (IQR). The MAD gives
better results for the case for low number statistics and a high
fraction of pixels affected by e.g. cosmic ray hits.  Furthermore, the
library integration in the pipeline slightly changed - see
section~\ref{library_integration} for details.

\subsubsection{Release version 0.1.0}
The \HDRL release version 0.1.0 can be included from

http://svnhq2.hq.eso.org/p2/tags/Pipelines/common/hdrl/hdrl-0.1.0

Various methods for bad pixel detection are added in this release (see
section~\ref{sec:bpm:main})

\subsubsection{Release version 0.0.3}
The \HDRL release version 0.0.3 can be included from

http://svnhq2.hq.eso.org/p2/tags/Pipelines/common/hdrl/hdrl-0.0.3

It is the first prototype release.

\subsection{Dependencies}
\label{library_dependencies}

Relationship with \CPL and other libraries:

The latest hdrl library depends on

\begin{itemize}
\item The Common Pipeline Library (CPL) version 7.0 or
  higher. \textbf{Please note that CPL must be compiled with wcs
    functionality available}
\item The GSL - GNU Scientific Library version 1.16 or higher.
\item The \href{https://github.com/liberfa/erfa}{ERFA} (Essential
  Routines for Fundamental Astronomy) library version 1.3 or
  higher. ERFA is a C library containing key algorithms for
  astronomy, and is based on the \href{http://www.iausofa.org}{SOFA
    library} published by the International Astronomical Union (IAU).
\item The \href{https://curl.se/libcurl/}{libcurl} (The multiprotocol
  file transfer library) library - any recent version.
\end{itemize}

%\subsection{Code Standards}
%
%\subsection{Templates}
%
%\subsection{Testing}
%
%\subsubsection{Unit tests}
%\subsubsection{Integration tests}
%\subsubsection{Automatic testing tools}

\endgroup
