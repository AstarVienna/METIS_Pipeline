
\subsection{Flat}
\label{flat:main}

Flat field frames give information on the response of the detector,
allowing to measure variations in efficiency at small
(pixel-to-pixel), intermediate (fringing) and large (the blaze
function) scale.  Whereas in the optical regime the flats are usually
twilight or dome flats, in the NIR regime the flats sometimes consists
of a pair of illuminated (denoted as ON-Flats) and not-illuminated
(OFF-Flats) frames.

In this section we present algorithms to combine single flatfields into
a master flatfield with increased SNR.


\subsubsection{High frequency master flatfield}
\label{flat:algorithms:hf}

This algorithm derives the high frequency part of the master flatfield
-- often also denoted as pixel-to-pixel variation.
\paragraph{Algorithm - short description}
\label{flat:algorithms:hf:short}

The algorithm first divides each input image by the smooth image
obtained with a median filter (the latter is controlled by the
parameters \verb,filter_size_x, and \verb,filter_size_y,). Concerning
the error propagation, the smoothed image is considered to be
noiseless, i.e. the relative error associated to the normalised images
is the same as the one of the input images. Then all residual images
are collapsed into a single master flatfield. The collapsing can be
done with all methods currently implemented in HDRL~(see
section~\ref{sec:imagelist:collapsing} for an overview).

To distinguish between illuminated and not illuminated regions/pixels
(i.e. orders in an echelle flat image) the user may provide an
optional static mask \verb,stat_mask, to the algorithm. In this case
the smoothing procedure is done twice, once for the illuminated region
and once for the blanked region. This ensures that the information of
one region does not influence the other regions during the smoothing
process.

\paragraph{Functions}
\label{flat:algorithms:hf:functions}

The high frequency master flatfield is computed using the following
function

\begin{lstlisting}

cpl_error_code hdrl_flat_compute(
        hdrl_imagelist          *  hdrl_data,
        const cpl_mask          *  stat_mask,
        const hdrl_parameter    *  collapse_params,
        hdrl_parameter          *  flat_params,
        hdrl_image              ** master,
        cpl_image               ** contrib_map)
\end{lstlisting}

The output \verb+master+ and \verb+contrib_map+ products are filled
with the resulting master flat and associated contribution map images.
The corresponding object data has to be deleted by the user when not
required any more.

\paragraph{Inputs}
\label{flat:algorithms:hf:inputs}

The list of flatfields that should be combined into a master flatfield
is passed to the function with the \\
\verb+hdrl_imagelist+
\verb+hdrl_data+ input function-parameter. If a static mask is needed
to distinguish between illuminated and dark regions, this can be
passed with the \verb+cpl_mask+ \verb+stat_mask+ input
function-parameter. The static mask is an optional input
function-parameter, i.e. one can pass NULL if no static mask is
needed.

Note that the function will overwrite the input imagelist in order to
conserve memory. Its contents after the call are undefined and it must
be deleted by the caller.

The \verb+collapse_params+ controls the collapsing algorithm and the
\verb+flat_params+ controls the smoothing algorithm.

The input \verb+flat_params+ parameter is created by the function
\begin{lstlisting}
hdrl_parameter * hdrl_flat_parameter_create(
        cpl_size                   filter_size_x,
        cpl_size                   filter_size_y,
        hdrl_flat_method           method)
\end{lstlisting}
In order to use the high frequency algorithm for master flatfield determination,
the \verb+method+ parameter must be set to \verb+HDRL_FLAT_FREQ_HIGH+. The
\verb+filter_size_x+ and \verb+filter_size_y+ parameters defining the smoothing
kernel must have an odd integer value greater then 0. If both values are
set to 1, no smoothing is done by the algorithm.


The \verb+collapse_param+ parameter input of the collapse function 
is a \verb+hdrl_parameter+. Its type defines the collapse method 
which is applied. Currently available are the following collapsing
parameter creation utilities:
\begin{itemize}\itemsep-1pt \parskip0pt \parsep0pt\small
\item \verb+hdrl_collapse_mean_parameter_create()+
\item \verb+hdrl_collapse_median_parameter_create()+
\item \verb+hdrl_collapse_weighted_mean_parameter_create()+
\item \verb+hdrl_collapse_sigclip_parameter_create(kappa_low, kappa_high, niter)+
\item \verb+hdrl_collapse_minmax_parameter_create(nlow, nhigh)+
\item \verb+hdrl_collapse_mode_parameter_create(histo_min, histo_max, bin_size,+\\\verb+mode_method, error_niter)+
\end{itemize}

Note that these parameters are dynamically allocated and must be
deleted when not needed.  For convenience HDRL provides preallocated
parameters for collapsing operations (e.g. \verb+HDRL_COLLAPSE_MEAN+)
which do not require any additional parameters. These can be used
anywhere a regular parameter can be used (deletion with
\verb+hdrl_parameter_delete+ has no effect on them). See
section~\ref{sec:imagelist:collapsing} for more details.

\paragraph{Outputs}
\label{flat:algorithms:hf:outputs}

The result is a \verb+hdrl_image+ stored in the function-parameter
\verb+master+ containing the master flatfield and its associated error
as well as an integer contribution map (\verb+contrib+) counting how
many values contributed to each pixel of the image.


\subsubsection{Low frequency master flatfield}
\label{flat:algorithms:lf}

This algorithm derives the low frequency part of the master flatfield
-- often also denoted as the shape of the flatfield.

\paragraph{Algorithm - short description}
\label{flat:algorithms:lf:short}

The algorithm multiplicatively normalizes the input images by the
median (considered to be noiseless) of the image to unity. An optional
static mask \verb,stat_mask, can be provided to the algorithm in order
to define the pixels that should be taken into account when computing
the normalisation factor. This allows the user to normalize the
flatfield e.g. only by the illuminated section. In the next step, all
normalized images are collapsed into a single master flatfield. The
collapsing can be done with all methods currently implemented in
hdrl~(see section~\ref{sec:imagelist:collapsing}). Finally, the master
flatfield is smoothed by a median filter controlled by
\verb,filter_size_x, and \verb,filter_size_y,. The associated error of
the final masterframe is the error derived via error propagation of
the previous steps, i.e. the smoothing itself is considered
noiseless. Please note that, if the smoothing kernel is set to unity,
i.e. \verb,filter_size_x = 1, and \verb,filter_size_y = 1, no final
smoothing will take place but the resulting masterframe is simply the
collapsed normalized flatfields.

\paragraph{Functions}
\label{flat:algorithms:lf:functions}

The called functions are the same as for the high frequency
masterflat (see section~\ref{flat:algorithms:hf:functions})

\paragraph{Inputs}
\label{flat:algorithms:lf:inputs}

The input parameters are the same as for the high frequency masterflat
(see section~\ref{flat:algorithms:hf:inputs}) with the exception that
in order to trigger the computation of the low frequency flatfield,
the flatfield \verb+method+ must be set to \verb+HDRL_FLAT_FREQ_LOW+.

\paragraph{Outputs}
\label{flat:algorithms:lf:outputs}

As for the high frequency masterflat algorithm, the result is a
\verb+hdrl_image+ stored in the function-parameter \verb+master+
containing the master flatfield and its associated error as well as an
integer contribution map (\verb+contrib+) counting how many values
contributed to each pixel of the image.


\subsubsection{Examples}
\begin{lstlisting}
hdrl_image* master;
cpl_image* contrib_map;

/* initialize the parameters of a sigma clipping combination */
hdrl_parameter * collapse_params;
collapse_params = hdrl_collapse_sigclip_parameter_create(3., 3., 5)

/* initialize the parameters of the flatfield */
hdrl_parameter * flat_params;
flat_params = hdrl_flat_parameter_create(7, 7, HDRL_FLAT_FREQ_LOW) ;

/* Do the actual flatfield computation */
hdrl_flat_compute(input_imagelist, stat_mask, collapse_params,
                                     flat_params, &master, &contrib_map);

/* cleanup the parameter*/
hdrl_parameter_delete(collapse_params);
hdrl_parameter_delete(flat_params);
\end{lstlisting}



\subsubsection{Master Flat Parameters in a Recipe}

In the context of a recipe, all input parameters needed by the master flat
computation may be provided as input parameters.

The creation of \verb+cpl_parameter+ objects for the collapse recipe interface is
facilitated by:
\begin{itemize}
\item \verb+hdrl_collapse_parameter_create_parlist+
\item \verb+hdrl_collapse_parameter_parse_parlist+
\end{itemize}
The former provides a \verb+cpl_parameterlist+ which can be appended
to the recipe parameter list, the latter parses the recipe parameter
list for the previously added parameters and creates a ready to use
collapse \verb+hdrl_parameter+.

The creation of \verb+cpl_parameter+ objects for the flat recipe interface is
facilitated by:
\begin{itemize}
\item \verb+hdrl_flat_parameter_create_parlist+
\item \verb+hdrl_flat_parameter_parse_parlist+
\end{itemize}
The former provides a \verb+cpl_parameterlist+ which can be appended
to the recipe parameter list, the latter parses the recipe parameter
list for the previously added parameters and creates a ready to use
flat \verb+hdrl_parameter+.


\begin{lstlisting}
/* prepare defaults for recipe */


/* Create collapse default parameters */

hdrl_parameter * sigclip_def = 
    hdrl_collapse_sigclip_parameter_create(3., 3., 5);
hdrl_parameter * minmax_def =
    hdrl_collapse_minmax_parameter_create(1., 1.);
hdrl_parameter * mode_def =
hdrl_collapse_mode_parameter_create(10., 1., 0., HDRL_MODE_MEDIAN, 0);

/* create collapse parameters with context RECIPE_NAME under the
collapse hierarchy */

cpl_parameterlist * pflatcollapse =hdrl_collapse_parameter_create_parlist(
RECIPE_NAME, "collapse", "MEDIAN", sigclip_def, minmax_def, mode_def) ;

hdrl_parameter_delete(sigclip_def); 
hdrl_parameter_delete(minmax_def);
hdrl_parameter_delete(mode_def);

/* add to recipe parameter list, change aliases, disable not required ones etc. */

for (cpl_parameter * p = cpl_parameterlist_get_first(pflatcollapse) ;
        p != NULL; p = cpl_parameterlist_get_next(pflatcollapse))
    cpl_parameterlist_append(self, cpl_parameter_duplicate(p));
cpl_parameterlist_delete(pflatcollapse);


/* Create flat default parameters */
hdrl_parameter * deflts = hdrl_flat_parameter_create(5, 5,
        HDRL_FLAT_FREQ_LOW);
/* create flat parameters with context RECIPE_NAME under the
   flat hierarchy */
cpl_parameterlist * flat_param = hdrl_flat_parameter_create_parlist(
            RECIPE_NAME, "flat", deflts) ;
/* add to recipe parameter list, change aliases,
   disable not required ones etc.
 */
hdrl_parameter_delete(deflts) ;
\end{lstlisting}

This will create a esorex man-page as follows:
{\footnotesize
\begin{verbatim}
esorex --man-page hdrldemo_flat_nir
...
--collapse.method             : Method used for collapsing the data. <MEAN |
                                WEIGHTED_MEAN | MEDIAN | SIGCLIP | MINMAX | 
                                MODE> [MEDIAN]
--collapse.sigclip.kappa-low  : Low kappa factor for kappa-sigma
                                clipping algorithm. [3.0]
--collapse.sigclip.kappa-high : High kappa factor for kappa-sigma
                                clipping algorithm. [3.0]
--collapse.sigclip.niter      : Maximum number of clipping iterations for
                                kappa-sigma clipping. [5]
--collapse.minmax.nlow        : Low number of pixels to reject for the minmax
                                clipping algorithm. [1.0]
--collapse.minmax.nhigh       : High number of pixels to reject for the
                                minmax clipping algorithm. [1.0]
--collapse.mode.histo-min     : Minimum pixel value to accept for mode
                                computation. [10.0]
--collapse.mode.histo-max     : Maximum pixel value to accept for mode
                                computation. [1.0]
--collapse.mode.bin-size      : Binsize of the histogram. [0.0]
--collapse.mode.method        : Mode method (algorithm) to use. <MEDIAN |
                                WEIGHTED | FIT> [MEDIAN]
--collapse.mode.error-niter   : Iterations to compute the mode
...

--flat.filter-size-x          : Smoothing filter size in x-direction. [5]
--flat.filter-size-y          : Smoothing filter size in y-direction. [5]
--flat.method                 : Method to use for the master flatfield calculation. 
                               <low | high> [low]
...
\end{verbatim}
}
