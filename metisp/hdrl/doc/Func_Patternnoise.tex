\subsection{Fixed Pattern Noise (fpn)}
\label{patternnoise:main}

This section describes how HDRL identifies regular
noise on detector data. A classical example is provided by the pick noise,
i.e. low-amplitude, quasi-periodical patterns super-imposed on the
normal read-noise. It is due to electronic interference and might show
up or disappear on short timescales (days or hours).

Those artifacts are visible only on the detector data but of course
exist also on other calibration data and on science data where they
may compromise the detector sensitivity.


\subsubsection{Algorithm - short description}

The algorithm implements three steps that can be described as follows:

\begin{enumerate}\itemsep-1pt \parskip0pt \parsep0pt\small
\item Derive the power spectrum of the image using the Fast Fourier
  Transform (FFT):
    \begin{eqnarray*}
      fft                 & = & FFT\_2D(\ img\ ) \\
      power\_spec         & = & abs(\ fft\ )^2
    \end{eqnarray*}
  \item Mask the peak of the power spectrum that corresponds to 
    the pixel-to-pixel variations. By default only one pixel
    placed in the corner on the bottom left is masked (the DC
    component), but more pixels can be masked providing an
    optional bad pixel mask, e.g. if there is a pixel-to-pixel 
    cross talk on longer spatial scale).
    \begin{eqnarray*}
      power\_spec\_filter & = & filter\_peak(\ power\_spec,\ 1,\ 1\ )
    \end{eqnarray*}
  \item Calculate the standard deviation $std$ and the standard
    deviation based on the Median Absolute Deviation (MAD) $std_{mad}$
    of the power spectrum by taking the masked regions into account:
    \begin{eqnarray*}
      std                  & = & stdev(\ power\_spec\_filter\ ) \\
      std_{mad}             & = & mad(\ power\_spec\_filter\ ) * 1.4826
    \end{eqnarray*}
\end{enumerate}

As mentioned above, when computing the standard deviation and the
MAD-based standard deviation on the power spectrum the algorithm
excludes the masked region. For this the user can provide an optional
mask \verb+mask_in+ or use the \verb+dc_mask_x+ and \verb+dc_mask_y+
function parameter to create one on the fly. The mask created on the
fly will start at pixel (1,1) and extend in both direction up to
(\verb+dc_mask_x+, \verb+dc_mask_y+)\footnote{Please note that in the
  current implementation the power spectrum contains the DC component
  (the DC term is the 0 Hz term and is equivalent to the average of
  all the samples in the window) in pixel (1,1).}. Moreover, the mask
created on the fly by setting \verb+dc_mask_x+ and \verb+dc_mask_y+
and the optional mask (\verb+mask_in+) are combined and are both taken
into account when calculating the statistics (\verb+std+ and
\verb+std_mad+). Furthermore, the final (combined) mask is attached to
the returned \verb+power_spectrum+ image as normal cpl\_mask and can
be retrieved with the CPL function \verb+cpl_image_get_bpm(power_spectrum)+

\newpage

\subsubsection{Function}

The power spectrum and it's statistics are computed by the following
function:
\begin{lstlisting}
cpl_error_code hdrl_fpn_compute(
        cpl_image       *  img_in,
        const cpl_mask  *  mask_in,
        const cpl_size     dc_mask_x,
        const cpl_size     dc_mask_y,
        cpl_image       ** power_spectrum,
        double          *  std,
        double          *  std_mad)
\end{lstlisting}

The \verb+power_spectrum+ pointer is filled by the function with an
allocated image which must be deleted by the user when not required
anymore.

%
%Possible #_cpl_error_code_ set in this function:
%- CPL_ERROR_NULL_INPUT          If img_in is NULL
%- CPL_ERROR_ILLEGAL_INPUT       If dc_mask_x < 1 or dc_mask_y < 1
%- CPL_ERROR_ILLEGAL_INPUT       If the power_spectrum is NOT NULL
%- CPL_ERROR_ILLEGAL_INPUT       If img_in contains bad pixels
%- CPL_ERROR_INCOMPATIBLE_INPUT  If mask NOT NULL and size(mask) != size(img_in)


\subsubsection{Inputs}
\label{sec:algorithms:fpn:inputs}

The following function parameters have to be passed to the
hdrl\_fpn\_compute function:
\begin{itemize}\itemsep-1pt \parskip0pt \parsep0pt\small
\item \verb+img_in+: the input image. Please note that bad pixels are
  not allowed
\item \verb+mask_in+ (or \verb+NULL+): an optional mask which is used
  when deriving the standard deviations on the power spectrum
\item \verb+dc_mask_x+: x-pixel window (>= 1) to discard DC component
  starting form pixel (1, 1)
\item \verb+dc_mask_y+: y-pixel window (>= 1) to discard DC component
  starting from pixel (1, 1)
\end{itemize}

\subsubsection{Outputs}
\label{sec:algorithms:fpn:outputs}

The result is the computed power spectrum (\verb+power_spectrum+),
the standard deviation (\verb+std+) of the power spectrum, the MAD
based standard deviation of the power spectrum (\verb+std_mad+) and
the final mask used to compute \verb+std+ and \verb+std_mad+ which can
be retrieved with \verb+cpl_image_get_bpm(power_spectrum)+

Please note that the \verb+power_spectrum+ pointer is filled by the
function with an allocated image which must be deleted by the user
when not required anymore.

\subsubsection{Example}
\label{sec:algorithms:fpn:example}
This section briefly describes the call to the function. For a more
detailed implementation please see also the \verb+hdrldemo_fpn+
example recipe.

{\footnotesize
\begin{lstlisting}
/* Load the image and the (optional) mask */
cpl_mask  *mask_in = cpl_mask_load(...);
cpl_image *img_in  = cpl_image_load(...);


/* Compute the power spectrum and the statistics */
cpl_image *power_spectrum  = NULL;
double    std       = -1.;
double    std_mad   = -1.;

hdrl_fpn_compute(img_in, mask_in, dc_mask_x, dc_mask_y,
                 &power_spectrum, &std, &std_mad);

/* save the power spectrum and the used mask */
...

/* Cleanup the memory  */
cpl_image_delete(power_spectrum);
cpl_image_delete(img_in);
cpl_mask_delete(mask_in);
\end{lstlisting}
}
