\subsection{Object catalogue generation}
\label{catalogue:main}

In order to provide astrometric and photometric calibration
information, the \HDRL implements a functionality to generate a
catalogue of detected objects (i.e. stars, galaxies).

Objects are detected and parametrised using the processed input image
(\verb+img+) and confidence map (\verb+cnf+). A confidence map is
similar to a weight map. It gives an indication of the relative
quantum efficiency of each pixel, rather than an indication of the
Poisson uncertainty on the pixel flux and should be normalised to a
mean of 100. ``Bad'' pixels (either dead or hot) are given a confidence
value of zero. If stacked images are used, then the confidence map of
each single image should be propagated in order to get a confidence
map of the mosaic image. This confidence map should then be
re-normalized to a mean of 100 and passed to the routine as confidence
map.

A high-level summary of the implemented data reduction sequence is:
\begin{itemize}
\item estimate the local sky background over the image and track any
    variations at adequate resolution to eventually remove them,
\item detect objects/blends of objects and keep a list of pixels
    belonging to each blend for further analysis (see \cite{Irwin85}
    for details)
  \item parametrise the detected objects, i.e. perform astrometry,
    photometry and a shape analysis.
\end{itemize}

\subsubsection{Algorithm - short description}


If requested, (\verb+bkg_estimate+) the possibly-varying sky
background is estimated and removed automatically, prior to object
detection, using a combination of robust iteratively-clipped
estimators.

Any variation in sky level over the frame is dealt with by forming a
coarsely sampled background map grid (specified by
\verb+bkg_mesh_size+).  Within each background grid pixel, an
iteratively k-sigma clipped median value of `sky' is computed based on
the flux values within the grid pixel zone. A robust estimate of sigma
is computed by using the Median of the Absolute Deviation (MAD) from
the median. This will then be further processed to form the frame
background image.

After removing the (possibly) varying background component, a similar
robust estimate of the average sky level and sky noise per pixel is
made. This allows to robustly obtain the detection threshold for
object analysis.

Individual objects are detected using a ``standard match'' filter
approach. Since the only sources difficult to locate are those
marginally above the sky noise, to assume (after factoring in the
confidence map information) the noise as uniform, is a good
approximation. The majority of objects detected in this way 
will have a shape dominated by the point spread function (PSF), 
which thereby defines the filter to use (controlled by 
\verb+bkg_smooth_fwhm+.)

To be detected an object should consist of at least
\verb+obj_min_pixels+ contiguous pixels above a threshold controlled
by \verb+obj_threshold+ $\times$ sky-rms. Moreover, one can also
activate a deblending algorithm (\verb+obj_deblending+) in order to
disentangle overlapping objects.

Pixels are weighted in the catalogue generation according to their
value in the confidence map.  Hence if a pixel is marked as bad, then
it is not included in the flux computation over the aperture.  
The number of bad pixels
with an aperture is reported in the 'Error\_bit\_flag' column of the
output table.  The presence of bad pixels will also be reflected in
the average confidence for the aperture (column 'Av\_conf').

Please note that one also has to specify the effective gain
(\verb+det_eff_gain+) and the (\verb+det_saturation+) limit to the
routine. The gain is mainly used for the error estimation of the
various object measurements whereas the saturation limit is used to
mark and exclude saturated pixels.

%
%
%
%
%    \verb+obj_min_pixels+
%    \verb+obj_threshold+
%    \verb+obj_deblending+
%    \verb+obj_core_radius
%
%    \verb+bkg_estimate+
%    \verb+bkg_mesh_size+
%    \verb+bkg_smooth_fwhm+
%
%    \verb+det_eff_gain+
%    \verb+det_saturation+
%
%    resulttype+
%

A detailed description of the algorithm and the output table can be
found in appendix~\ref{chap:algorithms:catalogue}

 
\subsubsection{Functions}
\label{catalogue:functions}

The object catalogue, the background and the segmentation map are
computed by the following function call:

\begin{lstlisting}
hdrl_catalogue_result * hdrl_catalogue_compute(
      const cpl_image * image, 
      const cpl_image * confidence_map,
        const cpl_wcs * wcs, 
       hdrl_parameter * param)
\end{lstlisting}
             
where \verb+image+ is the input image, \verb+confidence_map+ is the
(optional) confidence map, \verb+wcs+ contains the (optional) world
coordinate system and \verb+params+ is a \verb+hdrl_parameter+
structure specifying the parameters controlling the catalogue
generation.
           
\subsubsection{Inputs}
\label{catalogue:inputs}

The input parameter to be passed to the function are created by the
following function call:

\begin{lstlisting}
hdrl_parameter * param = hdrl_catalogue_parameter_create(
      int                    obj_min_pixels,
      double                 obj_threshold, 
      cpl_boolean            obj_deblending,
      double                 obj_core_radius,
      cpl_boolean            bkg_estimate, 
      int                    bkg_mesh_size,
      double                 bkg_smooth_fwhm, 
      double                 det_eff_gain,
      double                 det_saturation,
      hdrl_catalogue_options resulttype);
\end{lstlisting}


where
  \verb+obj_min_pixels+ is the minimum pixel area for each detected object,
  \verb+obj_threshold+  is the detection threshold in sigma above sky,
  \verb+obj_deblending+ is a parameter to switch on/off deblending,\\
  \verb+obj_core_radius+ is the value of the core radius in pixels,
  \verb+bkg_estimate+  is the parameter to switch on/off the
  background estimation from the input image,
  \verb+bkg_mesh_size+ is the background smoothing box size,
  \verb+bkg_smooth_gauss_fwhm+ is the FWHM of the Gaussian kernel used
  in convolution for object detection,
  \verb+det_eff_gain+ is the detector gain value to convert ADU counts to electron counts,
  \verb+det_saturation+  is the detector saturation value, and
  \verb+resulttype+ is the \verb+hdrl_catalogue_options+ parameter which
controls the different result types. The latter is a bit-map (the
values can be combined with bitwise or) defining the computed and
returned results:

\begin{itemize}
\item \verb+HDRL_CATALOGUE_CAT_COMPLETE+: The full catalogue (see
  appendix XXX)
\item \verb+HDRL_CATALOGUE_BKG+: The background map
\item \verb+HDRL_CATALOGUE_SEGMAP+: The segmentation map
  \footnote{Contains the pixels attributed to each object, i.e. all
    pixels belonging to object XXX in the catalogue have a value of
    XXX in the segmentation map}
\item \verb+HDRL_CATALOGUE_ALL+:\\  \verb+HDRL_CATALOGUE_CAT_COMPLETE+ |
  \verb+HDRL_CATALOGUE_BKG+ | \verb+HDRL_CATALOGUE_SEGMAP+  
\end{itemize}

If a certain result type is not requested by setting the appropriate
bit-pattern, the function returns a pointer to NULL.

\subsubsection{Outputs}

The function creates an output \verb+hdrl_catalogue_result+ structure
containing the following quantities (or NULL, depending on the
hdrl\_catalogue\_options \verb+resulttype+):

\begin{lstlisting}
typedef struct {
    /* object catalogue */
    cpl_table * catalogue;
    /* segmentation map */
    cpl_image * segmentation_map;
    /* background map */
    cpl_image * background;
    /* Additional information */
    cpl_propertylist * qclist;
} hdrl_catalogue_result;
\end{lstlisting}

The \verb+hdrl_catalogue_result+ object has to be deleted by the user
when not required any more by calling the function
\verb+hdrl_catalogue_result_delete()+.

Currently the following additional informations are returned in the
cpl\_propertylist \verb+qclist+:
{\footnotesize
\begin{verbatim}
APCOR1   / Stellar aperture correction - 1/2x core flux
APCOR2   / Stellar aperture correction - core/sqrt(2) flux
APCOR3   / Stellar aperture correction - 1x core flux
APCOR4   / Stellar aperture correction - sqrt(2)x core flu
APCOR5   / Stellar aperture correction - 2x core flux
APCOR6   / Stellar aperture correction - 2*sqrt(2)x core f
APCOR7   / Stellar aperture correction - 4x core flux
APCORPK  / Stellar aperture correction - peak height
\end{verbatim}
}
and \verb+SYMBOL1+ to \verb+SYMBOL9+. 

If the symbol keywords are present in the header of the catalogue
fitsfile they are used by e.g. skycat or gaia to properly display the
objects in the image.

The APCOR keywords can be used to convert (for stellar objects) fixed
aperture fluxes to total magnitudes, as for this an aperture
correction must be applied. This correction is calculated from the
curve-of-growth and accounts for missing flux in a particular
aperture. For example, the magnitude (m) of an object using the flux
in the third bin ($i = 3$) may be calculated as

\begin{equation}
m = \tt{MAGZPT} - 2.5log_{10}(\tt{APER\_FLUX\_3/EXPTIME}) - \tt{APCOR3}
\end{equation}

With {\tt APER\_FLUX\_3} being a column in the catalogue table, {\tt
  MAGZPT} the zeropoint and EXPTIME the exposure time \footnote{For more information, refer to \url{http://casu.ast.cam.ac.uk/surveys-projects/vst/technical/catalogue-generation}}.




\subsubsection{Examples}

Detect objects on an image (img) with a given confidence map (cnf) and
wcs informations (wcs):


\begin{lstlisting}
  cpl_image * img;
  cpl_image * cnf;
  cpl_wcs * wcs;

  /* initialize the catalogue parameters */

  hdrl_parameter * p =  
       hdrl_catalogue_parameter_create(5,        /* obj_min_pixel */      
                                       2.5,      /* obj_threshold */     
                                       CPL_TRUE, /* obj_deblending */    
                                       3.0,      /* obj_core_radius */   
                                       CPL_TRUE, /* bkg_estimate */      
                                       64,       /* bkg_mesh_size */     
                                       2.0,      /* bkg_smooth_fwhm */   
                                       2.8,      /* det_eff_gain */      
                                       65200,    /* det_saturation */    
                                       HDRL_CATALOGUE_ALL);

  /* catalogue COMPUTATION */

  hdrl_catalogue_result * res = hdrl_catalogue_compute(img, cnf, wcs, p);

  /* save the catalogue, the background and the segmentation map */

  /* delete the result structure */
  hdrl_catalogue_result_delete(res);
\end{lstlisting}

\subsubsection{Catalogue Parameters in a Recipe}

In the context of a recipe, all input parameters needed by the
catalogue computation may be provided as input parameters.

The creation of \verb+cpl_parameter+ objects for the catalogue recipe interface is
facilitated by:
\begin{itemize}
\item \verb+hdrl_catalogue_parameter_create_parlist+
\item \verb+hdrl_catalogue_parameter_parse_parlist+
\end{itemize}
The former provides a \verb+cpl_parameterlist+ which can be appended
to the recipe parameter list, the latter parses the recipe parameter
list for the previously added parameters and creates a ready to use
catalogue \verb+hdrl_parameter+.

\begin{lstlisting}
 /* Create catalogue default parameters */

  hdrl_parameter * c_def =
      hdrl_catalogue_parameter_create(4, 2.5, CPL_TRUE, 5.0, CPL_TRUE, 64,
                                      2., 3.0, HDRL_SATURATION_INIT, 
                                      HDRL_CATALOGUE_ALL);
  cpl_parameterlist * s_param =
      hdrl_catalogue_parameter_create_parlist(RECIPE_NAME, "", c_def);
  hdrl_parameter_delete(c_def) ;

  for (cpl_parameter * p = cpl_parameterlist_get_first(s_param);
          p != NULL; p = cpl_parameterlist_get_next(s_param))
      cpl_parameterlist_append(self, cpl_parameter_duplicate(p));
  cpl_parameterlist_delete(s_param);
\end{lstlisting}

This will create a esorex man-page as follows:
{\footnotesize
\begin{verbatim}
esorex --man-page hdrldemo_catalogue
...
  --obj.min-pixels         : Minimum pixel area for each detected object. [4]
  --obj.threshold          : Detection threshold in sigma above sky. [2.5]
  --obj.deblending         : Use deblending?. [TRUE]
  --obj.core-radius        : Value of Rcore in pixels. [5.0]
  --bkg.estimate           : Estimate background from input, if false it is assumed
                             into is already background corrected with median 0.
                             [TRUE]
  --bkg.mesh-size          : Background smoothing box size. [64]
  --bkg.smooth-gauss-fwhm  : The FWHM of the Gaussian kernel used in convolution
                             for object detection. [2.0]
  --det.effective-gain     : Detector gain value to rescale convert intensity to
                             electrons. [3.0]
  --det.saturation         : Detector saturation value. [inf.0]
...
\end{verbatim}
}

If parlist is the recipe parameters list, we also offer a function to 
generate the Catalogue parameters directly from the recipe parameters:

\begin{lstlisting}
/* Parse the Catalogue Parameters */

hdrl_parameter * p = 
   hdrl_catalogue_parameter_parse_parlist(parlist, RECIPE_NAME);
 \end{lstlisting}
