\subsection{Image Lists}
Similar to CPL that provides \verb+cpl_imagelist+ to store collections
of equally dimensioned \verb+cpl_image+ objects, HDRL provides an
equivalent \verb+hdrl_imagelist+.  The API is again similar but, like
with images, the simple arithmetic operations and collapse operations
propagate errors according to the linear error propagation theory.  For
details please refer to the reference.

\subsubsection{Views}
In addition to handling simple error propagation automatically
\verb+hdrl_imagelist+ allows creating multiple views of the list which point to
the same data buffers.
These views can be used in any place regular \verb+hdrl_imagelist+ are used and
they are also deleted with \verb+hdrl_imagelist_delete+ (in contrast to
\verb+cpl_image+ views which must be deleted with \verb+cpl_image_unwrap+).
Also, any \verb+hdrl_image+ extracted from a view can be deleted using \verb+hdrl_image_delete+.   

Views allow applying operations on subsets of the imagelists without creating
copies.
This is especially useful when dealing with large swap or disk backed files.

Due to technical limitations in CPL only currently views of full rows of images
are available.

\paragraph{Example}
\begin{lstlisting}
void process(hdrl_imagelist * list)
{
    hdrl_imagelist_add_scalar(list, 5. , 1.);
    hdrl_imagelist_mul_scalar(list, 2., 0.);
}

void process_big_data(hdrl_imagelist * large_list)
{
    size_t ny = 100;
    /* process the large list in chunks of several rows,
       this can be very beneficial if the images contained in the
       imagelist are allocated from a disk backed memory map as it
       performs more operations per load */
    for (size_t i = 0; i < ny; i += 10) {
        /* create a view of the large imagelist only containing 10 rows */
        hdrl_imagelist * view;
        view = hdrl_imagelist_row_view(large_list, i, i + 10);
        process(view);
        hdrl_imagelist_delete(view);
    }
    /* now the full input imagelist has been processed */
}
\end{lstlisting}

If the user has constraints on the amount of RAM that can be
allocated, and need to chunk an imagelist, HDRL
provides an iterator interface to row views of \verb+hdrl_imagelist+.

\begin{lstlisting}
/* create an iterator providing views containing blocksize rows
   the iterator owns the view and will take care of deallocating
   it on each iteration */
hdrl_iter * it = hdrl_imagelist_get_iter_row_slices(himlist, blocksize,
                                                    HDRL_ITER_OWN_OUTPUT);

/* hdrl_iter_next returns a view on the data or NULL when its done */
for (hdrl_imagelist * v = hdrl_iter_next(it);
     v != NULL;
     v = hdrl_iter_next(it)) {
    process(v);
}
hdrl_iter_delete(it);
\end{lstlisting}

\subsubsection{Collapse Operations}
\label{sec:imagelist:collapsing}

Collapse operations like sum, mean or standard deviations can be
performed with \verb+hdrl_imagelist+. For operations were the error
propagation formula is well defined,
the errors will be accounted for to determine the result.  In most cases a
collapse operation only has one or two outputs, the result image and a
integer contribution map counting how many values contributed to each
pixel of the image. These collapse operations are called via
\verb+hdrl_imagelist_collapse+:
\begin{lstlisting}
cpl_error_code hdrl_imagelist_collapse(
        const hdrl_imagelist    *   himlist,
        const hdrl_parameter    *   param,
        hdrl_image              **  out,
        cpl_image               **  contrib)
\end{lstlisting}
The \verb+out+ and \verb+contrib+ pointers are filled with allocated result
images which must be deleted by the user when not required anymore.

The parameter \verb+param+ is a \verb+hdrl_parameter+ who's type
defines the exact collapse method which is applied.  Currently
available are:
\begin{itemize}
\item \verb+hdrl_collapse_mean_parameter_create()+
\item \verb+hdrl_collapse_median_parameter_create()+
\item \verb+hdrl_collapse_weighted_mean_parameter_create()+
\item \verb+hdrl_collapse_sigclip_parameter_create(kappalow, kappahigh, niter)+
\item \verb+hdrl_collapse_minmax_parameter_create(nlow, nhigh)+
\item \verb+hdrl_collapse_mode_parameter_create(histo_min, histo_max, bin_size,+\\
\verb+mode_method, error_niter)+
\end{itemize}

Note that these parameters are dynamically allocated and must be deleted again.
For convenience HDRL provides preallocated parameters for collapses which do
not require any additional parameter. These can be used anywhere a regular
parameter can be used and deletion with \verb+hdrl_parameter_delete+ has no
effect on them:

\begin{itemize}
\item \verb+HDRL_COLLAPSE_MEAN+
\item \verb+HDRL_COLLAPSE_MEDIAN+
\item \verb+HDRL_COLLAPSE_WEIGHTED_MEAN+
\end{itemize}


The sigma clipping and minmax collapse method can additionally return
the low and high rejection thresholds used to calculate the
mean. These results can only be obtained with
\verb+hdrl_imagelist_collapse_sigclip+ and
\verb+hdrl_imagelist_collapse_minmax+ which takes two additional
output pointer arguments which will be filled with the allocated
results.

\begin{lstlisting}
cpl_error_code hdrl_imagelist_collapse_sigclip(
        const hdrl_imagelist    *   himlist,
        double                      kappa_low,
        double                      kappa_high,
        int                         niter,
        hdrl_image              **  out,
        cpl_image               **  contrib,
        cpl_image               **  reject_low,
        cpl_image               **  reject_high);
\end{lstlisting}

\begin{lstlisting}
cpl_error_code hdrl_imagelist_collapse_minmax(
        const hdrl_imagelist    *   himlist,
        double                      nlow,
        double                      nhigh,
        hdrl_image              **  out,
        cpl_image               **  contrib,
        cpl_image               **  reject_low,
        cpl_image               **  reject_high);
\end{lstlisting}

\begin{lstlisting}
 cpl_error_code hdrl_imagelist_collapse_mode(
        const hdrl_imagelist    *   himlist,
        double                      histo_min,
        double                      histo_max,
        double                      bin_size,
        hdrl_mode_type              mode_method,
        cpl_size                    error_niter,
        hdrl_image              **  out,
        cpl_image               **  contrib) ;
\end{lstlisting}                  

Please note that all methods but the mode are doing \textbf{error
  propagation}. The mode method is special in this case as it
\textbf{calculates the error from the data}. The calculation depends
on the parameter \verb+error_niter+. If the parameter is set to 0, it
is doing an analytically error estimation. If the parameter is larger
than 0, the error is calculated by a bootstrap Montecarlo simulation
from the input data with the value of the parameter specifying the
number of simulations. In this case the input data are perturbed with
the bootstrap technique and the mode is calculated \verb+error_niter+
times. From this modes the standard deviation is calculated and
returned as error.

