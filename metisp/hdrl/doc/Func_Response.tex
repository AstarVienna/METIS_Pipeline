\subsection{Response}
\label{response:main}

This section deals with the HDRL module estimating the response as a 1D function of wavelength.

\subsubsection{Algorithm - short description}
The algorithm is divided in two parts: \textit{Telluric correction} and \textit{Response calculation}. In the provided implementation the \textit{Telluric correction} is optional and can be disabled by the user.

\paragraph{Telluric Correction}

The Telluric Correction functionality tries to find the best telluric model among the ones provided and uses it to correct the observed spectrum.
The steps of the algorithm are as follows and they are repeated for each telluric model. The steps are grouped in two: \textit{Spectrum correction} and \textit{Quality evaluation}.

\textbf{Spectrum correction: }aims to correct the observed spectrum with the provided telluric model.
\begin{enumerate}
\item Convert wavelength scale (in nm) of the telluric model and the
  observation to natural logarithm if the ration $\lambda$/FWHM is
  constant. A strong prerequisite is that the sampling step of the
  logarithmic wavelength scale is very fine for both spectra to
  provide proper sampling. These resampled spectra are used only for
  the cross correlation described below;
\item Extract the range of wavelengths needed to calculate the cross correlation (water vapour feature of medium strength);
\item Cross correlate the telluric model with the observation spectrum
  over the extracted range. The cross correlation is needed to
  determine the shift and difference in resolution between the
  telluric model spectrum and the observation;
\item Fit a Gaussian around the cross correlation peak: finding the shift between the two spectra in sub-pixel precision and the FWHM;
\item Repeat the previous two steps refining the cross-correlation interval around the identified peak position;
\item Shift the telluric model by the shift obtained by the previous step;
\item Create a Gaussian kernel of size equal to the measured FWHM;
\item Convolve the shifted telluric model with the Gaussian kernel
  (this assumes that the resolution of the telluric model is much
  higher than that of the observation);
\item Convert the shifted and convolved telluric model to linear wavelength space (if needed). Resample it using integration to the wavelengths of the observed spectrum;
\item Divide the observed spectrum by the convolved and shifted telluric model: this is the \textit{corrected observed spectrum};
\end{enumerate}

\textbf{Quality evaluation: } aims to evaluate the quality of the \textit{corrected observed spectrum} calculated in the previous step.
\begin{enumerate}
\item Using the predefined continuum points, fit a spline to the \textit{corrected observed spectrum} without smoothing;
\item Divide the \textit{corrected observed spectrum} by the continuum fit;
\item Calculate the mean $m_r$ of this ratio, skipping high absorption
  areas as well as regions without telluric absorption. 
\end{enumerate}

The quantity $q = |m_r - 1|$ is used as quality indicator for the fit of the telluric model. The model having the lowest $q$ is deemed the best and its corresponding \textit{corrected observed spectrum} is used in the \textit{Response calculation} step. 
If telluric correction is disabled by the user, the \textit{Response calculation} step uses the unmodified observed spectrum.

\paragraph{Response Calculation}

\textbf{Velocity compensation: } this step can be disabled by the user and it is divided in two steps.
\begin{enumerate}
\item Determine the radial velocity of the observed (possibly telluric- corrected) spectrum by fitting a Gaussian to a known stellar line;
\item Apply the radial velocity to the stellar model spectrum as $\lambda = \lambda_0(1+\frac{RV}{c})$.
\end{enumerate}


The remaining steps to complete the response calculation are:
\begin{enumerate}
\item Calculate the raw response as described in equation \ref{eq:resp};
\item Define fit points at which the median of the flux within a given window is determined, avoiding region of very {high telluric absorption} as well as the cores of the stellar lines;
\item Fit a spline through the points determined in the previous step.
\end{enumerate}

The response is calculated according to 

\begin{equation}
R(\lambda)=  \frac{I_{std-ref}^{(r)}(\lambda) \cdot 10^{[ 0.4(A_p- A_m)E_x^{(r)}(\lambda)] } \cdot G \cdot T_{ex}}{I_{std}(\lambda)}
\label{eq:resp}
\end{equation}
where:

\begin{itemize}

\item $I_{std}(\lambda)$ is the observed 1D spectrum [ADU] as function of wavelength [nm]. If telluric correction is enabled, 

\item  $E_x^{(r)}(\lambda)$ is the resampled atmospheric extinction [mag/airmass] as function of wavelength [nm]. The corresponding, user-provided spectrum is denoted as $E_x(\lambda)$;

\item $A_m$ is the airmass of the observed standard star spectrum;

\item $A_p$ is a parameter to indicate if the response is computed at $A_m=0$ or at a given value (usually the one at which the reference standard star spectrum may be tabulated);

\item $G$ is the gain  of the detector in [ADU/e];

\item $I_{std-ref}^{(r)}(\lambda)$ resampled reference standard star spectrum. The wavelength is expressed in [nm]. The corresponding, user-provided spectrum is denoted as $I_{std-ref}(\lambda)$; 
\end{itemize} 

$E_x(\lambda)$  and $I_{std-ref}(\lambda)$ are resampled by the provided routine to match the wavelengths $I_{std}(\lambda)$ is defined on. If they already match the resampling is skipped. $R(\lambda)$ is defined for $\lambda$ only if $\lambda$ is contained in the interval $E_x(\lambda)$  and $I_{std-ref}(\lambda)$ are defined on. 

\subsubsection{Functions}
The response is computed by the following function call:

\begin{lstlisting}
hdrl_response_result *
hdrl_response_compute(
        const hdrl_spectrum1D * obs_s,
        const hdrl_spectrum1D * ref_s,
        const hdrl_spectrum1D * E_x,
        const hdrl_parameter * telluric_par,
        const hdrl_parameter * velocity_par,
        const hdrl_parameter * calc_par,
        const hdrl_parameter * fit_par)
\end{lstlisting}

\subsubsection{Inputs}
\label{sec:algorithms:response:inputs}
The input parameters to be passed to \verb+hdrl_response_compute+ are created executing the following function calls:

\begin{lstlisting}

hdrl_parameter * telluric_par = 
hdrl_response_telluric_evaluation_parameter_create(telluric_models,
        x_corr_w_step, xcorr_half_win, xcorr_normalize, 
        xcorr_is_shift_in_log, quality_areas, fit_areas, 
        lmin, lmax);
\end{lstlisting}


\verb+telluric_par+ contains the parameters needed to perform the telluric correction, the parameters can be divided in parameters needed to perform che cross-correlation, and hence determine the shift, and parameters needed to calculate the quality of the fit.



\begin{lstlisting}
hdrl_parameter * velocity_par = 
hdrl_spectrum1D_shift_fit_parameter_create(velocity_wguess,
    		velocity_range_wmin, velocity_range_wmax, 
    		velocity_fit_wmin, velocity_fit_wmax, 
    		velocity_fit_half_win);
\end{lstlisting}


\verb+velocity_par+ contains the parameters needed to perform the velocity correction.



\begin{lstlisting}
hdrl_parameter * calc_par = 
hdrl_response_parameter_create(Ap, Am, G, Tex);
\end{lstlisting}


\verb+calc_par+ contains the parameters needed to calculate the result of \ref{eq:resp}.



\begin{lstlisting}
hdrl_parameter * fit_par =
    hdrl_response_fit_parameter_create(11, fit_points, 
	fit_wrange, high_abs_regions);
\end{lstlisting}


\verb+fit_par+ contains the parameters needed to execute the final fit, after \ref{eq:resp} is performed.




\subsubsection{Outputs}
The function returns a struct of type \verb+hdrl_response_result+ containing the response as an \verb+hdrl_spectrum1D+ and other parameters that can be used for QC purposes. For a complete description of such parameters, please, refer to the documentation of \verb+hdrl_response_result+.

\subsubsection{Examples}
The following listing shows how to calculate the efficiency using the user-provided spectra and parameters.
The user has already provided the following spectra:
\begin{itemize}
\item \verb+I_std+: observed spectrum, wavelength in [nm];
\item \verb+I_std_ref+: model of the spectrum of the observed star, wavelength in [nm]. If defined on different wavelengths than \verb+I_std+ the function will resample it using \textit{Akima} interpolation;
\item \verb+E_x+: atmospheric extinction wavelength in [nm]. If defined on different wavelengths than \verb+I_std+ the function will resample it using \textit{Akima} interpolation \footnote{See: \url{https://www.gnu.org/software/gsl/manual/html_node/Interpolation-Types.html}}.
\end{itemize}
\begin{lstlisting}

hdrl_parameter * telluric_par = 
hdrl_response_telluric_evaluation_parameter_create(telluric_models,
        x_corr_w_step, xcorr_half_win, xcorr_normalize, 
        xcorr_is_shift_in_log, quality_areas, fit_areas, 
        lmin, lmax);

hdrl_parameter * velocity_par = 
hdrl_spectrum1D_shift_fit_parameter_create(velocity_wguess,
    		velocity_range_wmin, velocity_range_wmax, 
    		velocity_fit_wmin, velocity_fit_wmax, 
    		velocity_fit_half_win);
    		
hdrl_parameter * calc_par = 
hdrl_response_parameter_create(Ap, Am, G, Tex);

hdrl_parameter * fit_par =
    hdrl_response_fit_parameter_create(11, fit_points, 
	fit_wrange, high_abs_regions);

hdrl_response_result *
response_res = hdrl_response_compute(
            		I_std, 
            		I_std_ref,
            		E_x,
            		telluric_par,
            		velocity_par,
            		calc_par ,
            		fit_par);

hdrl_parameter_delete(telluric_par);
hdrl_parameter_delete(velocity_par);
hdrl_parameter_delete(calc_par);
hdrl_parameter_delete(fit_par);

/*... use the content of response_res  */

hdrl_response_result_delete(response_res);

\end{lstlisting}

\subsubsection{Notes}
\begin{itemize}
\item The fluxes of the observed and model spectra must be expressed in the same units. Usually the model is given in  erg/s/cm2/Angstrom. Therefore the user should correct the observed spectrum accordingly, by, e.g. a factor equal to the sampling bin size expressed in Angstrom unit;
\item The response should be determined in optimal (photo-metric) atmospheric conditions.
Furthermore, in case of different modes, the results of one mode cannot be compared with the results of the other. For example, if the observation facility has support for a Laser Guide Star (LGS) observation can be obtained either using LGS or using a Natural Guide Star (NGS). Results obtained with NGS cannot be compared with the ones obtained with LGS;
\item Observations should be obtained in the same optical setting mode.
\end{itemize}
