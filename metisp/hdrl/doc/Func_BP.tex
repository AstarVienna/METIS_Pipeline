\subsection{Bad-pixel detection}
\label{sec:bpm:main}

\subsubsection{Bad-pixel detection on a single image}
\label{sec:bpm_2d}
The recipe derives bad pixels on individual images.

\paragraph{Algorithm - short description}


The algorithm first smoothes the image by applying the methods
described below. Then it subtracts the smoothed image and derives bad
pixels by thresholding the residual image, i.e. all pixels exceeding
the threshold are considered as bad. To compute the upper and lower
thresholds, it measures a robust rms (a properly scaled Median Absolute
Deviation), which is then scaled by the parameters \verb,kappa_low, and
\verb,kappa_high,. Furthermore, the algorithm is applied iteratively
for a number of times defined by the parameter \verb,maxiter,. During each iteration the newly found
bad pixels are ignored. Please note, that the thresholding values are
applied as median (residual-image) $\pm$ thresholds. This makes the
algorithm more robust in the case that the methods listed below are
not able to completely remove the background level, e.g due to an
exceeding number of bad pixels in the first iteration.
 
Two methods are currently available to derive a smoothed version of
the image:
\begin{itemize}
\item Applying a filter like e.g. a median filter to the image. The
  filtering can be done by all modes currently supported by cpl and is
  controlled by the filter-type \verb,filter,, the border-type
  \verb,border,, and by the kernel size in x and y, i.e.
  \verb,smooth_x,, and \verb,smooth_y,.
\item Fitting a Legendre polynomial to the image of order
  \verb,order_x, in x and \verb,order_y, in y direction.  This method
  allows you to define \verb,steps_x, $\times$ \verb,steps_y, sampling
  points (the latter are computed as the median within a box of
  \verb,filter_size_x, and \verb,filter_size_y,) where the polynomial
  is fitted. This substantially decreases the fitting time for the
  Legendre polynomial.
\end{itemize}


\paragraph{Functions}

The bad pixels are computed using the following function call

\begin{lstlisting}
cpl_mask * bpm_2d = hdrl_bpm_2d_compute(
        const hdrl_image        *   img_in,
        const hdrl_parameter    *   params)
\end{lstlisting}
                                                                      
\paragraph{Inputs}

The input parameters that need to be passed to the function for
smoothing the image by a filter are created by
\begin{lstlisting}
hdrl_parameter * params = hdrl_bpm_2d_parameter_create_filtersmooth(
        double              kappa_low,
        double              kappa_high,
        int                 maxiter,
        cpl_filter_mode     filter,
        cpl_border_mode     border,
        int                 smooth_x,
        int                 smooth_y);
\end{lstlisting}

The input parameters that need to be passed to the function for
smoothing the image by  fitting a polynomial 

\begin{lstlisting}
hdrl_parameter * params = hdrl_bpm_2d_parameter_create_legendresmooth(
        double              kappa_low,
        double              kappa_high,
        int                 maxiter,
        int                 steps_x,
        int                 steps_y,
        int                 filter_size_x,
        int                 filter_size_y,
        int                 order_x,
        int                 order_y);
\end{lstlisting}

\paragraph{Outputs}

The result is a mask of type cpl\_mask containing the newly found bad
pixels. Please note that already known bad pixels given to the routine
will not be included in the output mask.


\subsubsection{Bad-pixel detection on a stack of identical images}
 The recipe derives bad pixels on a stack of (identical) images 

\paragraph{Algorithm  - short description}

This algorithm assumes that the mean level of the different images are
the same, if this is not the case, the master-image as described below
will be biased. The algorithm first collapses the stack of images by
using the median to generate a master-image. Then it subtracts the
master image from each individual image and derives the bad pixels on
the residual-images by thresholding, i.e. all pixels exceeding the
threshold are considered as bad.

Three methods are currently available to derive the bad pixels on the
residual images:

\begin{itemize}
\item \verb,method = HDRL_BPM_3D_THRESHOLD_ABSOLUTE,: It uses
  \verb,kappa_low, and \verb,kappa_high, as absolute threshold.
\item \verb,method = HDRL_BPM_3D_THRESHOLD_RELATIVE,: It scales the
  measured rms on the residual-image with \verb,kappa_low, and
  \verb,kappa_high, and uses it as threshold. For the rms a properly
  scaled Median Absolute Deviation (MAD) is used.
\item \verb,method = HDRL_BPM_3D_THRESHOLD_ERROR,: It scales the
  propagated error of each individual pixel with \verb,kappa_low, and
  \verb,kappa_high, and uses it as threshold.
\end{itemize}

\paragraph{Functions}

The bad pixels are computed using the following function call

\begin{lstlisting}
cpl_imagelist * bpm_3d = hdrl_bpm_3d_compute(
        const hdrl_imagelist    *   imglist_in,
        const hdrl_parameter    *   params)
\end{lstlisting}

\paragraph{Inputs}

The input parameters that need to be passed to the function for are
created by
\begin{lstlisting}
hdrl_parameter * hdrl_bpm_3d_parameter_create(
        double              kappa_low,
        double              kappa_high,
        hdrl_bpm_3d_method  method);
\end{lstlisting}

\paragraph{Outputs}

The result is a cpl\_imagelist containing the newly found bad pixels
for each input image with the same pixel coding as for a cpl bad
pixel mask, i.e. 0 for good pixels and 1 for bad pixels. Please note
that already known bad pixels given to the routine will not be
included in the output mask.


\subsubsection{Bad-pixel detection on a sequence of images}
The recipe derives bad pixels on a sequence of images by fitting a
polynomial along each pixel sequence of the images.

\paragraph{Algorithm - short description}

Three options are available to convert the information from the fit
into a bad pixel map.
\begin{itemize}
\item chi-method: Relative cutoff on the chi distribution of all
  fits. Pixels with chi values which exceed of \verb,mean, $\pm$
  \verb,cutoff, $\times$ \verb,standard deviation, are considered bad.
\item fit-coefficient-method: Relative cutoff on the distribution of
  the fit coefficients. Pixels with fit coefficients which exceed of
  \verb,mean, $\pm$ \verb,cutoff, $\times$ \verb,standard deviation,
  are considered bad.
\item p-value-method: Pixels with low \verb,p-value,. When the errors
  of the pixels are correct the \verb,p-value, can be interpreted as
  the probability with which the pixel response fits the chosen model.
\end{itemize}

\paragraph{Functions}
The bad pixels are computed using the following function call

\begin{lstlisting}
cpl_error_code hdrl_bpm_fit_compute(
        const hdrl_parameter *  params,
        const hdrl_imagelist *  imglist_in,
        const cpl_vector     *  sample_position,
        cpl_image            ** out_mask)
\end{lstlisting}

\paragraph{Inputs}

The input parameters that need to be passed to the function when using
the chi-method are:
\begin{lstlisting}
hdrl_parameter * hdrl_bpm_fit_parameter_create_rel_chi(
        int                 degree,
        double              rel_chi_low,
        double              rel_chi_high);
\end{lstlisting}

The input parameters that need to be passed to the function when using
the fit-coefficient-method are:
\begin{lstlisting}
hdrl_parameter * hdrl_bpm_fit_parameter_create_rel_coef(
        int                 degree,
        double              rel_coef_low, 
        double              rel_coef_high);
\end{lstlisting}


The input parameters that need to be passed to the function when using
the p-value-method are:
\begin{lstlisting}
hdrl_parameter * hdrl_bpm_fit_parameter_create_pval(
        int                 degree, 
        double              pval);
\end{lstlisting}

\paragraph{Outputs}

The result is an integer cpl\_image (denoted out\_mask) containing the newly
found bad pixels for each input image.\\

For the \verb+rel_coef+ parameter the value of out\_mask encodes the
coefficient that was outside the relative threshold as a power of two.
E.g. if coefficient 0 and 3 of the fit where not within the threshold for a
pixel, it will have the value $2^0 + 2^3 = 9$.
All other parameters return an out\_mask with nonzero values marking pixels
outside the selection thresholds.

Please note that already known bad pixels given to the routine will
not be included in the output mask.

\subsubsection{Cosmic Ray Hits detection}

This recipes determines cosmic-ray-hits and/or bad-pixels on a single
image.

\paragraph{Algorithm  - short description}

This routine determines cosmic-ray-hits and/or bad-pixels following
the algorithm (LA-Cosmic) described in van Dokkum,
PASP,113,2001,p1420-27). HDRL does not use use error model as described in
the paper but the error image passed to the function. Moreover it does
several iterations with \verb,max_iter, defining an upper limit for
the number of iterations, i.e. the iteration stops if no new bad
pixels are found or \verb,max_iter, is reached. In each iteration this function
replaces the detected cosmic ray hits by the median of the surroundings
5x5 pixels taking into account the pixel quality information. The
input parameter \verb,sigma_lim, and \verb,f_lim, refere to
$\sigma_{lim}$ and $f_{lim}$ as described in the paper mentioned
above.

\paragraph{Functions}

The bad pixels/cosmics are computed using the following function call:

\begin{lstlisting}
cpl_mask * bpm_lacosmic = hdrl_lacosmic_edgedetect(
        const hdrl_image        *   img_in,
        const hdrl_parameter    *   params)
\end{lstlisting}
                                                                      
\paragraph{Inputs}

The input parameters that need to be passed to the function for are
created by:

\begin{lstlisting}
hdrl_parameter * hdrl_lacosmic_parameter_create(
        double              sigma_lim,
        double              f_lim,
        int                 max_iter);
\end{lstlisting}

\paragraph{Outputs}

The result is a mask of type cpl\_mask containing the newly found bad
pixels / cosmic-ray-hits. Please note that already known bad pixels given to the routine
will not be included in the output mask.

\subsubsection{Examples}
To follow Overscan layout we need to make an example of how to define
a bad pixel parameter creation for each supported method.


\subsubsection{Bad-Pixel Parameters in a Recipe}
\paragraph{Bad-pixel detection on a single image}

In the following example, we are using bias or dark images
where we set the kappa sigma parameters
kappa\_low and kappa\_high to 3, 3, and
maxiter to 2, 
the number of sampling coordinates in X and Y direction,
step\_x, step\_y to 20, 20,
the median filter X and Y sizes parameters  
filter\_size\_x, filter\_size\_y to 11, 11,
the Legendre polynomial order parameters 
order\_x, order\_y to 3, 3.


\begin{lstlisting}
hdrl_parameter * bpm_2d_param = 
    hdrl_bpm_2d_parameter_create_legendresmooth(3., 3., 2, 20, 20, 
    11, 11, 3, 3) ;
\end{lstlisting}

Then the bad pixel locations are stored in a mask by the following
function which receives as inputs an hdrl image, himage, and the 
previously defined bad pixels:

\begin{lstlisting}
cpl_mask * mask = hdrl_bpm_2d_compute(himage, bpm_2d_param) ;
\end{lstlisting}



\paragraph{Bad-pixel detection on a set of images}

In the following example, we are using linearity flat images
where we set the kappa sigma parameters
kappa\_low and kappa\_high to 3,3, and
maxiter to 2, 
the filter mode parameter to CPL\_FILTER\_MEDIAN,
the filter border parameter to 
CPL\_BORDER\_FILTER,
the smoothing kernel X and Y size parameters   
filter\_size\_x,filter\_size\_y to 3,3.


\begin{lstlisting}
hdrl_parameter * bpm_2d_param = 
    hdrl_bpm_2d_parameter_create_filtersmooth(3, 3, 2, 
    CPL_FILTER_MEDIAN, CPL_BORDER_FILTER, 3, 3) ;
\end{lstlisting}

Then the newly detected bad pixel locations are stored in a cpl\_imagelist
by the following function which receives as inputs an hdrl image,
himage, and the previously defined bad pixels:

\begin{lstlisting}
imlist = hdrl_bpm_3d_compute(himlist, bpm_param) ;
\end{lstlisting}

\paragraph{Cosmic Ray Hits detection}

In the following example, we are using raw frames representing deep exposures
where we set the sigma\_lim  parameter to 20,
the f\_lim parameter to 2.0,
and the max\_iter parameter to 5.

\begin{lstlisting}
hdrl_parameter * params =
            hdrl_lacosmic_parameter_create(20., 2.0, 5);
\end{lstlisting}

Then the cosmic ray hits locations are stored in a mask by the
following function which receives as inputs an hdrl image, himage, and 
the previously defined bad pixels:

\begin{lstlisting}
cpl_mask * mask = hdrl_lacosmic_edgedetect(image, params);
\end{lstlisting}

