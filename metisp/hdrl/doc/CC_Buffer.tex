In order to help handling datasets larger than the available memory and swap space
HDRL provides an experimental object \verb+hdrl_buffer+ which can be used to
allocate disk backed memory maps.

It provides following functions:
\begin{itemize}
\item \verb+hdrl_buffer_new()+: create a new buffer object
\item \verb+hdrl_buffer_allocate(size_t n)+: allocate \verb+n+ bytes of memory
\item \verb+hdrl_buffer_free(void *p)+: free memory pointed to by \verb+p+,
    depending on the position in the memory pool it might actually not free the
    memory.
\item \verb+hdrl_buffer_delete(hdrl_buffer *)+: delete the buffer object and all memory it
    allocated.
\end{itemize}

It is designed to be used like a memory pool (sometimes also called memory arena
or object stack) which allows fast allocation and deletion of many objects but
the ability to delete random objects in random order is limited.
It is not threadsafe so if it is used in a multithreaded environment each
thread must create and manage its own \verb+hdrl_buffer+.

For good performance usage of this memory should be blocked to chunks of the
available RAM to avoid unnecessary disk IO.

Note that on 32 bit machines one is still limited to 4GiB of allocations due to
the address space limitations, on 64 bit machines one is only limited by the
available disk space.

Memory maps can be slower than swap space backed memory (due to higher
complexity and the kernel flushing dirty pages to disk periodically).
In order to help test the impact the buffer object will use \verb+cpl_malloc+
for memory allocation if the environment variable \verb+HDRL_BUFFER_MALLOC+ is
set.
On systems with sufficient amounts of swap space this can be the more efficient
mode of operation.

\subsubsection{Examples}
\begin{lstlisting}
hdrl_buffer * buf = hdrl_buffer_new();
/* allocate 2 GiB of memory */
void * mem = hdrl_buffer_allocate(1ull << 31ull);
void * mem2 = hdrl_buffer_allocate(100);
do_stuff(mem, mem2);
/* free all memory again */
hdrl_buffer_delete(buf);
\end{lstlisting}

HDRL provides a function \verb+hdrl_image_new_from_buffer+ to allocate a
\verb+hdrl_image+ from a \verb+hdrl_buffer+ which can be treated like a normal
image, including deletion with \verb+hdrl_image_delete+ and creation of views.

\begin{lstlisting}
hdrl_buffer * buf = hdrl_buffer_new();
hdrl_imagelist * hl = hdrl_imagelist_new()
for (size_t i = 0; i < 1000; i++) {
    hdrl_image * img = hdrl_image_new_from_buffer(2000, 2000, buf);
    hdrl_image_add_scalar(img, 5., 3.);
    hdrl_imagelist_set(hl, img, i);
}
hdrl_image * out; cpl_image * contrib;
hdrl_imagelist_collapse(hl, HDRL_COLLAPSE_MEDIAN, &out, &contrib);

/* This is very inefficient as it will need to load the data from disk
   twice, instead one should loop over the images or use the iterator
   interface to work in RAM sized blocks */
hdrl_imagelist_add_scalar(hl, 5., 3.);
hdrl_imagelist_mul_scalar(hl, 5., 3.);

/* free all memory again, note the imagelist has to be explicitly deleted
   because structure has not been allocated from the buffer */
hdrl_imagelist_delete(hl);
hdrl_buffer_delete(buf);
\end{lstlisting}

