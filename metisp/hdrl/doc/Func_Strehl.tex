\subsection{Strehl}
\label{strehl:main}

The most commonly used metric for evaluating the AO correction is the Strehl
ratio. The Strehl ratio is defined as the ratio of the peak image intensity
from a point source compared to the maximum attainable intensity using an
ideal optical system limited only by diffraction over the telescope aperture.
The Strehl ratio is very frequently used to perform the quality control of
the scientific data obtained with the AO assisted instrumentation.

 
\subsubsection{Algorithm - short description}

 The implemented HDRL function assumes that the input raw image
 contains a single object and it is
 already pre-processed. This means that the instrument signature (bad pixels,
CRH detection, bias level and dark current subtraction,  flat fielding
etc.) has been removed and the contribute from the sky background subtracted.
 
 This function allows also the user to correct a residual
 local sky background evaluated in an annular region centred on the peak of the
 (expected single) object PSF, by setting the values of the
 corresponding parameters, which control the minimum and maximum radii
 of the annular region,  \verb+bkg_radius_low+,
 \verb+bkg_radius_high+ (in arcsec units). 

 The PSF is identified and its integrated flux, whose determination is
 controlled by the parameter \verb+flux_radius+, in arcsec, is normalized to 1. 

 Next the PSF barycentre is computed and used to generate the
 theoretical normalised PSF. This depends on: 

 i) the telescope pupil
 characteristics (telescope radius and central obstruction parameters, 
 \verb+m1_radius+, \verb+m2_radius+); 

 ii) the wavelength (parameter \verb+wavelength+ ) at which the image 
 has been obtained, all expressed in m units; 
 
 iii) the detector pixel scale on 
 sky in arcsec (parameters \verb+pixel_scale_x+, \verb+pixel_scale_y+). 
 
 Then the Strehl ratio is obtained by dividing the maximum intensity
 of the image PSF by the maximum intensity of the ideal PSF. The
 associated error is finally computed.
 
\subsubsection{Functions}

The Strehl of an image and its error are computed by the following
function call:

\begin{lstlisting}
hdrl_strehl_result
hdrl_strehl_compute(const hdrl_image * himg, hdrl_parameter* params)

\end{lstlisting}
             
where \verb+hima+ is the HDRL input image on which the Strehl has to
be computed, and \verb+params+ is a \verb+hdrl_parameter+ structure
specifying the parameters controlling the Strehl computation.
           
\subsubsection{Inputs}
\label{sec:algorithms:strehl:inputs}
The input parameters to be passed to the function to compute the
Strehl are created by the following function call:

\begin{lstlisting}
hdrl_parameter * params = hdrl_strehl_parameter_create(
        double              wavelength, 
        double              m1_radius, 
        double              m2_radius, 
        double              pixel_scale_x, 
        double              pixel_scale_y, 
        double              flux_radius, 
        double              bkg_radius_low, 
        double              bkg_radius_high);
\end{lstlisting}

\subsubsection{Outputs}


The function creates an output \verb+hdrl_strehl_result+ structure
containing the following quantities:

\begin{lstlisting}
typedef struct {
    /* computed strehl value and its propagated error */
    hdrl_value strehl_value;
    /* computed x and y position of the star peak */
    double star_x, star_y;
    /* star peak and its propagated error */
    hdrl_value star_peak;
    /* star flux and its propagated error */
    hdrl_value star_flux;
    /* star background and its propagated error */
    hdrl_value star_background;
    /* star background error estimated from image
     * on normal data sqrt(pi / 2) larger that star_background error as 
     * it is estimated via a median */
    double computed_background_error;
    /* number of pixels used for background estimation */
    size_t nbackground_pixels;
} hdrl_strehl_result;
\end{lstlisting}

\subsubsection{Examples}
\label{sec:algorithms:strehl:example}

If the user would like to compute the Strehl of an image observed at
1.635 $\mu$m with a telescope 
\footnote{The telescope diameter and central
  obstruction values indicated in this example refer to the Palomar 
telescope.} 
of 5.08 m diameter with a central obstruction of 1.8288 m,
and the detector pixel size on Sky is
$33.1932 \mbox{ mas} \times 33.1932 \mbox{ mas}$ (1 mas = 0.001 arcsec), 
integrating the observed image PSF over
a radius of 1.5 arcsec where the sky background is computed on the
annular region defined by the internal and external radii respectively
of 1.5 and 2.5 arcsec, the following function calls would do the job:  

\begin{lstlisting}

  hdrl_strehl_result          strehl;

  hdrl_parameter * params = hdrl_strehl_parameter_create(1.635e-6,
  5.08/2, 1.8288/2, 0.0331932, 0.0331932, 1.5, 1.5, 2.5);

  strehl = hdrl_strehl_compute(hima, params);
\end{lstlisting}

Please note that for ESO-VLT instruments, the UT telescope diameter is
8.115 m, while the central obstruction depends on the instrument used,
some instrument deploy a buffle, some not. The user is recommended to
adopt the following obstruction values:
\begin{itemize}
\item 1.116m for UT1-3 with no baffle (KMOS, NACO, VISIR, SPHERE)
\item 1.550m for UT1-3 with baffle (FORS, X-shooter, UVES, FLAMES)
\item 1.550m for UT4 (all instruments) 
\end{itemize}

\subsubsection{Strehl Parameters in a Recipe}
The computation of the Strehl is usually part of a more complex recipe
to reduce standard stars or science objects. Concerning the parameters
to be defined to compute this quantity the user has to create a 
\verb+hdrl_parameter+ object as described in 
Section~\ref{sec:algorithms:strehl:inputs}.
