\subsection{Bias}
\label{sec:algorithms:bias:main}

Bias frames give the read out of the CCD detector for zero integration 
time with the shutter closed. That means that any frame acquisition
with shutter open (either a calibration or a science observation) need
to be bias corrected. Bias frames are acquired in optical (e.g UVB or
VIS) domain.
In order to have a more accurate estimation of the bias, they are typically
taken as a set of (usually five) bias frames that have to be combined into
a master to increase the S/N level.

\subsubsection{Algorithm - short description}
\label{sec:algorithms:bias:algo}

Combining multiple images to one master image is handled by
\verb+hdrl_imagelist_collapse+~(see
section~\ref{sec:imagelist:collapsing}).  In order to be robust
against outliers (like cosmic ray hits) it is recommended to use the
median or sigma clipping collapse methods. Both methods are robust
against outliers but the median has a low statistical efficiency so
its result will have a higher uncertainty than collapsing images with
few outliers via sigma clipping.

\subsubsection{Functions}
\label{sec:algorithms:bias:functions}

The master bias is computed by using the general
\verb+hdrl_imagelist_collapse+ function:

\begin{lstlisting}
cpl_error_code hdrl_imagelist_collapse(
        const hdrl_imagelist    *   himlist,
        const hdrl_parameter    *   param,
        hdrl_image              **  out,
        cpl_image               **  contrib)
\end{lstlisting}
The \verb+out+ and \verb+contrib+ pointers are filled with allocated
result images which must be deleted by the user when not required
anymore.

\subsubsection{Inputs}
\label{sec:algorithms:bias:inputs}

The input parameter \verb+param+ that needs to be passed to the
collapse function is a \verb+hdrl_parameter+ who's type defines the
exact collapse method which is applied. Currently available are:
\begin{itemize}
\item \verb+hdrl_collapse_mean_parameter_create()+
\item \verb+hdrl_collapse_median_parameter_create()+
\item \verb+hdrl_collapse_weighted_mean_parameter_create()+
\item \verb+hdrl_collapse_sigclip_parameter_create(kappa_low, kappa_high, niter)+
\item \verb+hdrl_collapse_minmax_parameter_create(nlow, nhigh)+
\item \verb+hdrl_collapse_mode_parameter_create(histo_min, histo_max, bin_size,+\\\verb+mode_method, error_niter)+

\end{itemize}

Note that these parameters are dynamically allocated and must be
deleted again.  For convenience HDRL provides preallocated parameters
for collapses which do not require any additional parameter. These
can be used anywhere a regular parameter can be used and deletion with
\verb+hdrl_parameter_delete+ has no effect on them. See
section~\ref{sec:imagelist:collapsing} for more details.

\subsubsection{Outputs}
\label{sec:algorithms:bias:outputs}

In most cases a collapse operation only has one or two outputs, the
resulting master-image (\verb+out+) and a integer contribution map
(\verb+contrib+) counting how many values contributed to each pixel of
the image. The sigma clipping and minmax collapse method can
additionally return the low and hight rejection thresholds used to
calculate the mean.  These results can only be obtained with
\verb+hdrl_imagelist_collapse_sigclip+ and
\verb+hdrl_imagelist_collapse_minmax+ See
section~\ref{sec:imagelist:collapsing} for the usage of these
functions.

\subsubsection{Examples}
\begin{lstlisting}
hdrl_image* master;
cpl_image* contrib_map;
/* initialize the parameters of a sigma clipping combination */
hdrl_parameter * sigclip_par;
sigclip_par = hdrl_collapse_sigclip_parameter_create(3., 3., 5)

/* do the combination */
hdrl_imagelist_collapse(data, sigclip_par ,&master, &contrib_map);

/* cleanup */
hdrl_parameter_delete(sigclip_par) ;
hdrl_imagelist_delete(data);
\end{lstlisting}

\subsubsection{Master Bias Parameters in a Recipe}
\label{sec:algorithms:bias:parameters}

In the context of a recipe, all input parameters needed by the master bias
computation may be provided as input parameters.
The creation of \verb+cpl_parameter+ objects for the recipe interface is
facilitated by:
\begin{itemize}
\item \verb+hdrl_collapse_parameter_create_parlist+
\item \verb+hdrl_collapse_parameter_parse_parlist+
\end{itemize}
The former provides a \verb+cpl_parameterlist+ which can be appended to the
recipe parameter list, the latter parses the recipe parameter list for the
previously added parameters and creates a ready to use \verb+hdrl_parameter+.

\begin{lstlisting}
/* prepare defaults for recipe */

hdrl_parameter * sigclip_def = 
    hdrl_collapse_sigclip_parameter_create(3., 3., 5);
hdrl_parameter * minmax_def =
    hdrl_collapse_minmax_parameter_create(1., 1.);
hdrl_parameter * mode_def =
hdrl_collapse_mode_parameter_create(10., 1., 0., HDRL_MODE_MEDIAN, 0);
    
/*create parameters with context RECIPE_NAME under the collapse hierarchy*/

cpl_parameterlist * pbiascollapse =hdrl_collapse_parameter_create_parlist(
RECIPE_NAME, "collapse", "MEDIAN", sigclip_def, minmax_def, mode_def) ;

hdrl_parameter_delete(sigclip_def); 
hdrl_parameter_delete(minmax_def);
hdrl_parameter_delete(mode_def);

/* add to recipe parameter list, change aliases,
disable not required ones etc. */

for (cpl_parameter * p = cpl_parameterlist_get_first(pbiascollapse) ;
        p != NULL; p = cpl_parameterlist_get_next(pbiascollapse))
    cpl_parameterlist_append(self, cpl_parameter_duplicate(p));
cpl_parameterlist_delete(pbiascollapse);

\end{lstlisting}

This will create a esorex man-page as follows:
{\footnotesize
\begin{verbatim}
esorex --man-page hdrldemo_bias
...

--collapse.method             : Method used for collapsing the data. <MEAN |
                                WEIGHTED_MEAN | MEDIAN | SIGCLIP | MINMAX | 
                                MODE> [MEDIAN]
--collapse.sigclip.kappa-low  : Low kappa factor for kappa-sigma
                                clipping algorithm. [3.0]
--collapse.sigclip.kappa-high : High kappa factor for kappa-sigma
                                clipping algorithm. [3.0]
--collapse.sigclip.niter      : Maximum number of clipping iterations for
                                kappa-sigma clipping. [5]
--collapse.minmax.nlow        : Low number of pixels to reject for the minmax
                                clipping algorithm. [1.0]
--collapse.minmax.nhigh       : High number of pixels to reject for the
                                minmax clipping algorithm. [1.0]
--collapse.mode.histo-min     : Minimum pixel value to accept for mode
                                computation. [10.0]
--collapse.mode.histo-max     : Maximum pixel value to accept for mode
                                computation. [1.0]
--collapse.mode.bin-size      : Binsize of the histogram. [0.0]
--collapse.mode.method        : Mode method (algorithm) to use. <MEDIAN |
                                WEIGHTED | FIT> [MEDIAN]
--collapse.mode.error-niter   : Iterations to compute the mode
                                error. [0]
...
\end{verbatim}
}
