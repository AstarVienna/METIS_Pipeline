\section{Efficiency Calculation}
\label{chap:algorithms:efficiency}

\subsection{Introduction}
In order to determine the efficiency of an instrument (including
telescope and detector) spectrophotometric standard stars are observed
on photometric nights and with setups that collect as much as possible
(e.g. wide slits for slit spectrographs). After correcting for atmopsheric
effects, photon energy, and mirror collecting 
area the ratio between the observed spectrum and the reference
spectrum of the standard star will provide the efficiency of the system.

The observed spectra of the standard
stars are first corrected for atmospheric effects:
\begin{enumerate}
\item The continuous {\em atmospheric extinction} is usually corrected
  with a standard extinction curve obtained for a given observatory,
  scaled with the airmass of the observation. During non-photometric
  conditions the actual extinction may be higher than the one
  tabulated in the extinction curve. The impact of this correction
  decreases with increasing wavelength. 
\item Wavelength above about 600\,nm\footnote{Ozone absorption affects
    the bluest part of ground-based data at about 310--330\,nm, but these are
    noticeable only in high S/N data.} are affected by {\em telluric
    absorption lines}. If these are not corrected the efficiency
  will contain a mixture of instrument response and atmospheric
  response, especially if the reference spectra are free from telluirc
  lines (space-based data or model spectra). Reference spectra for
  many standard stars used for ESO instruments (e.g. UVES, FORS2,
  VIMOS) contain telluric lines
  (e.g. \cite{Hamuy+92,Hamuy+94}). For such reference spectra a
  telluric correction of the observed spectra is useless and one
  should instead interpolate the efficiency across the regions of
  significant telluric absorption.
\end{enumerate}

Possibly the wavelength scale between the observed standard star
spectrum and the reference one needs to be aligned, either to correct
differences in radial velocity between the observed and the reference
spectrum in case of model spectra (e.g. X-shooter, SINFONI) and/or to
correct shifts introduced by imprecise positioning of the flux
standard star in the slit (e.g. VIMOS). If the differences in
wavelength become significant for a given resolution the division of
the reference spectrum by the observed one will create pseudo-P\,Cyg
prodiles, which may distort the efficiency.

These requirements results in the algorithmic steps described in
Sect.\,\ref{efficiency:main}.

\subsection{Testing}
Once the response determination has been implemented in HDRL and
incorporated in the SINFONI pipeline QCG tested it by comparing the
resulting values to the SPIFFI commissioning report
(VLT-TRE-MPE-14720-8003) and found good agreement. The efficiency is
trended at \url{http://www.eso.org/observing/dfo/quality/SINFONI/reports/HEALTH/trend_report_EFFICIENCY_HC.html}
