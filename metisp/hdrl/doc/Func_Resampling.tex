\newpage

\subsection{Image and Cube Resampling}
\label{resampling:main}

A common problem in astronomy is the resampling of 2D images and 3D cubes onto a
common grid. Ideally, this is done only once in the data reduction workflow as
each sub-pixel resampling redistributes the flux and creates correlation. In the
resampling \HDRL project we would like to address two different
scenarios. Assuming that one would like to re-grid and stack 10 dithered images/cubes
with pre-determined World Coordinate System (WCS). Then one should be able to:

\begin{itemize}
\item Combine all images/cubes in a single cloud of points (we use a
table for this step - see below) and when creating the resampled output
image/cube all information is used in a single interpolation step, i.e. no
image-by-image interpolation is done.
\item Resample each individual image/cube on the same output grid (image by image
interpolation) and combine the single resampled images in a second step with
hdrl stacking methods.
\end{itemize}
Both scenarios interpolate only once, but the second scenario is important in
case not all bad pixels could be properly determined and inserted in the bad
pixel mask upfront. Then the implemented HDRL pixel resampling function offers the possibility to use a robust pixel
combination method (median or $\kappa\sigma$~clipping) to combine the
images/cubes. 

\subsubsection{Algorithm - short description}
\label{resampling:algorithms}

The implemented interpolation algorithms are based on the MUSE pipeline and work
for 2D images and 3D cubes. The 2D and 3D interpolation is done in 2-dimensional
and 3-dimensional spaces, respectively. Currently there are six different
interpolation methods implemented:

\begin{center}
\begin{itemize}
\item {\bf Nearest:} Nearest  resampling
\item {\bf Linear:} Weighted resampling using an inverse distance weighting function
\item {\bf Quadratic:} Weighted resampling using a quadratic inverse distance weighting function
\item {\bf Renka:} Weighted resampling using a Renka weighting function
\item {\bf Drizzle:} Weighted resampling using a drizzle-like weighting scheme
\item {\bf Lanczos:} Weighted resampling using a lanczos-like restricted sinc as weighting function
\end{itemize}
\end{center}

\paragraph{Nearest Neighbour}
\label{resampling:algorithms:nearest}
The algorithm does not use any weighting function but simply uses the value of
the nearest neighbour inside an output voxel\footnote{In 3D computer graphics, a voxel represents a value on a regular grid in three-dimensional space. See \http{https://en.wikipedia.org/wiki/Voxel} for more information.} centre as the final output value. If
there is no nearest neighbour inside the voxel (but e.g. only outside), the voxel
is marked as bad. This speeds up the algorithm considerably. There are no
control parameter for this method.

\paragraph{Linear}
\label{resampling:algorithms:linear}
The algorithm uses a linear inverse distance weighting function ($\frac{1}{r}$)
for the interpolation. The parameter \verb+loop_distance+ controls the number of
surrounding pixels that are taken into account on the final grid, e.g. a
\verb+loop_distance+ of 1 uses 3 pixels (x - 1, x, x + 1) in each dimension,
i.e.  9 in total for a 2D image and 27 in total for a 3D cube. Moreover, if the
parameter \verb+use_errorweights+ is set to TRUE, an additional weight, defined
as 1/variance, is taken into account\footnotemark.
 
\paragraph{Quadratic}
\label{resampling:algorithms:quadratic}
The algorithm uses a quadratic inverse distance weighting function
($\frac{1}{r^2}$) for the interpolation. The parameter \verb+loop_distance+
controls the number of surrounding pixels that are taken into account on the
final grid, e.g. a \verb+loop_distance+ of 1 uses 3 pixels (x - 1, x, x + 1) in
each dimension, i.e.  9 in total for a 2D image and 27 in total for a 3D
cube. Moreover, if the parameter \verb+use_errorweights+ is set to
TRUE, an additional weight, defined as 1/variance,
is taken into account\footnotemark[\value{footnote}].
 
\paragraph{Renka}
\label{resampling:algorithms:renka}
The algorithm uses a modified Shepard-like distance weighting function following
Renka for the interpolation. The parameter \verb+critical_radius+ defines the
distance beyond which the weights are set to 0 and the pixels are therefore not
taken into account. The parameter \verb+loop_distance+ controls the number of
surrounding pixels that are taken into account on the final grid, e.g. a
\verb+loop_distance+ of 1 uses 3 pixels (x - 1, x, x + 1) in each dimension,
i.e.  9 in total for a 2D image and 27 in total for a 3D cube. Moreover, if the
parameter \verb+use_errorweights+ is set to TRUE,
an additional weight, defined as 1/variance, is taken into account\footnotemark[\value{footnote}].

\paragraph{Drizzle}
\label{resampling:algorithms:drizzle}
The algorithm uses a drizzle-like distance weighting function for the
interpolation. The down-scaling factors \verb+pix_frac_x+, \verb+pix_frac_y+,
and \verb+pix_frac_lambda+, for x, y, and wavelength direction control the
percentage of flux of the original pixel/voxel that drizzles into the target
pixel/voxel. The parameter \verb+loop_distance+ controls the number of
surrounding pixels that are taken into account on the final grid, e.g. a
\verb+loop_distance+ of 1 uses 3 pixels (x - 1, x, x + 1) in each dimension,
i.e.  9 in total for a 2D image and 27 in total for a 3D cube. Moreover, if the
parameter \verb+use_errorweights+ is set to TRUE,
an additional weight defined as 1/variance is taken into account\footnotemark[\value{footnote}].

\paragraph{Lanczos}
\label{resampling:algorithms:lanczos}
The algorithm uses a restricted $sinc$ distance weighting function
($\frac{sinc(r)}{sinc(r/kernel\_size)}$) with the kernel size given by the
parameter \verb+kernel_size+ for the interpolation. The parameter
\verb+loop_distance+ controls the number of surrounding pixels that are taken
into account on the final grid, e.g. a \verb+loop_distance+ of 1 uses 3 pixels
(x - 1, x, x + 1) in each dimension, i.e.  9 in total for a 2D image and 27 in
total for a 3D cube. Moreover, if the parameter \verb+use_errorweights+ is set
to TRUE, an additional weight defined as
1/variance is taken into account\footnotemark[\value{footnote}].



\footnotetext{Please note that only if the variance of a pixel is $> 0$ this
  additional weight is applied}

\subsubsection{Functions}
The input data is resampled using the following function call
\begin{lstlisting}
hdrl_resample_result * hdrl_resample_compute(
        const cpl_table     * restable,
        hdrl_parameter      * method,
        hdrl_parameter      * outputgrid,
        const cpl_wcs       * wcs);
\end{lstlisting}

The routine is not directly working on an image or a cube but on a table
(\verb+restable+). For 2D images, the table is created by the function
\verb+hdrl_resample_image_to_table()+, whereas for a 3D data cube the
function \verb+hdrl_resample_imagelist_to_table()+ generates the appropriate
table. In the case that many images/cubes have to be combined into a single
mosaic, the two functions can be called multiple times and the returned tables
should be merged into a single table by using \verb+cpl_table_insert()+.

\subsubsection{Inputs}
The \verb+hdrl_resample_compute+ function has four arguments:

\paragraph*{restable}
The first function argument \verb+restable+ is a \verb+cpl_table+ with
information on the data to be resampled.  It can be created in case of a 2D
image or a 3D data cube with the following function call:

\begin{lstlisting}
cpl_table * hdrl_resample_image_to_table(
        const hdrl_image     * hima,
        const cpl_wcs        * wcs);  
\end{lstlisting}

\begin{lstlisting}
cpl_table * hdrl_resample_imagelist_to_table(
        const hdrl_imagelist * himlist,
        const cpl_wcs        * wcs);
\end{lstlisting}

where \verb+ hima+ is a hdrl image containing the 2D image and \verb+ himlist+ a
hdrl imagelist containing the 3D cube that should be resampled. The \verb+wcs+
is a cpl wcs object that encodes the world coordinate system of the given
image/cube. 


In case the above mentioned functions can not be used to create the table
(e.g. the pixel to sky mapping is very complex and can not be encoded by the cpl
wcs object) the pipeline developer has to create and fill the table. The table should be
created and initialised as follows:

{
\footnotesize
\begin{lstlisting}
  tab = cpl_table_new(size);
  
  /* create the table columns */
  cpl_table_new_column(tab, HDRL_RESAMPLE_TABLE_RA,     HDRL_RESAMPLE_TABLE_RA_TYPE);
  cpl_table_new_column(tab, HDRL_RESAMPLE_TABLE_DEC,    HDRL_RESAMPLE_TABLE_DEC_TYPE);
  cpl_table_new_column(tab, HDRL_RESAMPLE_TABLE_LAMBDA, HDRL_RESAMPLE_TABLE_LAMBDA_TYPE);
  cpl_table_new_column(tab, HDRL_RESAMPLE_TABLE_DATA,   HDRL_RESAMPLE_TABLE_DATA_TYPE);
  cpl_table_new_column(tab, HDRL_RESAMPLE_TABLE_BPM,    HDRL_RESAMPLE_TABLE_BPM_TYPE);
  cpl_table_new_column(tab, HDRL_RESAMPLE_TABLE_ERRORS, HDRL_RESAMPLE_TABLE_ERRORS_TYPE);

  /* init column values */
  cpl_table_fill_column_window_double(tab, HDRL_RESAMPLE_TABLE_RA,    0, size, 0.);
  cpl_table_fill_column_window_double(tab, HDRL_RESAMPLE_TABLE_DEC,   0, size, 0.);
  cpl_table_fill_column_window_double(tab, HDRL_RESAMPLE_TABLE_LAMBDA,0, size, 0.);
  cpl_table_fill_column_window_double(tab, HDRL_RESAMPLE_TABLE_DATA,  0, size, 0.);
  cpl_table_fill_column_window_int(tab,    HDRL_RESAMPLE_TABLE_BPM,   0, size, 0);
  cpl_table_fill_column_window_double(tab, HDRL_RESAMPLE_TABLE_ERRORS,0, size, 0.);
\end{lstlisting}
}

Please note that the interpolation function \verb+hdrl_resample_compute()+ was
only tested with tables created by \verb+hdrl_resample_image_to_table()+ or
\verb+hdrl_resample_imagelist_to_table()+, i.e. mappings encoded by a cpl wcs
object. Creating and using a handcrafted table could lead to side effects that the
interpolation function cannot properly handle.


\paragraph*{method}
Currently six different interpolation methods can be used to resample the data. The corresponding hdrl parameter \verb+method+ is created executing the
appropriate function call:

\begin{itemize}

\item {\bf Nearest:} Nearest neighbour resampling (see
section~\ref{resampling:algorithms:nearest}) 
\begin{lstlisting}
hdrl_parameter * hdrl_resample_parameter_create_nearest(void);
\end{lstlisting}

\item {\bf Linear:} Weighted resampling using inverse distance weighting
function (see section~\ref{resampling:algorithms:linear})
\begin{lstlisting}
hdrl_parameter * hdrl_resample_parameter_create_linear(
        const int    loop_distance,
        cpl_boolean  use_errorweights);  
\end{lstlisting}

\verb+loop_distance+: in pixel units ($>= 0$) \\ 
\verb+use_errorweights+: cpl\_boolean (CPL\_TRUE or CPL\_FALSE)  

\item {\bf Quadratic:} Weighted resampling using quadratic inverse distance
        weighting function (see section~\ref{resampling:algorithms:quadratic})
\begin{lstlisting}
hdrl_parameter * hdrl_resample_parameter_create_quadratic(
        const int    loop_distance,
        cpl_boolean  use_errorweights);  
\end{lstlisting}

\verb+loop_distance+: in pixel units ($>= 0$) \\ 
\verb+use_errorweights+: cpl\_boolean (CPL\_TRUE or CPL\_FALSE)  

\item {\bf Renka:} Weighted resampling using Renka weighting function (see
        section~\ref{resampling:algorithms:renka}) 
\begin{lstlisting}
hdrl_parameter * hdrl_resample_parameter_create_renka(
        const int    loop_distance,
        cpl_boolean  use_errorweights,
        const double critical_radius);  
\end{lstlisting}

\verb+loop_distance+: in pixel units ($>= 0$) \\ 
\verb+use_errorweights+: cpl\_boolean (CPL\_TRUE or CPL\_FALSE)\\  
\verb+critical_radius+: in pixel units ($> 0$) 

\item {\bf Drizzle:} Weighted resampling using a drizzle-like weighting scheme
        (see section~\ref{resampling:algorithms:drizzle}) 
\begin{lstlisting}
hdrl_parameter * hdrl_resample_parameter_create_drizzle(
        const int    loop_distance,
        cpl_boolean  use_errorweights,
        const double pix_frac_x,
        const double pix_frac_y,
        const double pix_frac_lambda);
  
\end{lstlisting}
      
\verb+loop_distance+: in pixel units ($>= 0$) \\ 
\verb+use_errorweights+: cpl\_boolean (CPL\_TRUE or CPL\_FALSE)\\  
\verb+pix_frac_x+: fraction of pixel/voxel ($> 0$) \\ 
\verb+pix_frac_y+: fraction of pixel/voxel ($> 0$) \\ 
\verb+pix_frac_lambda+: fraction of pixel/voxel ($> 0$) \\ 

\item {\bf Lanczos:} Weighted resampling using a lanczos-like restricted sinc
        for weighting (see section~\ref{resampling:algorithms:lanczos}) 
\begin{lstlisting}
hdrl_parameter * hdrl_resample_parameter_create_lanczos(
        const int   loop_distance,
        cpl_boolean use_errorweights,
        const int   kernel_size);
\end{lstlisting}

\verb+loop_distance+: in pixel units ($>= 0$) \\ 
\verb+use_errorweights+: cpl\_boolean (CPL\_TRUE or CPL\_FALSE)\\  
\verb+kernel_size+: in pixel units ($> 0$) 
\end{itemize}


\paragraph*{outputgrid}
The \verb+outputgrid+ parameter defines basic properties of the resampled
image/cube. Depending on the input data (image or cube) the \verb+outputgrid+
parameter should be created with the appropriate function call,
i.e. \\ \verb+hdrl_resample_parameter_create_outgrid2D_userdef()+ for an image
and \\ \verb+hdrl_resample_parameter_create_outgrid3D_userdef()+ for a
cube. For convenience, the \verb+outputgrid+ parameter can also be created with
\verb+hdrl_resample_parameter_create_outgrid2D()+ or
\verb+hdrl_resample_parameter_create_outgrid3D()+. In this case, only the step
sizes in right ascension (\verb+delta_ra+), declination (\verb+delta_dec+) and
wavelength (\verb+delta_lambda+) of the output image/cube have to be given. All
the rest is automatically derived from the data by the
\verb+hdrl_resample_compute()+ function. The four functions read as follows:

\begin{lstlisting}
hdrl_parameter * hdrl_resample_parameter_create_outgrid2D_userdef(
        const double delta_ra,
        const double delta_dec,
        const double ra_min,
        const double ra_max,
        const double dec_min,
        const double dec_max,
        const double fieldmargin);  
\end{lstlisting}

\begin{lstlisting}
hdrl_parameter * hdrl_resample_parameter_create_outgrid3D_userdef(
        const double delta_ra,
        const double delta_dec,
        const double delta_lambda,
        const double ra_min,
        const double ra_max,
        const double dec_min,
        const double dec_max,
        const double lambda_min,
        const double lambda_max,
        const double fieldmargin);
\end{lstlisting}

where \verb+delta_ra+, \verb+delta_dec+, and \verb+delta_lambda+ are the output
grid steps in right ascension, declination and wavelength. Moreover, these two
functions allow the user to set the full output grid by defining
\verb+ra_min+, \verb+ra_max+,  \verb+dec_min+, \verb+dec_max+, and
\verb+lambda_min+, \verb+lambda_max+. These parameters define the minimum and
maximum boundaries of the output image in terms of right ascension, declination
and wavelength. Furthermore, with the parameter \verb+fieldmargin+ one can add a
margin to the output image in all spacial directions.

The parameters \verb+delta_ra+, \verb+delta_dec+, \verb+ra_min+, \verb+ra_max+,
\verb+dec_min+, and \verb+dec_max+ have to be given in degrees [deg], whereas the
parameters \verb+delta_lambda+, \verb+lambda_min+, and \verb+lambda_max+ have to
be given in meter [m]. The \verb+fieldmargin+ has to be given in percent with 0
adding no margin to the output image/cube

For convenience, the \HDRL also offers the following convenience functions that
calculate the boundaries of the output image/cube from the data and
automatically set the fieldmargin:


\begin{lstlisting}
hdrl_parameter * hdrl_resample_parameter_create_outgrid2D(
        const double delta_ra,
        const double delta_dec);  
\end{lstlisting}

\begin{lstlisting}
hdrl_parameter * hdrl_resample_parameter_create_outgrid3D(
        const double delta_ra,
        const double delta_dec,
        const double delta_lambda);  
\end{lstlisting}

\paragraph*{wcs}
This input parameter specifies the World Coordinate System which should be
representative of the images that are going to be resampled. The resampling
functions uses the wcs input (CD matrix) mostly do determine the scales between
the input and output grid. Please note, that in case the user would like to
combine images/cubes with substantially different pixel sizes into a single
output image/cube, the single tables have to be properly scaled to the same
scales before merging them into the final table.


\subsubsection{Outputs}
\label{sec:algorithms:resampling:outputs}

The \verb+hdrl_resample_compute()+ function creates an output
\verb+hdrl_resample_result+ structure containing the following quantities:

\begin{lstlisting}
typedef struct {
        /* cpl propertylist containing the output wcs information */
        cpl_propertylist * header;
        /* hdrl imagelist containing the data, the errors, and the bpm */
        hdrl_imagelist   * himlist;
} hdrl_resample_result;
\end{lstlisting}

This structure contains for convenience a cpl propertylist \verb+header+ with
the output wcs that can directly be used by the cpl saving functions to save the
image/cube. In case the wcs object itself is needed, the cpl function
\verb+cpl_wcs_new_from_propertylist(header)+ should be used. Moreover, the
structure contains the resampled image/cube as hdrl imagelist \verb+himlist+. If
a 2D image was resampled, the returned imagelist is of size 1 and the resampled
hdrl image can be accessed with \verb+hdrl_imagelist_get(himlist, 0)+.

The \verb+hdrl_resample_result+ object has to be deleted by the user
when not required any more by calling the function
\verb+hdrl_resample_result_delete()+.

Please note that the output is using the Gnomonic projection\footnote{see
  \url{https://en.wikipedia.org/wiki/Gnomonic_projection}}. The gnomonic
projection is from the centre of a sphere to a plane tangential to the
sphere. The sphere and the plane touch at the tangent point. This in encoded in
the header file by:
\begin{lstlisting}
  CTYPE1  = 'RA---TAN'           / Gnomonic projection
  CTYPE2  = 'DEC--TAN'           / Gnomonic projection
\end{lstlisting}

\newpage
\subsubsection{Example}
\label{sec:algorithms:resampling:example}
This section briefly describes a typical cube regridding example .
{\footnotesize
  \begin{lstlisting}
  /* Load input data, errors and bpm frames */
  cpl_imagelist* imlist_data  = cpl_imagelist_load(name, CPL_TYPE_DOUBLE, extension_data);
  cpl_imagelist* imlist_error = cpl_imagelist_load(name, CPL_TYPE_DOUBLE, extension_error);
  cpl_imagelist* imlist_bpm   = cpl_imagelist_load(name, CPL_TYPE_INT, extension_bpm);
  
  /* Load input wcs */
  cpl_propertylist *xheader_data = cpl_propertylist_load(name, extension_data);
  cpl_wcs *wcs = cpl_wcs_new_from_propertylist(xheader_data);
  cpl_propertylist_delete(xheader_data);
  
  /* Add the bad pixel mask to the corresponding cpl image */
  cpl_size size = cpl_imagelist_get_size(imlist_data);
  for(cpl_size k = 0; k < size; k++) {
     cpl_image* data = cpl_imagelist_get(imlist_data, k);
     cpl_image* bpm  = cpl_imagelist_get(imlist_bpm, k);
     cpl_mask*  mask = cpl_mask_threshold_image_create(bpm, 0, INT_MAX);
     cpl_image_reject_from_mask(data, mask);
     cpl_mask_delete(mask);
     cpl_imagelist_set(imlist_data, data, k);
  }

   /* Create the hdrl imagelist and free memory */
  hdrl_imagelist* hlist = NULL;
  hlist = hdrl_imagelist_create(imlist_data, imlist_error);
  hdrl_imagelist_delete(imlist_data);
  hdrl_imagelist_delete(imlist_error);
  hdrl_imagelist_delete(imlist_bpm);
  
  /* store the imagelist information into the cpl_table and free memory */
  cpl_table* restable  = hdrl_resample_imagelist_to_table(hlist, wcs);
  hdrl_imagelist_delete(hlist);
  
  /* Define the output grid */
  hdrl_parameter *outgrid = NULL;
  outgrid = hdrl_resample_parameter_create_outgrid2D(5.5e-05, 5.5e-05, 2.8e-10);

  /* define the method - here LANCZOS */
  hdrl_parameter *method = NULL;
  method = hdrl_resample_parameter_create_lanczos(1, CPL_FALSE, 2);
  
  /* Do the resampling */
  hdrl_resample_result *result = NULL;
  result = hdrl_resample_compute(restable, method, outgrid, wcs);

   /* save the relevant results */
...
   /* Cleanup the memory */
   hdrl_parameter_delete(method);
   hdrl_parameter_delete(outgrid);
   hdrl_resample_result_delete(result);
   cpl_wcs_delete(wcs);
   cpl_table_delete(restable);
\end{lstlisting}
}

\subsubsection{Additional Information}

As mentioned above, the \verb+hdrl_resample_compute+ function requires as an
input a cpl table and not hdrl images or an hdrl imagelist. This choice was made
for flexibility but comes with a cost on the memory consumption.  Moreover, the
current implementation does not distinguish between a 2D image and a 3D cube in
the table structure, i.e. the wavelength column is always filled. Furthermore,
cpl functions are usually used to read the input images and they are then
inserted into a hdrl image, which implies a small overhead.

We can understand the requirements in term of memory consumption as follows.
In the typical use case the developer has input three cubes: one containing the data, stored in a float\footnote{usually 4 bytes} data type, one the error, stored in a float data type, and the bad pixels, stored in an integer\footnote{usually 4 bytes} data type.
Then the \verb+cpl_imagelist+ objects are inserted into a hdrl imagelist
which has the data and error in double precision and each has a bpm in
unsigned~char\footnote{usually 1 bytes} precision [see step 2. below].
Then the information of the hdrl imagelist is stored into a cpl table [see
step 3. below].
Finally one has to define and fill the hdrl imagelist containing the
final resampled image/cube [see step 4. below].
The size of the resampled image
[step 4. below] strongly depends on the use-case: i.e. if the input images/cubes
are mostly stacked, or if a mosaic is created and, if a mosaic is created,
its size strongly depends on possible gaps between the CCDs.


\begin{enumerate}
\item cpl\_imagelist (read the raw data x, y, $\lambda$)
\begin{itemize}
\item data (float) + error (float) + bpm (int) \textbf{[requires $\approx 3 \times float$]}
\end{itemize}
\item $\Rightarrow$ hdrl\_imagelist (join the raw data/error/bpm)
\begin{itemize}
\item $\Rightarrow$ data ($2\times$ float) + error ($2\times$ float) + bpm
($2\times$ unsigned char $\approx 0.5 \times float$) \textbf{[requires $\approx 4.5 \times float$]}
\end{itemize}
\item $\Rightarrow$ cpl\_table (transform raw data x, y, $\lambda$ $\rightarrow$
Ra, Dec, $\lambda$)
\begin{itemize}
\item Ra ($2\times$ float) + Dec ($2\times$ float) + $\lambda$ ($2\times$ float) + data ($2\times$ float) + error ($2\times$ float) + bpm (int)  \textbf{[requires $\approx 11 \times float$]}
\end{itemize}
\item $\Rightarrow$ hdrl\_imagelist (resample Ra, Dec, $\lambda$  $\rightarrow$ x, y, $\lambda$)
\begin{itemize}
\item $\Rightarrow$ data ($2\times$ float) + error ($2\times$ float) + bpm
($2\times$ unsigned char $\approx 0.5 \times float$) \textbf{[requires $\approx 4.5 \times float$]}
\end{itemize}
\end{enumerate}

After reading the data [step 1.] and inserting the data into an hdrl imagelist
[step 2.] the data of step 1. can be deleted. The total image consumption of the
transition of step 2 and step 3 depends on the data. If you have multi extension
frames like for VIRCAM you can populate the table extension by extension and
therefore you are mostly dominated by the size of your table. If you can not
work extension by extension, e.g. if your input is a MUSE cube you can delete
the hdrl imagelist of step 2 only after you fully populate the cpl table of
step 3. This means that you have to keep the data involved in step 2 and
step 3 in memory at the same time\footnote{see section
\ref{sec:largedata:buffer} for a possibility to
address this problem}. Then, once the table is filled, you may delete the hdrl
imagelist created in step 2. As written above, the size of the resampled image [step 4.]
strongly depends on the use-case so here we assume that the resampled image is a
mosaic without gaps in between (no stacking). Then for transition between
steps 3 and 4 one needs to have the table and the resampled output image in
memory at the same time.

Usually, memory (RAM) requirements are dominated by the data allocated in step 2~+~3  or step 3~+~4. This means that as a rule of thumb one would
need \textbf{[$\approx 11 + 4.5 = 15.5 \times float \times \mbox{number of input pixels}$]} of memory.

This shows that if your data requires a certain amount of RAM to be stored as
raw data you need \textbf{at
least 6 times more memory}: $3\times$ float $\Rightarrow$ $15.5 \times$
float, to perform the corresponding resampling. Please note, that in case you
do not have error images/cubes and/or bpm
images/cubes in input you have to assume that they also would be there, when you
calculate the maximum amount of memory consumption as previously described.
Even if you pass NULL to the hdrl\_imagelist\_create() function, it will
then create an error and bpm image on the fly. In other words, if you only have
information on the input data (no error and no bpm), you need
\textbf{at least 18 times more memory} to do the resampling.


