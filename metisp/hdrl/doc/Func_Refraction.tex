\subsection{Differential Atmospheric Refraction (DAR)}
\label{refraction:main}

The effects of atmospheric refraction can be readily seen in any spectrograph or IFU. Atmospheric refraction will displace a source, or its spectrum, 
by an amount that is dependent on the source wavelength and the angular distance of the source from the zenith.  
This effect is due to the stratified density structure of our atmosphere, and the displacement will be toward the zenith and will be largest for shorter wavelengths.
Because of this latter attribute, differential atmospheric refraction is not generally associated with infrared instruments. However, it can be 
readily seen in SINFONI data cubes observed with the largest wavelength coverage (H+K) and in the smallest pixel scales (25 mas). 
Here, the shift can be as large as 6 pixels from the beginning of the data cube to its end (see Figures \ref{fig:SINFONI_trace} and \ref{fig:SINFONI_image}).

This module uses an analytical approach to compute the expected differential refraction as a function of wavelength, zenith angle, and the refractive index of air
which, in turn, depends on temperature, pressure, and water vapour pressure.

The {\tt hdrl\_dar} routines require the following inputs (all of which are, generally, available in the input data headers):
\begin{itemize}
\item the ambient atmospheric parameters: temperature, pressure, and humidity as contained in the environmental keyword headers:
TEL.AMBI.TEMP, TEL.AMBI.PRES.START/END, and TEL.AMBI.RHUM, respectively.
\item the instrument rotation angle on the sky
\item the parallactic angle of instrument
\item and, the world-coordinate system (WCS)
\end{itemize}

With this input, the differential atmospheric refraction is calculated (optionally, also with an error propagation), and provides an output of the $X$ and $Y$-axis shifts as a function of the
\verb+cpl_vector+ with the input wavelengths.  The resulting shift corrections can be directly apply to the pixel image in order to correct this effect.
The next section describes the algorithms used to make this calculation.


\subsubsection{Algorithm - short description}

This module contains routines to calculate the refractive index of air \footnote{See: \url{https://emtoolbox.nist.gov/Wavelength/Documentation.asp\#AppendixA}, for the formulae used.}. 
The main loop compute the differential refractive offset for the input reference wavelengths, and stores the shift in the coordinates, taking into account the instrument rotation angle 
on the sky and the parallactic angle at the time of observation.

The differential atmospheric refraction is calculated according to the algorithm from Filippenko (1982, PASP, 94, 715). This algorithm uses the Owens formula 
which converts relative humidity in water vapour pressure. 


\paragraph{Owens saturation pressure}

This function computes the saturation pressure using the J.C. Owens calibration (1967, Applied Optics, 6, 51-59). 
The saturation pressure is given by:

\begin{equation}
s_p = -10474 +116.43\ T -0.43284\ T^2 +0.00053840\ T^3
\label{eq:owens}
\end{equation}

where verb+T+ is the temperature in Kelvins.


\paragraph{Filippenko refractive index}

At sea level (P=760 \verb+mm Hg+ , T = 15 $^oC$) the refractive index of dry air is given by (Edl\'en 1953; Coleman, Bozman, and Meggers 1960):
\begin{equation}
(n( \lambda )_{15,760}-1)10^6 = 64.328 + \frac{29498.1}{146-(1/ \lambda )^2} +\frac{255.4}{41-(1/ \lambda )^2}
\end{equation}
where $\lambda$ is the wavelength of light in vacuum (microns). Since observatories are usually located at high altitudes, the index of refraction must be corrected for the 
lower ambient temperature and pressure (P) (Barrell 1951):
\begin{equation}
(n(\lambda)_{T,P} -1) = (n(\lambda)_{15,760} - 1) \cdot \frac{P[1+(1.049-0.0157\ T) 10^{-6}\ P]}{720.883 (1+0.003661\ T)}
\end{equation}

In addition, the presence of water vapour in the atmosphere reduces $(n-1)10^6$ by:
\begin{equation}
\frac{0.0624-0.000680/\lambda^2}{1 + 0.003661\ T} f
\end{equation}
where $f$ is the water vapour pressure in mm of Hg, and T is the air temperature in $^oC$ (Barrell 1951) and is expressed as:

\begin{equation}
f = 0.75006158 \cdot s_p \cdot h
\end{equation}
where $s_p$ is the saturation pressure of equation \ref{eq:owens} and $h$ is the relative humidity in \verb+[%]+.


\subsubsection{Function}

The differential atmospheric refraction is computed by the following function:
\begin{lstlisting}
cpl_error_code hdrl_dar_compute(
        const hdrl_parameter * params,
        const hdrl_value       lambdaRef, 
        const cpl_vector     * lambdaIn,
        cpl_vector           * xShift, 
        cpl_vector           * yShift, 
        cpl_vector           * xShiftErr, 
        cpl_vector           * yShiftErr)
\end{lstlisting}

The output products contain the resulting differential atmospheric refraction, and include the shift in pixels (with their associated error).
This results can be directly apply to the original image.
Each output value is associated at each wavelength value included in the \verb+lambdaIn+ [\AA] \verb+cpl_vector+.
The Filipenko method uses the value \verb+lambdaRef+ [\AA] as the reference wavelength from which the differential atmospheric
refraction is computed.  In other words, the \verb+xShift and yShift+ values will be zero at \verb+lambdaRef+.   Generally, \verb+lambdaRef+
is chosen to be the mid-point of the wavelength range covered by the input data.


\subsubsection{Inputs}
\label{sec:algorithms:refraction:inputs}
The input parameters to be passed to \verb+hdrl_dar_compute+ are created by executing the following function

\begin{lstlisting}
hdrl_parameter * hdrl_dar_parameter_create(
        hdrl_value             airmass,
        hdrl_value             parang,
        hdrl_value             posang,
        hdrl_value             temp,
        hdrl_value             rhum,
        hdrl_value             pres, 
        cpl_wcs              * wcs)
\end{lstlisting}


The input parameters of the function are:
\begin{itemize}
  \item \verb+airmass+: Air mass.
  \item \verb+parang+: Parallactic angle during exposure [deg].
  \item \verb+posang+: Position angle on the sky [deg].
  \item \verb+temp+: Temperature [$^o$C].
  \item \verb+rhum+: Relative humidity [\%].
  \item \verb+pres+: Pressure [mbar].
  \item \verb+wcs+: World coordinate system [deg].
\end{itemize}
Note that every parameter is an \verb+hdrl_value+, hence an error can be also provided and the routine supports error propagation.
See section \ref{sec:algorithms:refraction:parameters} for a summary of the error input parameters in terms of the esorex call to this recipe.

\textit{Please note that during the testing of the algorithm it was
  discovered that the propagated errors in the source shifts, computed
  when the error of the temperature (temp.error) and pressure
  (pres.error) are specified, are over-estimated.  We are currently
  investigating the source of this over-estimate and expect to be able
  to repair it soon.  The core functionality of the algorithm, in
  computing the expected source shifts due to atmospheric refraction,
  are not affected by this issue.}

\subsubsection{Outputs}
\label{sec:algorithms:refraction:outputs}

The function fills several input/output \verb+cpl_vector+ structures that contain the shift and error in each data axis.

This values are associated with each wavelength in the \verb+lambdaIn+ \verb+cpl_vector+ and they can be applied directly to the image.
The user is free to apply these values either as a continuous corrective shift to the data (requiring, of course, interpolation), or to round the
shifts to integer values and shift the data to a pixel grid.


\subsubsection{Example}
\label{sec:algorithms:refraction:example}
This section briefly describes the actual call to the function.

\begin{lstlisting}
/* Declare and init variables */
hdrl_value airmass, parang, posang, temp, rhum, pres;
cpl_wcs *wcs = cpl_wcs_new_from_propertylist(plPri);
...
/* Create hdrl_parameter from call the main function */
hdrl_parameter *h_par = hdrl_dar_parameter_create(
      airmass, parang, posang, temp, rhum, pres, wcs);
...
/* Declare and init the input (lambdaIn) and outputs cpl_vector */
cpl_vector *lambdaIn, *xShift, *yShift, *xShiftErr, *yShiftErr;
...
/* Execute main funtion and clean memory*/
hdrl_dar_compute(h_par, lambdaRef, lambdaIn,
      xShift, yShift, xShiftErr, yShiftErr);
hdrl_parameter_delete(h_par);
\end{lstlisting}


\subsubsection{DAR Parameters in a Recipe}
\label{sec:algorithms:refraction:parameters}

In the context of a recipe, all input parameters needed by the differential atmospheric refraction computation may be provided as input parameters. 
This will create a esorex man-page as follows:
{\footnotesize
  \begin{verbatim}
  $ esorex --man-page hdrldemo_dar
  
  --maxRatioNaNsInImage : Max ratio [%] between bad pixels (NaNs) and good pixels. [30.0]
  --lambdaRef-err       : Error [%] of the reference lambda. [0.0]
  --parang-err          : Error [%] of the parallactic angle. [0.0]
  --posang-err          : Error [%] of the position angle. [0.0]
  --temp-err            : Error [%] of the temperature. [0.0]
  --rhum-err            : Error [%] of the relative humidity. [0.0]
  --pres-err            : Error [%] of the pressure. [0.0]
  --airm.method         : Method of approximation to calculate airmass. 
                                <HARDIE_62 | YOUNG_IRVINE_67 | YOUNG_94> [HARDIE_62]
  --airm.ra-err         : Error [%] of the right ascension (using for calculate airmass). [0.0]
  --airm.dec-err        : Error [%] of the declination (using for calculate airmass). [0.0]
  --airm.lst-err        : Error [%] of the local sideral time (using for calculate airmass). [0.0]
  --airm.exptime-err    : Error [%] of the integration time (using for calculate airmass). [0.0]
  --airm.geolat-err     : Error [%] of the latitude (using for calculate airmass). [0.0]
  \end{verbatim}
}
where the parameters listed include:
\begin{itemize}
    \item  \verb+--maxRatioNaNsInImage+ allows the selection of valid images.  An image having a ratio of NaNs greater than that specified by \verb+--maxRatioNaNsInImage+ will be excluded
    from the analysis.
    \item \verb+--lambdaRef-err+ to \verb+--pres-err+ to set errors to the input parameters for DAR, that allow for the computation of error propagation.
    \item \verb+--airm.method+ for the selection of the method to used to approximate the airmass (see section \ref{airmass:main} for a more detailed description).
    \item \verb+--airm.ra-err+ to \verb+--airm.geolat-err+ to set the errors in the airmass calculation, that allow for the computation of error propagation (see section \ref{airmass:main} for a more detailed description).
\end{itemize}
