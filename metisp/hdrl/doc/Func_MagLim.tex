
\subsection{Limiting Magnitude}
\label{sec:algorithms:maglim:main}

The limiting magnitude is one of the mandatory header keywords in the
Phase 3 archival
standard\footnote{see \url{https://www.eso.org/sci/observing/phase3.html}
for the ESO Phase 3 Standard} and characterizes the depth of an
observation. According to the ESO Phase 3 standard, it is defined as
the magnitude of a unresolved source whose flux is 5 times the noise
background, i.e. the magnitude of a point like source detected with
$\frac{S}{N}=5$.

\subsubsection{Algorithm - short description}
\label{sec:algorithms:maglim:algo}

The limiting magnitude of an image is determined as follows:

\begin{itemize}
\item The input image \verb+image+ is convolved with a Gaussian kernel
  specified by the input parameters \verb+fwhm+ (the FWHM of the
  kernel), \verb+kernel_size_x+, and \verb+kernel_size_y+ (the
  extension of the kernel in x-direction and y-direction in pixel).
  To minimize border effects/artifacts during convolution, the input
  image gets enlarged by a border depending on the kernel size. The
  parameter \verb+image_extend_method+ can be used to specify the
  algorithm to compute the values of the out-of-bounds pixels during
  convolution (\verb+HDRL_IMAGE_EXTEND_NEAREST+ $\rightarrow$ use the
  nearest border pixel, \verb+HDRL_IMAGE_EXTEND_MIRROR+ $\rightarrow$
  mirror the image with respect to the border). Masked pixels are
  ignored in the convolution, i.e. the kernel is normalized ignoring
  masked pixels.
\item Compute the \verb+mode+ as described in
  section~\ref{sec:algorithms:robust_mean:mode} of the convolved
  image.
\item   Consider all the non-masked pixels in the convolved image below the MODE
(let's define them as valid pixels), and compute the noise as:
$$NOISE = 1.4826\cdot\mbox{MAD(valid pixels)}\cdot c$$

  The factor $c = \frac{1}{\sqrt{1-\frac{2}{\pi}}}\approx 1.6588967 $
  accounts for the fact that HDRL computes the statistics on half of the distribution
  (assuming a Gaussian noise distribution), MAD is the median absolute deviation,  and
  1.4826  is  the  scaling  factor to convert the MAD in the standard deviation (for a Gaussian
  distribution).  If MAD=0, then:
  
$$NOISE = \mbox{STDDEV(valid pixels)}\cdot c$$
where STDEV indicates the standard deviation.

\item Compute the limiting magnitude as:
$$ABMAGLIM = -2.5 log_{10}(5\cdot NOISE\cdot 4\pi\sigma^{2})+ZPT$$
with $\sigma = \frac{FWHM}{\sqrt{4 ln(4)}}\approx \frac{FWHM}{2.35482}$
\end{itemize}

\subsubsection{\label{sec:algorithms:maglim:functions}Functions}

The \HDRL implements the following function to derive the limiting magnitude:

\begin{lstlisting}
cpl_error_code hdrl_maglim_compute(
        cpl_image                * image,
	double                     zeropoint,
	double                     fwhm,
	cpl_size                   kernel_size_x,
        cpl_size                   kernel_size_y,
	hdrl_image_extend_method   image_extend_method,
	hdrl_parameter           * mode_parameter,
	double                   * limiting_magnitude);
\end{lstlisting}


\subsubsection{\label{sec:algorithms:maglim:inputs}Inputs}

The image where the limiting magnitude should be derived and its
zeropoint is passed to the function by the \verb+image+ and the
\verb+zeropoint+ argument.

The \verb+fwhm+ parameter sets the full width at half maximum of the
Gaussian convolution kernel whereas the extension of the kernel in
x-direction and y-direction (in pixel) is set by \verb+kernel_size_x+,
and \verb+kernel_size_y+, respectively.  The parameter
\verb+image_extend_method+ specifies the algorithm to compute the
values of the out-of-bounds pixels during convolution
(\verb+HDRL_IMAGE_EXTEND_NEAREST+ $\rightarrow$ use the nearest border
pixel, \verb+HDRL_IMAGE_EXTEND_MIRROR+ $\rightarrow$ mirror the image
with respect to the border).

The hdrl parameter \verb+mode_parameter+ controls the computation of
the mode of the distribution and is created by the
\verb+hdrl_collapse_mode_parameter_create()+ function. The parameter
of this function (\verb+histo_min+, \verb+histo_max+, \verb+bin_size+,
\verb+mode_method+, and \verb+error_niter+) are described in
section~\ref{sec:algorithms:robust_mean:mode}

Finally, the computed limiting magnitude is returned as
\verb+limiting_magnitude+
  
\subsubsection{\label{sec:algorithms:maglim:outputs}Outputs}

The result of the function is the limiting magnitude of the image
stored in the function parameter\\ \verb+limiting_magnitude+


\subsubsection{Example}
The following code is a example how to use HDRL to compute the limiting
magnitude from an image.

\begin{lstlisting}
 
 /* initialize the parameters of a the mode computation */

 double histo_min = 10.;
 double histo_max = 1.; /* min > max -> autoset by algorithm */
 double bin_size = 0.;  /* bin_size <= 0 -> autoset by algorithm */
 cpl_size error_niter = 0; /* analytic error */
 hdrl_mode_type mode_method = HDRL_MODE_MEDIAN; /* very robust */

 hdrl_parameter * mode_parameter = NULL;
 mode_parameter = hdrl_collapse_mode_parameter_create(histo_min,
                                                      histo_max,
                                                      bin_size,
                                                      mode_method,
                                                      error_niter);
                                                      
 /* initialize other parameters of the limiting magnitude computation */

 double zeropoint = 24.3;                         
 double fwhm = 3.50;
 cpl_size kernel_size_x = 10; /* about 3 times fwhm */
 cpl_size kernel_size_y = 10; /* about 3 times fwhm */
 hdrl_image_extend_method convolution_boundary = HDRL_IMAGE_EXTEND_MIRROR;
 double limiting_magnitude = 0.;
 
 /* compute the limiting magnitude */
 
 hdrl_maglim_compute(image, zeropoint, fwhm, kernel_size_x,
       			  kernel_size_y, convolution_boundary,
       			  mode_parameter, &limiting_magnitude);
                                 
\end{lstlisting}

