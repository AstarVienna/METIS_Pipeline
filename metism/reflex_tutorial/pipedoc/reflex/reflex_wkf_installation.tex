\section{Software Installation}\label{Sec:Software_Installation}

{\tt Esoreflex} and the workflows can be installed in different ways:
via package repositories, via the\linebreak {\tt install\_esoreflex}
script or manually installing the software tar files.

 The recommended way is to use the package repositories if your
 operating system is supported. The {\tt macports} repositories
 support macOS 10.14 to 11, while the {\tt rpm/yum} repositories
 support Fedora 28 to 32, CentOS 7, Scientific Linux 7. For any other
 operating system it is recommended to use the {\tt
   install\_esoreflex} script.  

 The installation from package repository requires administrative privileges
 (typically granted via sudo), as it installs files in system-wide directories 
 under the control of the package manager. If you want a local
 installation, or you do not have sudo privileges, or if you want to
 manage different installations on different directories, then use the
 {\tt install\_esoreflex} script. Note that the script installation
 requires that your system fulfill several software prerequisites,
 which might also need sudo privileges.
 
 Reflex 2.11.x needs java JDK 11 to be installed.
 
 Please note that in case of major or minor (affecting the first two
 digit numbers) Reflex upgrades, the user should erase the
 \verb+$HOME/KeplerData+, \verb+$HOME/.kepler+ directories if present,
 to prevent possible aborts (i.e. a hard crash) of the {\tt esoreflex}
 process.

 
\subsection{Installing Reflex workflows via \texorpdfstring{{\tt macports}}{macports} }

 This method is supported for the macOS operating system. It is assumed that
 macports\linebreak (\http{www.macports.org}) is installed.
 Please read the full documentation at \newline
 \http{www.eso.org/sci/software/pipelines/installation/macports.html}.

\subsection{Installing Reflex workflows via \texorpdfstring{{\tt rpm/yum/dnf}}{rpm/yum/dnf} }

 This method is supported for Fedora 28 to 32, CentOS 7,
 Scientific Linux 7 operating systems, and requires sudo rights. To install, please follow these steps

\begin{enumerate}

\item Configure the ESO repository (This step is only necessary if the ESO repository has not already been previously configured).
  \begin{itemize}
  \item If you are running Fedora, run the following commands:
\begin{verbatim}
sudo dnf install dnf-plugins-core
sudo dnf config-manager --add-repo=ftp://ftp.eso.org/pub/dfs/
            pipelines/repositories/stable/fedora/esorepo.repo
\end{verbatim}
\item If you are running CentOS 7, run the following commands:
\begin{verbatim}
sudo yum install yum-utils ca-certificates yum-conf-repos
sudo yum install epel-release
sudo yum-config-manager --add-repo=ftp://ftp.eso.org/pub/dfs/
           pipelines/repositories/stable/centos/esorepo.repo
\end{verbatim}
\item If you are running SL 7, run the following commands:
\begin{verbatim}
sudo yum install yum-utils ca-certificates yum-conf-repos
sudo yum install yum-conf-epel
sudo yum-config-manager --add-repo=ftp://ftp.eso.org/pub/dfs/
    pipelines/repositories/stable/sl/esorepo.repo
\end{verbatim}
  \end{itemize}

\item Install the pipelines
\begin{itemize}
\item The list of available top level packages for different instruments is given by:
\begin{verbatim}
sudo dnf list esopipe-\*-all # (Fedora)
sudo yum list esopipe-\*-all # (CentOS 7, SL 7)
\end{verbatim}
 
\item To install an individual pipeline use the following (This example is for X-Shooter. Adjust the package name to the instrument you require.):
\begin{verbatim}
sudo dnf install esopipe-xshoo-all # (Fedora)
sudo yum install esopipe-xshoo-all # (CentOS 7, SL 7)
\end{verbatim}

\item To install all pipelines use:
\begin{verbatim}
sudo dnf install esopipe-\*-all # (Fedora)
sudo yum install esopipe-\*-all # (CentOS 7, SL 7)
\end{verbatim}


\end{itemize}

\end{enumerate}


 For further information, please read the full documentation at \newline
 \http{www.eso.org/sci/software/pipelines/installation/rpm.html}.
 


\subsection{Installing Reflex workflows via \texorpdfstring{{\tt install\_esoreflex}}{install\_esoreflex}}

This method is recommended for operating systems other than what
indicated above, or if the user has no sudo rights. Software
dependencies are not fulfilled by the installation script, therefore
the user has to install all the prerequisites before running the
installation script.

The software pre-requisites for {\tt Reflex \reflexvers} may be found at:\newline
  \http{www.eso.org/sci/software/pipelines/reflex\_workflows}

To install the {\tt Reflex \reflexvers} software and demo data, 
please follow these instructions:
\begin{enumerate}
  \item From any directory, download the installation script:
        {\small
        \begin{verbatim}
        wget https://ftp.eso.org/pub/dfs/reflex/install_esoreflex
        \end{verbatim}
        }

  \item Make the installation script executable:
        {\small
        \begin{verbatim}
        chmod u+x install_esoreflex
        \end{verbatim}
        }

  \item Execute the installation script:
        {\small
        \begin{verbatim}
        ./install_esoreflex
        \end{verbatim}
        }
        and the script will ask you to specify three directories: the download
        directory {\tt \verb|<|download\_dir\verb|>|}, the software
        installation directory {\tt \verb|<|install\_dir\verb|>|}, 
        and the directory to be used to store the demo data 
        {\tt \verb|<|data\_dir\verb|>|}.
        If you do not specify these directories, then the installation script 
        will create them in the current directory with default names.
        
      \item Follow all the script instructions; you will be asked
        whether to use your Internet connection (recommended: yes),
        the pipelines and demo-datasets to install (note that the
        installation will remove all previously installed pipelines
        that are found in the same installation directory).
 % \item You will be asked whether you want to use your Internet connection.
 %       Unless you want to reuse already downloaded packages (only advanced
 %       users), use the default Yes.

%  \item You will be given a choice of pipelines (with the corresponding 
 %       workflows) to install. Please specify the numbers for the pipelines 
 %       you require, separated by a space, or type ``A'' for all pipelines.

 % \item For the pipelines to be installed you will be prompted for the 
 %       demo data sets to be installed. Type ``A'' for all demo datasets.
 %       Take into account that if you are installing in a directory that 
 %       already contains data, it won't be removed.

 % \item The script will also detect whether previous versions of the 
 %       workflows or Reflex were installed and in this case you have the
 %       option to update links or remove obsolete cache directories. It is 
 %       advised to use the defaults.

 % \item If some of the prerequisite binaries for {\tt Reflex} are not under one
 %       of the paths indicated by the command,
 %       {\small
 %       \begin{verbatim}
 %       getconf PATH
 %       \end{verbatim}
 %       }
 %       then you will need to add the appropriate paths as a colon separated
 %       list to the {\tt esoreflex.path} parameter in the configuration file
 %       {\tt \verb|<|install\_dir\verb|>|/etc/esoreflex.rc}. This will usually
 %       be necessary when the FITS viewer ({\tt fv}) is installed outside of
 %       {\tt /usr/bin}. As an example, assume {\tt fv} is installed into the
 %       directory {\tt /usr/local/fv5.4}, the file {\tt esoreflex.rc} should
 %       then have the line setting {\tt esoreflex.path} look similar to the
 %       following:
 %       {\small
 %       \begin{verbatim}
 %       esoreflex.path=/usr/local/fv5.4
 %       \end{verbatim}
 %       }
 %       In the case of OS~X {\tt /Applications/fv.app/Contents/MacOS/} is the
 %       typically installation directory. Thus, this should be similar to the
 %       following line instead:
 %       {\small
 %       \begin{verbatim}
 %       esoreflex.path=/opt/local/bin:/Applications/fv.app/Contents/MacOS
 %       \end{verbatim}
 %       }
%%
  \item To start {\tt Reflex}, issue the command:
        {\small
        \begin{verbatim}
        <install_dir>/bin/esoreflex
        \end{verbatim}
        }
        It may also be desirable to set up an alias command for starting the 
        {\tt Reflex} software, using the shell command {\tt alias}. 
        Alternatively, the {\tt PATH} variable can be updated to contain the
        {\tt \verb|<|install\_dir\verb|>|/bin} directory.
\end{enumerate}

