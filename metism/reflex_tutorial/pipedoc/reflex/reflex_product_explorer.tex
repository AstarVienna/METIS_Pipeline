\subsubsection{The ProductExplorer}
\label{sec:product_explorer}

The ProductExplorer is an interactive component in the \reflex\
workflow whose main purpose is to list the final products with the
associated reduction tree for each dataset and for each reduction
attempt (see Fig.~\ref{fig:reflex_provenance_explorer}).

\medskip

{\bf {\it Configuring the ProductExplorer}}

You can configure the ProductExplorer GUI to appear after or before
the data reduction. In the latter case you can inspect products as
reduction goes on.


1. To display the ProductExplorer GUI at the end of the datareduction:

\begin{itemize}

\item Click on the global parameter ``ProductExplorerMode'' before
  starting the data reduction. A configuration window will appear
  allowing you to set the execution mode of the Product
  Explorer. Valid options are:
  \begin{itemize}
  \item "Triggered" (default). This option opens the ProductExplorer
    GUI when all the selected datasets have been reduced.
  \item "Enabled". This option opens the ProductExplorer GUI at the
    end of the reduction of each individual dataset.
  \item ``Disable''. This option does not display the ProductExplorer GUI.
\end{itemize}
 \item Press the \includegraphics[width=.5cm]{reflex_run_button.png} button to start the workflow.
\end{itemize}

2. To display the ProductExplorer GUI ``before'' starting the data
reduction:

\begin{itemize}
 
\item double click on the composite Actor "Inspect previously reduced
  data". A configuration window will appear. Set to "Yes" the field
  "Inspect previously reduced data (Yes/No)". Modify the field
  "Continue reduction after having inspected the previously reduced
  data? (Continue/Stop/Ask)". "Continue" will continue the workflow
  and trigger the DataOrganizer. "Stop" will stop the workflow;
  "Ask" will prompt another window deferring the decision whether
  continuing or not the reduction after having closed the Product
  Explorer.

\item Press the \includegraphics[width=.5cm]{reflex_run_button.png} button to start the workflow. Now the ProductExplorer
  GUI will appear before starting the data organization and reduction.

\end{itemize}

\bigskip

{\bf {\it Exploring the data reduction products}}


The left window of the ProductExplorer GUI shows the executions for
all the datasets (see Fig.~\ref{fig:reflex_provenance_explorer}). Once
you click on a dataset, you get the list of reduction attemps. Green
and red flags identify successfull or unsucessfull reductions. Each
reduction is linked to the ``Description'' tag assigned in the
``Select Dataset'' window.

\medskip

1. To identify the desired reduction run via the ``Description'' tag, proceed as follows:

\begin{itemize}
 \item Click on the symbol at the left of the dataset name. The full list
   of reduction attempts for that dataset will be listed. The column
   Exec indicates if the reduction was succesful (green flag: "OK") or
   not (red flag: "Failed").  

 \item Click on the entries in the field "Description" to visualize
   the description you have entered associated to that dataset on the
   Select Dataset window when reducing the data.

 \item  Identify the desired reduction run. All the products are listed in
   the central window, and they are organized following the data
   reduction cascade.
\end{itemize}

You can narrow down the range of datasets to search by configuring the
field "Show" at the top-left side of the ProductExplorer (options are:
"All", "Successful", "Unsuccessful"), and specifying the time range
(Last, all, From-to).

\medskip

2. To inspect the desired file, proceed as follows:

\begin{itemize}

\item Navigate through the data reduction cascade in the
  ProductExplorer by clicking on the files.

 \item Select the file to be inspected and click with the mouse right-hand button. 
   The available options are:
   \begin{itemize}
    \item Options available always:
      \begin{itemize}
        \item Copy full path. It copies the full name of the file onto the
          clipboard. Shift+Ctr+v to past it into a terminal.

        \item Inspect Generic. It opens the file with the fits viewer
          selected in the main workflow canvas.

        \item Inspect with. It opens the file with an executable that can
          be specified (you have to provide the full path to the
          executable).
      \end{itemize}

   \item  Options available for files in the {\tt TMP\_PRODUCTS\_DIR} directory only:
      \begin{itemize}
        \item  command line. Copy of the environment configuration and recipe
        call used to generate that file.
        \item Xterm. It opens an Xterm at the directory containing the file.
     \end{itemize}
   \item  Options available for products associated to interactive windows only:
      \begin{itemize}
       \item Display pipeline results. It opens the interactive windows associated
        to the recipe call that generated the file. Note that this is for
        visualization purposes only; the recipe parameters cannot be
        changed and the recipe cannot be re-run from this window.

      \end{itemize}
      \end{itemize}
      \end{itemize}
