%\subsubsection{Step1: Data Organisation And
%  Selection}\label{sec:data_organizer}
\subsubsection{Data Organisation And
  Selection}\label{sec:data_organizer}


The {\tt DataOrganiser} (DO) is the first crucial component of a
Reflex workflow. The DO takes as input {\tt RAW\_DATA\_DIR} and {\tt
  CALIB\_DATA\_DIR} and it detects, classifies, and organises the
files in these directories and any subdirectories. The output of the
DO is a list of ``DataSets''. A DataSet is a special Set of Files
(SoF). A DataSet contains one or several science (or calibration)
files that should be processed together, and all files needed to
process these data. This includes any calibration files, and in turn
files that are needed to process these calibrations. Note that
different DataSets might overlap, i.e.  some files might be included
in more than one DataSet (e.g., common calibration files).

A DataSet lists three different pieces of information for each of its
files, namely 1) the file name (including the path), 2) the file
category, and 3) a string that is called the ``purpose'' of the file.
The DO uses the OCA\footnote{OCA stands for
  OrganisationClassificationAssociation and refers to rules, which
  allow to classify the raw data according to the contents of the
  header keywords, organise them in appropriate groups for processing,
  and associate the required calibration data for processing. They can
  be found in the directory $<${\tt install\_dir}$>${\tt
    /share/esopipes/}$<${\tt pipeline-version}$>${\tt /reflex/},
  carrying the extension{\tt .oca}.
The variable  $<${\tt install\_dir}$>$ depends on the operative system 
and installation procedure. For installation through {\tt rpm}:
  $<${\tt install\_dir}$>${\tt =/usr}; for installation through macport
  $<${\tt install\_dir}$>${\tt =/opt/local}; for installation through 
the installation script install\_esoreflex it depends on the path 
specified during installation, e.g.
$<${\tt install\_dir}$>${\tt =}$<${\tt specified\_path}$>${\tt /install}} 
rules to find the files to include
in a DataSet, as well as their categories and purposes.  The file
category identifies different types of files, and it
  is derived by information in the header of the file itself. A
  category could for example be {\tt RAW\_CALIBRATION\_1}, {\tt
    RAW\_CALIBRATION\_2} or {\tt RAW\_SCIENCE}, depending on the
  instrument.   The purpose of a file identifies the reason why a
file is included in a DataSet. The syntax is {\tt
  action\_1/action\_2/action\_3/ ... /action\_n}, where each {\tt
  action\_i} describes an intended processing step for this file (for
example, creation of a {\tt MASTER\_CALIBRATION\_1} or a {\tt
  MASTER\_CALIBRATION\_2}). 
The actions are defined in the OCA rules
and contain the recipe together with all file categories required to
execute it (and predicted products in case of calibration data).  For
example, a workflow might include two actions {\tt
  action\_1} and {\tt action\_2}. The former
creates {\tt MASTER\_ CALIBRATION\_1} from {\tt RAW\_CALIBRATION\_1},
and the later creates a {\tt MASTER\_CALIBRATION\_2} from {\tt
  RAW\_CALIBRATION\_2}. The {\tt action\_2} action needs
{\tt RAW\_CALIBRATION\_2} frames and the {\tt
  MASTER\_ CALIBRATION\_1} as input. In this case, these {\tt
  RAW\_CALIBRATION\_1} files will have the purpose {\tt
  action\_ 1/action\_2}. The same
  DataSet might also include {\tt RAW\_CALIBRATION\_1} with a different
  purpose; irrespective of their purpose the file category for all
  these biases will be {\tt RAW\_CALIBRATION\_1}.


The Datasets created via the {\tt DataOrganiser} will be displayed in
the {\tt DataSet Chooser}. Here the users have the possibility to inspect
the various datasets and decide which one to reduce. By default,
DataSets that have not been reduced before are highlighted for
reduction. Click either \fbox{\tt Continue} in order to continue with
the workflow reduction, or \fbox{\tt Stop} in order to stop the
workflow. A full description of the {\tt DataSet Chooser} is presented
in Section \ref{sec:dataset_chooser}.


Once the \fbox{\tt Continue} is pressed, the workflow starts to reduce
the first selected DataSet. Files are broadcasted according to their
purpose to the relevant actors for processing.

%\underline{Routing of files}\\

The categories and purposes of raw files are set by the DO, whereas
the categories and purpose of products generated by recipes are set by
the {\tt RecipeExecuter}.  The file
categories are used by the {\tt FitsRouter} to send files to particular
processing steps or branches of the workflow (see below). The purpose
is used by the {\tt SofSplitter} and {\tt SofAccumulator} to generate
input SoFs for the {\tt RecipeExecuter}.  
%Note that while the DO
%includes files into a DataSet for a reason, and records this reason as
%the ``purpose'' of the file, the workflow itself can use these files
%in a different manner. 
The {\tt SofSplitter} and {\tt SofAccumulator}
accept several SoFs as simultaneous input. The {\tt SofAccumulator}
creates a single output SoF from the inputs, whereas the {\tt
  SofSplitter} creates a separate output SoF for each purpose.
%Both actors have in common that
%files with a given purpose are only included in the output SoFs if
%files with such a purpose are included {\em each of the input SoFs}.


