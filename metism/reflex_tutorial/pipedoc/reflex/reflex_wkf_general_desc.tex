\section{The \instname \ Workflow \label{sec:wkf_general_desc}}

The \instname\,  workflow canvas is organised into a number of areas.
From top-left to top-right you
will find general workflow instructions, directory parameters, and
global parameters.  In the middle row you will find five boxes
describing the workflow general processing steps in order from left to
right, and below this the workflow actors themselves are organised
following the workflow general steps. 

\subsection{Workflow Canvas Parameters \label{sec:wkf_canvpar}}

The workflow canvas displays a number of parameters that may be set by
the user. Under ``Setup
Directories'' the user is only required to set the {\tt
  RAW\_DATA\_DIR} to the working directory for the dataset(s) to be
reduced, which, by default, is set to the directory containing the
demo data. The {\tt RAW\_DATA\_DIR}
is recursively scanned by the {\tt Data Organiser} actor for input raw
data. The directory {\tt CALIB\_DATA\_DIR}, which is by default within the
pipeline installation directory, is also scanned by the {\tt Data
  Organiser} actor to find any static calibrations that may be missing
in your dataset(s).  If required, the user may edit the directories
{\tt BOOKKEEPING\_DIR}, {\tt LOGS\_DIR}, {\tt TMP\_PRODUCTS\_DIR}, and
{\tt END\_PRODUCTS\_DIR}, which correspond to the directories where
book-keeping files, logs, temporary products and end products are
stored, respectively (see the Reflex User Manual for further details;
\cite{REFLEXMAN}).

There is a mode of the {\tt Data Organiser} that skips the built-in
data organisation and uses instead the data organisation provided by
the CalSelector tool. To use this mode, click on {\tt Use CalSelector
  associations} in the {\tt Data Organiser} properties and make sure
that the input data directory contains the XML file downloaded with
the CalSelector archive request (note that this does not work for all
instrument workflows).

Under the ``Global Parameters'' area of the workflow canvas, the user
may set the {\tt FITS\_VIEWER} parameter to the command used for
running his/her favourite application for inspecting FITS
files. Currently this is set by default to {\tt fv}, but other
applications, such as {\tt ds9}, {\tt skycat} and {\tt gaia} for
example, may be useful for inspecting image data. Note that
it is recommended to specify the full path to the visualization
application (an alias will not work).

By default the {\tt EraseDirs} parameter is set to {\tt false}, which
means that no directories are cleaned before executing the workflow,
and the recipe actors will work in Lazy Mode (see
Section~\ref{sec:lazy_mode}), reusing the previous pipeline recipe outputs
if input files and parameters are the same as for the previous
execution, which saves considerable processing time. Sometimes it is
desirable to set the {\tt EraseDirs} parameter to {\tt true}, which
forces the workflow to recursively delete the contents of the
directories specified by {\tt BOOKKEEPING\_DIR}, {\tt LOGS\_DIR}, and
{\tt TMP\_PRODUCTS\_DIR}. This is useful for keeping disk space usage
to a minimum and will force the workflow to fully re-reduce the data
each time the workflow is run.

The parameter {\tt RecipeFailureMode} controls the behaviour in case that
a recipe fails. If set to {\tt Continue}, the workflow will trigger the next
recipes as usual, but without the output of the failing recipe, which in most
of the cases will lead to further failures of other recipes without the user
actually being aware of it. This mode might be useful for unattended processing
of large number of datasets. If set to {\tt Ask}, a pop-up window will ask
whether the workflow should stop or continue. This is the default. 
Alternatively, the {\tt Stop} mode will stop the workflow execution immediately.


\ifx\reflexinteract\undefined
\else
The parameter {\tt GlobalPlotInteractivity} controls whether the interactive
windows will appear for those windows which are {\sl enabled} by default.
The possible values are {\tt true, false}.
Take into account that some windows are disabled in the default configuration
and therefore are not affected by this parameter.
\fi

The parameter {\tt ProductExplorerMode} controls whether the {\tt
  ProductExplorer} actor will show its window or not.  The possible
values are {\tt Enabled},   {\tt Triggered}, and {\tt Disabled}.
{\tt Enabled} opens the ProductExplorer GUI at the end of the reduction of
each individual dataset. {\tt Triggered} (default and recommended) opens
the ProductExplorer GUI when all the selected datasets have been
reduced. {\tt Disabled} does not display the ProductExplorer GUI.

