\section{Reducing your own data}\label{sec:owndataset} 

In this section we describe how to reduce your own data set.

First, we suggest the reader to familiarize with the workflow by
reducing the demo dataset first (Section \ref{sec:quick_start}), but
it is not a requirement.


\subsection{The {\tt esoreflex} command}\label{sec:esoreflex_command}

We list here some options associated to the \reflex\ command. We
recommend to try them to familiarize with the system.  In the
following, we assume the \reflex\ executable is in your path; if not
you have to provide the full path {\tt <install\_dir>/bin/esoreflex}

To see the available options of the \reflex\ command type:
\begin{verbatim}
esoreflex -h
\end{verbatim}
The output is the following.
\begin{verbatim}
-h | -help             print this help message and exit.
-v | -version          show installed Reflex version and pipelines and exit.
-l | -list-workflows   list available installed workflows and from 
                       ~/KeplerData/workflows.
-n | -non-interactive  enable non-interactive features.
-e | -explore          run only the Product Explorer in this workflow
-p <workflow> | -list-parameters <workflow>
                       lists the available parameters for the given 
                       workflow.
-config <file>         allows to specify a custom esoreflex.rc configuration 
                       file.
-create-config <file>  if <file> is TRUE then a new configuration file is 
                       created in ~/.esoreflex/esoreflex.rc. Alternatively 
                       a configuration file name can be given to write to. 
                       Any existing file is backed up to a file with a '.bak' 
                       extension, or '.bakN' where N is an integer.
-debug                 prints the environment and actual Reflex launch 
                       command used.
\end{verbatim}



\subsection{Launching the workflow}\label{sec:setup}

We list here the recommended way to reduce your own datasets. Steps 1
and 2 are optional and one can start from step 3.

\begin{enumerate}

\item Type: {\tt esoreflex -n <parameters>} \wkfname\ to launch the
  workflow non interactively and reduce all the datasets with default
  parameters.

  {\tt <parameters>} allows you to specify the workflow parameters,
  such as the location of your raw data and the final destination of
  the products.

  For example, type (in a single command line):

{\tt esoreflex -n \\
\indent ~ -RAW\_DATA\_DIR /home/user/my\_raw\_data \\
\indent ~ -ROOT\_DATA\_DIR /home/user/my\_reduction \\
\indent ~ -END\_PRODUCTS\_DIR \$ROOT\_DATA\_DIR/reflex\_end\_products \\
\indent ~ }\wkfn\

to reduce the complete datasets that are present in the directory {\tt
  /home/user/my\_raw\_data} and that were not reduced before. Final
products will be saved in  {\tt
  /home/user/my\_reduction/reflex\_ end\_products}, while book keeping,
temporary products, and logs will be saved in sub-directories of {\tt
  /home/user/my\_reduction/}. If the reduction of a dataset fails, the
reduction continues to the next dataset. It can take some time,
depending on the number of datasets present in the input directory.
For a full list of workflow parameters type {\tt esoreflex -p}
\wkfname. Note that this command lists only the parameters, but does not
launch the workflow.

Once the reduction is completed, one can proceed with optimizing the
results with the next steps.


\item Type:

{\tt esoreflex -e} \wkfn\

to launch the Product Explorer. The Product Explorer allows you to
inspect the data products already reduced by the \wkfname\ \reflex\
workflow. Only products associated with the workflow default
bookkeeping database are shown. To visualize products associated to
given bookkeeping database, pass the full path via the {\tt
  BOOKKEEPING\_DB} parameter:


{\tt esoreflex -e BOOKKEEPING\_DB <database\_path>} \wkfn\

to point the product explorer to a given {\tt <database\_path>}, e.g., {\tt /home/username/reflex/ reflex\_bookkeeping/test.db}

The Product Explorer allows you to inspect the products while the
reduction is running. Press the button \fbox{\tt Refresh} to update
the content of the Product Explorer.  This step can be launched in parallel to step 1.

A full description of the Product Explorer will be given in Section \ref{sec:product_explorer}

\item
Type: 

{\tt esoreflex } \wkfn\ {\tt \&}

to launch the \wkfname\ \reflex\ workflow.  The \wkfname\ workflow
window will appear (Fig.~\ref{fig:pipe_wkf_layout}). Please configure the
set-up directories {\tt ROOT\_DATA\_DIR}, {\tt RAW\_DATA\_DIR}, and
other workflow parameters as needed. Just double-click on them, edit
the content, and press \fbox{\tt OK}.  Remember to specify the same
{\tt <database\_path>} as for the Product Explorer, if it has been
opened at step \#2, to synchronize the two processes.

\item (Recommended, but not mandatory) On the main \reflex\ menu set
  {\tt Tools} --> {\tt Animate at Runtime} to 1 in order to highlight
  in red active actors during execution.

\item Press the button
  \includegraphics[width=0.5cm,height=0.5cm]{reflex_run_button.png} to
  start the workflow. First, the workflow will highlight and execute
  the {\tt Initialise} actor, which among other things will clear any
  previous reductions if required by the user (see
  Section~\ref{sec:wkf_canvpar}). Secondly, if set, the workflow will
  open the Product Explorer, allowing the user to inspect previously
  reduced datasets (see Section \ref{sec:product_explorer} for how to
  configure this option).
\end{enumerate}
 