\section{Frequently Asked Questions}

\begin{itemize}
   \item {\bf The error window fills the whole screen - how can I get
       to the \fbox{\tt Continue}/\fbox{\tt Stop} buttons?}

Press the \fbox{Alt} key together with your left mouse button to move
the window upwards and to the left. At the bottom the \fbox{\tt
  Continue}/\fbox{\tt Stop} buttons will be visible. This bug is known
but could not yet be fixed.

 \item {\bf I tried to {\tt Open} (or {\tt Configure}) an {\tt Actor} while the workflow is running and now it does not react any more. What should I do?}

   This is a limitation of the underlying Kepler engine. The only way
   out is to kill the workflow externally. If you want to change
   anything while a workflow is running you first need to pause it.

\item {\bf After a successful reduction of a data set, I changed this
    data set in some way (e.g. modified or removed some files, or
    changed the rules of the Data Organizer). When I restart Reflex,
    the Data Set Chooser correctly displays my new data set, but marks
    it as ``reduced ok'', even though it was never reduced before. What
    does this mean?}

  The labels in the column ``Reduced'' of the Data Set Chooser mark each
  dataset with ``OK'', ``Failed'' or ``-''. These labels indicate whether a
  data set has previously successfully been reduced at least once, all
  previous reductions failed, or a reduction has never been tried respectively.
  Data sets are identified by their name, which is derived from the first
  science file within the data set. As long as the data set name is
  preserved (i.e. the first science file in a data set has not changed),
  the Data Organizer will consider it to be the same data set. The Data
  Organizer recognizes any previous reductions of data sets it considers
  to be the same as the current one, and labels the current data set
  with ``OK'' if any of them was successful, even if the previously
  reduced data set differs from the current one.

  Note that the Product Explorer will list all the previous reductions of a
  particular data set only at the end of the reduction. This list might
  include successful and/or unsuccessful reduction runs with different
  parameters, or in your case with different input files.  The important
  fact is that these are all reductions of data sets with the same first
  raw science file. By browsing through all reductions of a particular
  raw science file, the users can choose the one they want to use.

   \item {\bf Where are my intermediate pipeline products?}
   Intermediate pipeline products are stored in the directory {\tt
   \verb|<|TMP\_PRODUCTS\_DIR\verb|>|} (defined on the workflow
 canvas, under Setup Directories)
   and organised further in directories by pipeline recipe.

%\item {\bf I have many DataSets in my data directory. How can I reduce
%  them interactively without having to wait a long time between
%  interactive windows being displayed?}

%Reduce all the DataSets at once with the interactive windows disabled
%for all interactive actors. When this reduction has finished, you
%should re-enable the interactive windows that you require, and run the
%workflow again. The workflow will run in Lazy mode and no time will be
%spent on pipeline reductions, unless you specifically change a
%parameter in one of the interactive windows.
        
%      Note that Lazy mode will not work if the workflow parameter {\tt
%        EraseDirs} is set to {\tt true}.

   \item {\bf Can I use different sets of bias frames to calibrate my
          flat frames and science data?}
   Yes. In fact this is what is currently implemented in the workflow(s).
   Each file in a DataSet has a purpose attached to it (\cite{REFLEXMAN}).
   It is this purpose that is used by the workflow to send the correct
   set of bias frames to the recipes for flat frame combination and 
   science frame reduction, which may or may not be the same set of bias 
   frames in each case.

   \item {\bf Can I run {\tt Reflex} from the command line?}
   Yes, use the command:
\begin{verbatim}
esoreflex -n <workflow_path>/<workflow>.xml
\end{verbatim}
   The -n option will set all the different options for Kepler and the
   workflows to avoid opening any GUI elements (including
   pipeline interactive windows).

   It is possible to specify workflow variables (those that appear in
   the workflow canvas) in the command line. For instance, the raw
   data directory can be set with this command:
\begin{verbatim}
esoreflex -n -RAW_DATA_DIR <raw_data_path> \
              <workflow_path>/<workflow>.xml
\end{verbatim}
   You can see all the command line options with the command 
   {\tt esoreflex -h}.

   Note that this mode is not fully supported, and the user should be 
   aware that the path to the workflow must be absolute and even if no
   GUI elements are shown, it still requires a connection to the window manager.
        
   \item {\bf How can I add new actors to an existing workflow?}
   You can drag and drop the actors in the menu on the left of the {\tt Reflex} 
   canvas. Under {\tt Eso-reflex -> Workflow} you may find all the actors
   relevant for pipeline workflows, with the exception of the recipe executer.
   This actor must be manually instantiated using
   {\tt Tools -> Instantiate Component}. Fill in the ``Class name'' field with 
   {\tt org.eso.RecipeExecuter} and in the pop-up window choose the required 
   recipe from the pull-down menu. To connect the ports of the actor, click on
   the source port, holding down the left mouse button, and release the mouse
   button over the destination port. Please consult the Reflex User Manual
   (\cite{REFLEXMAN}) for more information.

   \item {\bf How can I broadcast a result to different subsequent actors?}
   If the output port is a multi-port (filled in white), then you may have
   several relations from the port. However, if the port is a single port
   (filled in black), then you may use the black diamond from the toolbar.
   Make a relation from the output port to the diamond. Then make relations 
   from the input ports to the diamond. Please note that you cannot click to 
   start a relation from the diamond itself. Please consult the Reflex User 
   Manual (\cite{REFLEXMAN}) for more information.

   \item {\bf How can I manually run the recipes executed by Reflex?}
   If a user wants to re-run a recipe on the command line he/she has to go to
   the appropriate reflex\_book\_keeping directory, which is generally 
   reflex\_book\_keeping/<workflow>/<recipe\_name>\_<number> 
   There, subdirectories exist with 
   the time stamp of the recipe execution (e.g. 2013-01-25T12:33:53.926/). 
   If the user wants to re-execute the most recent processing he/she should 
   go to the {\tt latest} directory and then execute the script 
   {\tt cmdline.sh}. Alternatively, to use a customized {\tt esorex} command
   the user can execute
   \begin{verbatim}
   ESOREX_CONFIG="INSTALL_DIR/etc/esorex.rc"
   PATH_TO/esorex --recipe-config=<recipe>.rc <recipe> data.sof 
   \end{verbatim}
   where INSTALL\_DIR is the directory where Reflex and the pipelines were
   installed. 

   If a user wants to re-execute on the command line a recipe that used
   a specific raw frame, the way to find the proper data.sof in the bookkeeping
   directory is via {\tt grep <raw\_file> */data.sof}. 
   Afterwards the procedure is the same 
   as before.

   If a recipe is re-executed with the command explained above, the
   products will appear in the directory from which the recipe
   is called, and not in the reflex\_tmp\_products or reflex\_end\_products
   directory, and they will not be renamed. This does not happen if you use
   the {\tt cmdline.sh} script.

\ifx\reflexinteract\undefined
\else

   \item {\bf If I enter ``-'' into an empty integer parameter of an interactive
   window it is automatically completed to ``-1''. Why? }

   The parameters are validated for correctness according to their type 
   (e.g. string, integer, float). In the case of an integer or float parameter 
   ``-'' alone is considered an invalid input and is therefore automatically
   completed to ``-1''.  This is part of the validation of input done
   by the WxPython library.

\fi

\item {\bf Can I reuse the bookkeeping directory created by previous
    versions of the pipeline?}

  In general no. In principle, it could be reused if no major changes
  were made to the pipeline. However there are situations in which a
  previously created bookkeeping directory will cause problems due to
  pipeline versions incompatibility. This is especially true if the
  parameters of the pipeline recipes have changed. In that case,
  please remove the bookkeeping directory completely.

\item {\bf How to insert negative values into a textbox?}

  Due to a bug in wxPython, the GUI might appear to freeze when attempting to
  enter a negative number in a parameter's value textbox. This can be worked
  around by navigating away to a different control in the GUI with a mouse
  click, and then navigating back to the original textbox. Once focus is back
  on the original textbox the contents should be selected and it should be
  possible to replace it with a valid value, by typing it in and pressing the
  enter key.

\item {\bf I've updated my Reflex installation and when I run esoreflex the process aborts. How can I fix this problem?}
  
  As indicated in Section \ref{Sec:Software_Installation}, in case of major or
  minor (affecting the first two digit numbers) Reflex upgrades, the user
  should erase the
 \verb+$HOME/KeplerData+, \verb+$HOME/.kepler+ directories if present,
 to prevent possible aborts (i.e. a hard crash) of the esoreflex process.

\item {\bf How can include my analysis scripts and algorithms into the 
workflow?}

EsoReflex is capable of executing any user-provided script, if
properly interfaced. The most convenient way to do it is through
the Python actor. Please consult the tutorial on how to
insert Python scripts into a workflow available here:
\url{www.eso.org/sci/data-processing/Python_and_esoreflex.pdf}

\end{itemize}
