%\section{Introduction And Scope}
\section{Introduction to \Reflex}
This document is a tutorial designed to enable the user to to reduce
his/her data with the ESO pipeline run under an user-friendly
environmet, called {\tt EsoReflex}, concentrating on high-level issues
such as data reduction quality and signal-to-noise (S/N) optimisation.




{\tt EsoReflex} is the ESO Recipe Flexible Execution 
Workbench, an environment to run ESO VLT pipelines which employs a workflow 
engine 
% No response from kepler-project.org
%(Kepler\footnote{\http{kepler-project.org}}) 
to provide a real-time
visual representation of a data reduction cascade, called a workflow,
which can be easily understood by most astronomers. 
The basic philosophy and concepts of Reflex have been discussed by 
\href{https://ui.adsabs.harvard.edu/abs/2013A\%26A...559A..96F/abstract}{Freudling et al. (2013A\&A...559A..96F).}
Please reference this article if you use Reflex in a scientific publication.

Reflex and the data reduction workflows have been developed by ESO and 
instrument consortia and they are fully supported. If you have any issue, 
please have a look to
\href{https://support.eso.org}{https://support.eso.org} to see if this
has been reported before or 
\href{https://support.eso.org/new-ticket}{open a ticket}
for further support.


A workflow accepts science and calibration data, as downloaded from the
archive using the CalSelector
tool\footnote{\http{www.eso.org/sci/archive/calselectorInfo.html}} 
(with associated raw calibrations) and
organises them into DataSets, where each DataSet contains one
science object observation (possibly consisting of several science
files) and all associated raw and static calibrations required for a
successful data reduction. The data organisation process is fully
automatic, which is a major time-saving feature provided by the
software. The DataSets selected by the user for reduction are fed
to the workflow which executes the relevant pipeline recipes (or
stages) in the correct order.
%, providing optional user interactivity at
%key data reduction points with the aim of enabling the iteration of
%certain recipes in order to obtain better results. 
Full control of the
various recipe parameters is available within the workflow, and the
workflow deals automatically with optional recipe inputs via built-in
conditional branches. Additionally, the workflow stores the reduced
final data products in a logically organised directory structure 
employing user-configurable file names.

