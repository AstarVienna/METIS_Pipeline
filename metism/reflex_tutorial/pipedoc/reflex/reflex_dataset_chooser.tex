

\subsubsection{DataSetChooser}
 \label{sec:dataset_chooser}


 The {\tt DataSetChooser} displays the DataSets available in the
 ``Select Data Sets'' window, activating vertical and horizontal
 scroll bars if necessary (Fig.~\ref{fig:reflex_select_data_sets}). 

 Some properties of the DataSets are displayed: the name, the number
 of files, a flag indicating if it has been successfully reduced (a
 green OK), if the reduction attempts have failed or were aborted (a
 red FAILED), or if it is a new dataset (a black "-"). The column
 "Descriptions" lists user-provided descriptions (see below), other
 columns indicate the instrument set-up and a link to the night log.


 Sometimes you will want to reduce a subset of these DataSets rather
 than all DataSets, and for this you may individually select (or
 de-select) DataSets for processing using the tick boxes in the first
 column, and the buttons \fbox{\tt Deselect All} and \fbox{\tt Select
   Complete} at the bottom, or configure the ``Filter'' field at the
 bottom left. Available filter options are: "New" (datasets not
 previously reduced will be selected), "Reduced" (datasets previously
 reduced will be selected), "All" (all datasets will be selected), and
 "Failed" (dataset with a failed or aborted reduction will be
 selected).


 You may also highlight a single DataSet in blue by clicking on the
 relevant line. If you subsequently click on \fbox{\tt Inspect
   Highlighted}, then a ``Select Frames'' window will appear that
 lists the set of files that make up the highlighted DataSet including
 the full filename\footnote{keep the mouse pointer on the file
   name to visualize the full path name.}, the file category
 (derived from the FITS header), and a selection tick box in the right
 column. The tick boxes allow you to edit the set of files in the
 DataSet which is useful if it is known that a certain calibration
 frame is of poor quality (e.g: a poor raw flat-field frame).  The
 list of files in the DataSet may also be saved to disk as an {\tt
   ASCII} file by clicking on \fbox{\tt Save As} and using the file
 browser that appears.

  By clicking on the line corresponding to a particular file in the
  ``Select Frames'' window, the file will be highlighted in blue, and
  the file FITS header will be displayed in the text box on the right,
  allowing a quick inspection of useful header keywords.  If you then
  click on \fbox{\tt Inspect}, the workflow will open the file in the
  selected FITS viewer application defined by the workflow parameter
  {\tt FITS\_VIEWER}.

To exit from the ``Select Frames'' window, click \fbox{\tt Continue}.

To add a description of the reduction, press the button \fbox{\tt ...}
associated with the field "Add description to the current execution
  of the workflow" at the bottom right of the Select Dataset Window; a
  pop up window will appear. Enter the desired description (e.g. "My
  first reduction attempt") and then press \fbox{\tt OK}.
  In this way, all the datasets reduced in this execution, will be
  flagged with the input description. Description flags can be
  visualized in the {\tt SelectFrames} window and in the {\tt ProductExplorer},
  and they can be used to identify different reduction strategies.


To exit from the ``Select DataSets'' window, click either \fbox{\tt Continue}
in order to continue with the workflow reduction, or \fbox{\tt Stop} in
order to stop the workflow.


